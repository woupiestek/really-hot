\documentclass{tac}
\usepackage{amssymb, amsmath, stmaryrd}
\usepackage[backend=bibtex,citestyle=authoryear-icomp]{biblatex}
\usepackage[all]{xy}
\usepackage{url}

\title{A small object argument for incomplete categories}
\author{Wouter Pieter Stekelenburg}
\copyrightyear{2015,2016,2017,2018}
%\address{Faculty of Mathematics, Informatics and Mechanics\\
%University of Warsaw\\
%Banacha 2\\
%02-097 Warszawa\\
%Poland}
\eaddress{w.p.stekelenburg@gmail.com}
\keywords{realizability, simplicial homotopy, fibrant objects}
\amsclass{03D80, 18G30, 18G55}
\addbibresource{realizability}

%structural helpers
\newcommand\hide[1]{}
\newcommand\keyword[1]{\emph{#1}\label{#1}}

%symbols
\newcommand\cat\mathcal
\newcommand\icat\mathsf
\newcommand\of{\mathord:}
\newcommand\set[1]{\left\{#1\right\}}
\newcommand\dual{^\circ}
\newcommand\nno{\mathbb N}


\mathrmdef{Ar}
\mathrmdef{Ob}
\mathrmdef{id}
\mathrmdef{dom}
\mathrmdef{cod}
\newcommand\colim{\mathop{\mathrm{colim}}}
\mathrmdef{deco}
\newcommand\tuplet[1]{\left\langle #1 \right\rangle}
\newcommand\disc{_{\rm disc}}

\newcommand\pushed{\ar@{}[ul]|(.2)\ulcorner}

\newcommand\asif[1]{`#1'}

\begin{document}
\begin{abstract}
The small object argument %cite
is modified to work in categories like the effective topos %cite
that lack infinite colimits.
\end{abstract}

\maketitle

\section{Motivation}

\section{Small object arguments for complete categories}
As Garner explains %cite
the small object argument can be split up in three steps: the \emph{density comonad}, the \emph{pushout} and the \emph{transfinite composition}.

\paragraph{Density comonad}
\begin{definition}
For any functor $F\of \cat C\to \cat D$ the \keyword{codensity comonad} 
is the left Kan extension of $F$ along itself.
\end{definition}

The density comonad gives a best approximation of an object of the domain $\cat D$ in terms of the objects in the image of the functor $F$.
 
\begin{lemma} Let $\tuplet{D\of\cat D\to\cat D,\epsilon\of D\to \id_{
\cat C},\delta\of D \to DD}$ be the density comonad of $F\of\cat C\to\cat D$. For each object $Y$ of $\cat D$, $DY$ is the colimit of $f\mapsto\dom(f)\of (F/\cat D)\to \cat D$. Hence if $\cat D$ is complete, the density comonad exists whenever the domain $\cat C$ of $F$ is small.
\end{lemma}

A family of generic left morphisms $\set{c_i\of A_i\to B_i|i\in I}$ in a category $\cat A$ is essentially a functor from the discrete category $I\disc$ to the arrow category $\cat A/\cat A$. Due to the lack of morphisms in $I\disc$, the density comonad yields the weighted coproduct $\sum_{i\of I} \hom(a_i,f)\times a_i$, which is a left morphism of course.

\paragraph{Pushout}
If $\cat A$ has all pushouts, comonads on the arrow category $\cat A/\cat A$ become functorial factorization systems.

\[ \xymatrix{
DX\ar[d]_{Df}\ar[r]^{\epsilon (X)} & X\ar[dr]^f\ar[d]_{l(f)} \\
DY \ar[r]\ar@/_1ex/[rr]_{\epsilon (Y)}& \bullet\pushed \ar[r]^(.4){r(f)} & Y
}\]

Let $f\of X\to Y$, then there is a counit $\epsilon(f) = (\epsilon(X),\epsilon(Y))\of Df\to f$. The left factor $l(f)$ is the pushout of $Df$ along $\epsilon(X)$ and $r(f)$ is the factorization of the cone consisting of $f$ and $\epsilon(Y)$ through the colimiting cone.

While the left morphisms here have the left lifting property there is no guarantee that the right ones do however.

\paragraph{Transfinite composition}
\begin{definition}
An object $S$ of a category $\cat C$ is small, if for some regular cardinal $\kappa$, $\hom(S,\cdot)$ preserves $\kappa$-directed colimits.
\end{definition}

Effectively, this means $\hom(S,\cdot)$ preserves transfinite colimits of shape $\kappa$.

Repeatedly factorize the rightmost morphism up to any ordinal, taking the limit over the chain to reach each limit ordinal.
\begin{align*}
l^0(f) &= \id_X\\
r^0(f) &= f\\
l^{n+1}(f) &= l(r^n(f))\circ l^n(f)\\
r^{n+1}(f) &= r(r^n(f))\\
l^\lambda(f) &= \colim_{n<\lambda} l^n(f) \\
r^\lambda(f) &= \lim_{n<\lambda} r^n(f)
\end{align*}
This result in the factorization $f = r^\kappa(f)\circ l^\kappa(f)$. 

\begin{lemma} The morphism $r^\kappa(f)$ has the right lifting property with respect to all generic left morphisms.
\end{lemma}

\begin{proof}
For each generic left morphism $c\of A \to B$ and each $(a,b)\of c\to r^\kappa(f)$, the morphism $a$ factors through $\dom(r^n(f)) = \cod(l^n(f))$ for some $n<\kappa$ because $A$ is small and so $\hom(A,\cod(l^\kappa(f))) =\colim_{n<\lambda} \hom(A,\cod(l^n(f)))$.
Hence for some $a'\of A\to \dom(r^n(f))$, $(a',b)\of c\to r^n(f)$.

Since $l(r^n(f))$ is the pushout of the colimit of $c'\to r^n(f)$ for all generic left morphisms $c'$, $(a',b)$ factors through $l(r^n(f))$ as $(a',b')$ for some $b'\of B \to \cod(l^{n+1}(f))$. This induces a lift for $c$ against $r^{n+1}(f)$ and hence against $r^\kappa(f)$.

\[\xymatrix{
&& A\ar[r]^{c}\ar@/^2ex/[dr]^(.7)a\ar@{.>}@/_2ex/[dl]_(.7){a'} & B\ar@/^2ex/[dr]^(.7)b\ar@{.>}@/_2ex/[dl]_(.7){b'} \\
X \ar[r]^{l^n(f)} \ar@/_1ex/[rrr]_{l^{\kappa}(f)} \ar@/_4ex/[rrrr]_{f} & \bullet \ar[r]^{l(r^n(f))}  & \bullet \ar[r]\ar@/_1ex/[rr]_(.3){r^{n+1}(f)} & \bullet \ar[r]^{r^\kappa(f)} & Y
}\]
\end{proof}

\paragraph{Conclusion}
The small object arguments shows that in a cocomplete category, an family of morphism with small domains induces a weak factorization system. Garner %cite 
strengthens this into an algebraic factorization system.

\section{Dealing with incompleteness}
Each step of the small object argument assumes the presence of certain colimits, that realizability categories not always have.

The alternative presented here presumes that the ambient category $\cat A$ has other structure to replace the missing colimits. In particular, if $\cat A$ is locally Cartesian closed, then there is a notion of a colimit over an \emph{internal diagram} \ref{Internal diagram}, and $\cat A$ trivially has coproducts over internal families of objects.

Internal diagrams have density comonads if $\cat A$ also has equalizers, just like `external' functors with small domains do in cocomplete categories. So an internal diagram in the category $\cat A/\cat A$ of arrows of $\cat A$ can replace the family of generic left morphisms.

There is a kind of internal diagram from which it is relatively easy to get an internal factorization system, the \emph{presaturated} \ref{presaturated} ones. %ref
Presaturated diagrams can be constructed from others in many interesting cases. %ref

\paragraph{Internal diagrams}
\begin{definition} Let $\cat A$ be a category with finite limits category and let an internal category $\icat C = \tuplet{C_0,C_1,\id\of C_0\to C_1,\dom\of C_1\to C_0,\cod\of C_1\to C_0,\circ\of C_1\times_{C_0}C_1\to C_1}$ be an internal category of $\cat A$. 
The category induces a monad $M$ on the slice category $\cat A/C_0$, where where:
\begin{align*}
M(f) &= \lambda\set{\tuplet{c,x}| \dom(c)=f(x)}\cod(c) \\
\eta(x) &= \tuplet{\id(f(x)),x}\\
\mu(\tuplet{a,\tuplet{b,x}}) &= \tuplet{a\circ b,x}
\end{align*}
\keyword{Internal diagrams} are algebras for this monad.
\end{definition}

\begin{example}
In cocomplete categories $\cat A$ with finite limits, diagrams can be constructed from functors $F\of\cat C\to\cat A$ if $\cat C$ is a small category. The external category $\cat C$ has an internal counterpart $\icat C$ whose objects of objects and arrows are the result of taking coproducts of the terminal object over the objects and arrows of $\cat C$. The internal diagrams for $\icat C$ and the functors $\cat C\to \cat A$ are equivalent categories.
\end{example}

If the ambient category is locally Cartesian closed and has stable coequalizers, then the density comonad can be internalized as well.

\mathsfdef{diagrams}
\mathsfdef{sheaves}
\begin{definition}
For any internal category $\icat C$ of $\cat A$ 
let $\diagrams(\icat C)$ be the category of diagrams and diagram-homomorphisms over $\icat C$ and let $\sheaves(\icat C) = \diagrams(\icat C\dual)$ where $\icat C\dual$ is the dual category of $\icat C$. In the category of functors $\sheaves(\icat C)\times\diagrams(\icat C) \to\cat A$ let $\otimes$ be the codomain of the coequalizer of 
$p,q\of A\to B \of $ where:
\begin{align*}
A\tuplet{f,g} &= \set{\tuplet{x\of \dom(f),y\of C_1,z\of\dom(g)}|f(x) = \cod(y), \dom(y) = g(z)}\\
B\tuplet{f,g} &= \set{\tuplet{x\of \dom(f),y\of\dom(g)}|f(x) = g(z)}\\
p\tuplet{x,y,z}&=\tuplet{x\cdot y,z}\\
q\tuplet{x,y,z}&=\tuplet{x,y\cdot z}
\end{align*}
I.e. $f\otimes g$ the the quotient of of pairs $\tuplet{x\of\dom(f),y\of\dom(g)}$ by the least equivalence relation that equates $\tuplet{x\cdot y,z}$ and $\tuplet{x, y\cdot z}$ for each morphism $y$ of $\icat C$.

Thanks to local Cartesian closure each diagram $g$ induces a functor $g\Rightarrow  \of \cat A\to \sheaves(\icat C)$.
\begin{align*}
\dom(g\Rightarrow A) &= \set{\tuplet{y\of C_0,m\of\set{x | g(x) = y}\to A}}\\
(g\Rightarrow A)\tuplet{c,m} &= c\\
 \tuplet{\cod(h),m}\cdot h &= \tuplet{\dom(h),\lambda x.m(h\cdot x)}
\end{align*}

For each diagram $g$ the functor $g\Rightarrow$ is right adjoint to $\otimes g$. 
The \emph{density comonad} for $g$ is the comonad induced by the adjunction $\otimes g\dashv g\Rightarrow$.
\end{definition}

Intuitively, $\sheaves(\icat C)$ is both an internal colimit completion and a category of generalized objects of the internal category of $\icat C$ . The functor $\otimes g\of \sheaves(\icat C) \to \cat A$ externalizes the diagram. The density monad is the left Kan extension of $\otimes g$ along itself, which links this construction to the density comonad for a functor.

\hide{
we will need to revisit the concepts and notions throughout this document

some guidelines:
- the greater the scope, the more descriptive the identifier
- keep refactoring
- define small generic components and compose them 
}

An internal diagram in $\cat A/\cat A$ is essentially a morphism $c\of A\to B$ of internal diagrams for some internal category $\icat I$ in $\cat A$. The functor $\otimes c\of \sheaves(\icat I)\to \cat A/\cat A$ generates morphisms from the diagram, as an alternative to infinite coproducts.

\paragraph{Presaturation}
The strategy to getting a factorization system from an internal diagram is to first only consider diagrams that almost directly generate all the left morphisms.

\begin{definition}
A diagram $c\of A\to B$ over $\icat I$ in the arrow category $\cat A/\cat A$ is \keyword{presaturated} if it has the following two properties.
\begin{enumerate}
\item for each $f$ of $\sheaves(\icat I)$ and pushout $\tuplet{a,b}\of f\otimes c\to g$ there is a $\tuplet{a',b'}\of f\to g'$ in $\sheaves(\icat I)$ and an isomorphism $\tuplet{d,e}\of g'\otimes c\to g$ such that $\tuplet{d,e}\circ \tuplet{a',b'}\otimes c = \tuplet{a,b}$.
\[\xymatrix{
\bullet \ar[r]_{a'\otimes c}\ar[d]_{f\otimes c}\ar@/^1ex/[rr]^a &
\bullet \ar[r]_\sim \ar[d]^{h\otimes c} & \bullet\ar[d]^g \\
\bullet \ar[r]^{b'\otimes c} \ar@/_1ex/[rr]_b &
\bullet\pushed \ar[r]^\sim & \bullet
}\]
\item for each pair $f$ and $g$ of $\sheaves(\icat I)$ such that there is an isomorphism $h\of\cod(f\otimes c)\to \dom(g\otimes c)$, there there is an $\tuplet{a',b'}\of f\to k$ of $\sheaves(\icat I)$ such that $a'\otimes c$ is an isomorphism and $(k\otimes c)\circ (a'\otimes c)$ is isomorphic to $(g\otimes c)\circ h \circ (f\otimes c)$.
\[\xymatrix{
\bullet\ar[d]_{a'\otimes c}\ar[r]^{f\otimes c} & \bullet\ar[r]^h\ar[drr]_{b'\otimes c} & \bullet\ar[r]^{g\otimes c} & \bullet\ar[d]^\sim\\
\bullet\ar[rrr]_{k\otimes c} &&& \bullet 
}\]
\end{enumerate}
\end{definition}

\begin{lemma} The density comonad for a presaturated diagram is an algebraic factorization system. \hide{define this somewhere, or refer to literature}
\end{lemma}

\begin{proof}
Let $c\of A\to B$ be a presaturated diagram over $\icat I$ in the arrow category $\cat A/\cat A$. 
For each morphism $f\of X\to Y$ the counit $\epsilon_f\of(c\Rightarrow f)\times c\to f$ of the codensity monad factors as a composition following a pushout, which is of the form $h\otimes c$ by presaturation.
\[\xymatrix{
\bullet\ar[d]_{(c\Rightarrow f)\otimes c} \ar[r]^{\eta_{f,0}} & \bullet \ar[d]^{h\otimes c} \ar[r]^\sim & \bullet\ar[d]^f\\
\bullet \ar[r]\ar@/_1ex/[rr]_{\eta_{f,1}} & \bullet\pushed\ar[r] & \bullet
}\]
The counit is terminal in $(\otimes c)/f$, however, forcing $h \simeq (c\Rightarrow f)$ in a way that commutes with the counit.

Hence $f = l(f)\circ r(f)$ where $l(f)$ is isomorphic to $(c\Rightarrow f)\times c$ and $r(f)$ is isomorphic to $\eta_{f,1}$
The general lifting problem $\tuplet{a,b}\of g\otimes c\to r(f)$ first reduces to the case where $a$ is an isomorphism thanks to the pushout property of presaturated diagrams. Then, the composition property gets us a $k$ such that the composition of $l(f)$ with the pushout of $g\otimes c$ essentially is $k\circ c$.
\[\xymatrix{
&& \bullet\ar[r]^{g\otimes c}\ar[d]\ar[dl] & 
\bullet\ar[d]\\ 
\bullet\ar@/_1ex/[dd]_\id\ar[d]^{a'\otimes c}\ar[r]^{l(f)} & 
\bullet\ar[r]^\sim\ar[drr]^(.7){b'\otimes c}\ar[ddrr]_(.7){r(f)} & 
\bullet\ar[r]_{g'\otimes c} & 
\bullet\pushed\ar[d]^\sim\\
\bullet\ar[d]^\sim\ar[rrr]_{k\otimes c} &&& \bullet\ar[d]^\sim \\
\bullet\ar[rrr]_f &&& \bullet 
}\]
The finality of $c\Rightarrow f$ makes it a retract of $k$ and the unique morphism $k\to (c\Rightarrow f)$ is the solution to the lifting problem.
\end{proof}

\paragraph{Conclusion}
The problem of generating a factorization system from a family of generic left morphisms can be reformulated as the problem extending a given diagram to a presaturated one. We will show that new problem has solutions that don't require infinite colimits from the ambient category.

\section{Presaturating diagrams}
%here


% dense is enough? dense families don't have to be closed under composition?
At this point, assume that the ambient category $\cat A$ has a natural number object $\nno$ and a dense diagram $S$. In a locally Cartesian closed Category, a natural number objects permit the construction of many other inductive objects, like free monoids, which can help to substitute transfinite compositions. The existence of a dense diagram replaces another useful property: \emph{local presentability}. Rather than having each object represented as a small colimit of small objects, we let them be 





%domain of left morphism are memebrs of S!?





\hide{the difficulty is going to be showing that the construction delivers a presaturated diagram, and that may show presaturated diagram to be useless abstractions.}




% 1. replace locally presentable with 'dense family', 'small morphisms' etc.
% 2. generate the morphisma and objects
% 3. proof filteredness

\section{Getting the ambient category in shape}
%exact completions etc.
\hide{
- nno's cycles & horns
- W-types
-	etc.
}

\section{The realizability example}
\hide{
- simplicial assemblies & modest sets.
- exact completions for pushouts.
- exact completions as examples of homotopy categories.
}

\end{document}