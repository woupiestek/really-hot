\documentclass{tac}
\usepackage{amssymb, amsmath, stmaryrd}
\usepackage[backend=bibtex,citestyle=authoryear-icomp]{biblatex}
\usepackage[all]{xy}
\usepackage{url}

\title{A small object argument for incomplete categories}
\author{Wouter Pieter Stekelenburg}
\copyrightyear{2015,2016,2017,2018}
%\address{Faculty of Mathematics, Informatics and Mechanics\\
%University of Warsaw\\
%Banacha 2\\
%02-097 Warszawa\\
%Poland}
\eaddress{w.p.stekelenburg@gmail.com}
\keywords{realizability, simplicial homotopy, fibrant objects}
\amsclass{03D80, 18G30, 18G55}
\addbibresource{realizability}

%structural helpers
\newcommand\hide[1]{}
\newcommand\keyword[1]{\emph{#1}\label{#1}}

%symbols
\newcommand\cat\mathcal
\newcommand\of{\mathord:}
\newcommand\set[1]{\left\{#1\right\}}


\mathrmdef{Ar}
\mathrmdef{Ob}
\mathrmdef{id}
\mathrmdef{dom}
\mathrmdef{cod}
\newcommand\colim{\mathop{\mathrm{colim}}}
\mathrmdef{deco}
\newcommand\tuplet[1]{\left\langle #1 \right\rangle}
\newcommand\disc{_{\rm disc}}

\newcommand\pushed{\ar@{}[ul]|(.2)\ulcorner}

\begin{document}
\begin{abstract}
The small object argument %cite
is modified to work in categories like the effective topos %cite
that lack infinite colimits.
\end{abstract}

\maketitle

\section{Motivation}

\section{Small object arguments for complete categories}
As Garner explains %cite
the small object argument can be split up in three steps: the \emph{density comonad}, the \emph{pushout} and the \emph{transfinite composition}.

\paragraph{Density comonad}
\begin{definition}
For any functor $F\of \cat C\to \cat D$ the \keyword{codensity comonad} 
is the left Kan extension of $F$ along itself.
\end{definition}

The density comonad gives a best approximation of an object of the domain $\cat D$ in terms of the objects in the image of the functor $F$.
 
\begin{lemma} Let $\tuplet{D\of\cat D\to\cat D,\epsilon\of D\to \id_{
\cat C},\delta\of D \to DD}$ be the density comonad of $F\of\cat C\to\cat D$. For each object $Y$ of $\cat D$, $DY$ is the colimit of $f\mapsto\dom(f)\of (F/\cat D)\to \cat D$. Hence if $\cat D$ is complete, the density comonad exists whenever the domain $\cat C$ of $F$ is small.
\end{lemma}

A family of generic left morphisms $\set{c\of A_i\to B_i|i\in I}$ in a category $\cat A$ is essentially a functor from the discrete category $I\disc$ to the arrow category $\cat A/\cat A$. Due to the lack of morphisms in $I\disc$, the density comonad yields the weighted coproduct $\sum_{i\of I} \hom(a_i,f)\times a_i$, which is a left morphism of course.

\paragraph{Pushout}
If $\cat A$ has all pushouts, comonads on the arrow category $\cat A/\cat A$ become functorial factorization systems.

\[ \xymatrix{
DX\ar[d]_{Df}\ar[r]^{\epsilon (X)} & X\ar[dr]^f\ar[d]_{l(f)} \\
DY \ar[r]\ar@/_1ex/[rr]_{\epsilon (Y)}& \bullet\pushed \ar[r]^(.4){r(f)} & Y
}\]

Let $f\of X\to Y$, then there is a counit $\epsilon(f) = (\epsilon(X),\epsilon(Y))\of Df\to f$. The left factor $l(f)$ is the pushout of $Df$ along $\epsilon(X)$ and $r(f)$ is the factorization of the cone consisting of $f$ and $\epsilon(Y)$ through the colimiting cone.

While the left morphisms here have the left lifting property there is no guarantee that the right ones do however.

\paragraph{Transfinite composition}
\begin{definition}
An object $S$ of a category $\cat C$ is small, if for some regular cardinal $\kappa$, $\hom(S,\cdot)$ preserves $\kappa$-directed colimits.
\end{definition}

Effectively, this means $\hom(S,\cdot)$ preserves transfinite colimits of shape $\kappa$.

Repeatedly factorize the rightmost morphism up to any ordinal, taking the limit over the chain to reach each limit ordinal.
\begin{align*}
l^0(f) &= \id_X\\
r^0(f) &= f\\
l^{n+1}(f) &= l(r^n(f))\circ l^n(f)\\
r^{n+1}(f) &= r(r^n(f))\\
l^\lambda(f) &= \colim_{n<\lambda} l^n(f) \\
r^\lambda(f) &= \lim_{n<\lambda} r^n(f)
\end{align*}
This result in the factorization $f = r^\kappa(f)\circ l^\kappa(f)$. 

\begin{lemma} The morphism $r^\kappa(f)$ has the right lifting property with respect to all generic left morphisms.
\end{lemma}

\begin{proof}
For each generic left morphism $c\of A \to B$ and each $(a,b)\of c\to r^\kappa(f)$, the morphism $a$ factors through $\dom(r^n(f)) = \cod(l^n(f))$ for some $n<\kappa$ because $A$ is small and so $\hom(A,\cod(l^\kappa(f))) =\colim_{n<\lambda} \hom(A,\cod(l^n(f)))$.
Hence for some $a'\of A\to \dom(r^n(f))$, $(a',b)\of c\to r^n(f)$.

Since $l(r^n(f))$ is the pushout of the colimit of $c'\to r^n(f)$ for all generic left morphisms $c'$, $(a',b)$ factors through $l(r^n(f))$ as $(a',b')$ for some $b'\of B \to \cod(l^{n+1}(f))$. This induces a lift for $c$ against $r^{n+1}(f)$ and hence against $r^\kappa(f)$.

\[\xymatrix{
&& A\ar[r]^{c}\ar@/^2ex/[dr]^(.7)a\ar@{.>}@/_2ex/[dl]_(.7){a'} & B\ar@/^2ex/[dr]^(.7)b\ar@{.>}@/_2ex/[dl]_(.7){b'} \\
X \ar[r]^{l^n(f)} \ar@/_1ex/[rrr]_{l^{\kappa}(f)} \ar@/_4ex/[rrrr]_{f} & \bullet \ar[r]^{l(r^n(f))}  & \bullet \ar[r]\ar@/_1ex/[rr]_(.3){r^{n+1}(f)} & \bullet \ar[r]^{r^\kappa(f)} & Y
}\]
\end{proof}

\paragraph{Conclusion}
The small object arguments shows that in a cocomplete category, an family of morphism with small domains induces a weak factorization system. Garner %cite 
strengthens this into an algebraic factorization system.

\section{Dealing with incompleteness}
Each step of the small object arguments assumes the presence of certain colimits, that realizability categories not always have. There are alternatives that will work than to other structure that these categories have.
 \emph{Internal diagrams} have density comonads in locally cartesian closed categories. %is that all?
 Missing pushouts can be added through \emph{exact completions} in many cases.
The hardest part is the transfinite composition, however and our solution comes down to ensuring that the internal diagram is so complete, that its codensity monad does not require any further steps.

\paragraph{Internal diagrams}
\begin{definition} Let $\cat A$ be a locally cartesian closed category and let an internal category $C = \tuplet{C_0,C_1,\id\of C_0\to C_1,\dom\of C_1\to C_0,\cod\of C_1\to C_0,\circ\of C_1\times_{C_0}C_1\to C_1}$ be an internal category of $\cat A$. An \emph{internal diagram} $F$ is a tuple $\tuplet{f\of D\to C_0, \cdot\of C_1\times_{C_0} D\to D}$ where %here
\end{definition}

\begin{example}
In a cocomplete categories $\cat A$ with finite limits, diagrams can be constructed from functors $F\of\cat C\to\cat A$ where $\cat C$ is a small category.
\end{example}



\hide{
collect requirements:
- lccc for the internalized density comonads
- pushouts
- internalized smallness
}

\section{Smallnesses}
\hide{
- nno's cycles & horns
- W-types
-	etc.
}
\section{The realizability example}
\hide{
- simplicial assemblies & modest sets.
- exact completions for pushouts.
- exact completions as examples of homotopy categories.
}

\end{document}