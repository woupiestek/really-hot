\documentclass[csh.tex]{subfiles}
\begin{document}

\section{Descent}
This purpose of this section is to replace arguments based on \emph{minimal fibrations} in simplicial homotopy theory, in particular those that are related to homotopy type theory. We assume that there is a class $M$ of \emph{modest fibrations} in $\ambient\s$ with the following properties.
\begin{itemize}
\item $M$ is closed under pullbacks along arbitrary morphisms.
\item $M$ is closed under composition.
\item $M$ is closed under fibred exponentiation. This means that if $f:X\to Z$ belongs to $M$ and $g:Y\to Z$ is an arbitrary morphism, the fibred exponential $f^g_Z\of(\product{z\of Z}{X_z^{Y_z}})\to Z$ is in $M$ too.

This requirement may give us more limits than we need, but the intended model of modest morphisms in $\Asm\ex$ satisfies it.
%TODO: investigate which exponentials we actually need.

\item $M$ contains all \emph{regular} monomorphisms--a monomorphism is \emph{regular} is it is the equalizer of some parallel pair.
\end{itemize} 

The interesting case is where $M$ has the following structure in addition to the properties above.

\begin{definition} A \keyword{universal modest fibration} is a modest fibration $\uni\of U\to V$ such that every modest fibration $f\of X\to Y$ is the pullback of $\uni$ along a unique morphism $\chi(f)\of Y\to V$.
\end{definition}

The following fact makes the universal modest fibration a potential model of homotopy type theory.

\begin{theorem} If $\uni\of U\to V$ is a universal modest fibration, then $V$ is fibrant. \label{fibrant universe} \end{theorem}

\begin{proof} For each horn inclusion $h\of \horn_k[n]\to\simplex[n]$ and each pair of $v\of \horn_k[n]\to V$, the modest fibration $v\ri(\uni)\of \horn_k[n]\times_V U \to \horn_k[n]$ descends along $h$ to form a modest fibration $D(v\ri(\uni))\of \bullet\to\simplex[n]$ by lemma \ref{descent}. There is a $\chi(D(v\ri(\uni)))\of\simplex[n]\to V$ and $\chi(h_*(v\ri(\uni)))\circ h$ equals $v$ by definition \ref{universal modest fibration}.
\end{proof}

\begin{lemma}[Descent]
For each horn inclusion $h\of \horn_k[n]\to\simplex$ and each fibration $f\of X\to \horn_k[n]$ there is a fibration $Df\of Y\to\simplex[n]$ such that $f$ is the pullback of $Df$ along $h$.
\label{descent}
\end{lemma}

\begin{proof} There is no horn inclusion for $n=0$. In the case $n=1$, there are two maps $1\to\simplex[1]$. In both case $Df=\pi_0\of \simplex[1]\times X \to\simplex[1]$ suffices. For all the cases were $n>0$ we use the following construction.

The functor $D$ is defined to have the following relation with a functor $K\of\simCat/[n]\to\ambient\s/\horn_k[n]$.
\[ \ambient\s/\simplex[n](\simplex(\xi), Df)\simeq \ambient\s/\horn_k[n](K(\xi), f) \]
This is possible because $\ambient\s/\simplex[n]$ is equivalent to the category of presheaves over $\simCat/[n]$ and the Yoneda lemma applies to those presheaves. %lemma?
The relation of $D$ and $K$ allows us to reduce the problem of lifting a horn against $Df$ to the problem of lifting some finite colimit of objects in the image of $K$ against $f$. These finite colimits have the required left lifting property by lemma \ref{left lifting property}.
\end{proof}

The rest of this section works out the definition and the properties of $K$.

%The functor K
\begin{definition}
The following defines the functor $K\of\simCat/[n]\to\ambient\s/\horn_k[n]$.
\begin{enumerate}
\item A function $\xi\of[m]\to[n]$ cuts $[m]$ into $n+1$ posets $\xi_j = \set{i\of[m]|\xi(i)=j}$. 
\item Let $\norm \xi$ be the number of elements of the product $\product{i\of ([n]-\set k)}{\xi_i}$. 
\item Define $K_0(\xi)\of [m+\norm\xi]\to [n]$ as follows.
\[ 
	K_0(\xi)(i) = \left\{
		\begin{array}{cc}
			\xi(i) & \xi(i)<k \\
			k & \xi(i-\norm\xi)\leq k \leq \xi(i)\\
			\xi(i-\norm\xi) & k<\xi(i-\norm\xi)
		\end{array}
	\right.
\]
\item Let $\ka(\xi)\of\product{i\of([n]-\set k)}{\xi_i} \to [m+\norm\xi]$ be the \emph{nondecreasing} injection $\ka$ which sends the lexicographical product $\product{i\of([n]-\set k)}{\xi_i}$ to the interval in $[m+\norm\xi]$ which starts at the least $i$ such that $K_0(\xi)(i)=k$.

The lexicographical product means that $\product{i\of([n]-\set k)}{\xi_i}$ get the lexicographical ordering. This ordering determines priority by comparing elements in sequence. That means $\vec x < \vec y$ if there are $i$ such that $x_i \neq y_i$, and for the least such $i$ $x_i<y_i$. The map $K_0$ adds the lexicographical product of most of the fibres of a map $\xi\of[m]\to[n]$ to the fibre over $k$. %more?

\item Let $\la(\xi)\of[m]\to[m+\norm\xi]$ be the unique nondecreasing injection which skips the image of $\ka$. This means $\la(i)=i$ if $\xi(i)<k$ and $\la(i)=i+\norm\xi$ if $\xi(i)\geq k$. Moreover $K_0(\xi)\circ\la(\xi) = \xi$.
\item For each morphism $\phi\of\xi\to\xi'$ in $\simCat/[n]$ let $\phi_i\of\xi_i\to\xi'_i$ be the fibrewise morphism and let $\product{i\of([n]-\set k)}{\phi_i}$ be the corresponding map of the (lexicographical) products $\product{i\of([n]-\set k)}{\xi_i}\to\product{i\of([n]-\set k)}{\xi'_i}$.
\item Let $K_1(\phi)\of K_0(\xi)\to K_0(\xi')$ be the unique nondecreasing function which satisfies $K_1(\phi)\circ \ka(\xi) = \ka(\xi')\circ \product{i\of([n]-\set k)}{\phi_i}$ and $K_0(\phi)\circ \la(\xi) = \la(\xi')\circ \phi$.
\end{enumerate}

The maps $(K_0,K_1)$ define an endofunctor of $\simCat/[n]$. To get the functor, let $K(\xi) = h\ri(\simplex(K_0(\xi)))$ for objects $\xi$ of $\simCat/[n]$ and $K(\phi) = h\ri(\simplex(K_1(\phi)))$ for morphisms, where $h\ri$ is the reindexing functor $\ambient\s/\simplex[n]\to\ambient\s/\horn_k[n]$ along the horn inclusion $h\of \horn_k[n]\to\simplex[n]$. Concretely, $K$ satisfies the following equations for all objects $\xi$ of $\simCat/[n]$, all $x\of \base(\dom(K(\xi)))$ and all arrows $\phi$ of $\simCat/[n]$.
\begin{align*}
\base(\dom(K(\xi))) &= \set{ x\of\base\simplex[m+\norm\xi]| K_0(\xi)\circ x\of\base\horn_k[n]}\\
K(\xi)(x) &= K_0(\xi)\circ x \\
\dim(x) &= \dom(x) \\
x\cdot\phi &= x\circ\phi \\
K(\phi)(x) &= K_1(\phi)\circ x
\end{align*}
\end{definition}

Thanks to the following property, the natural equivalence of homsets above extends to horn inclusions.

\begin{lemma} An \keyword{intersection} is a pullback square of monomorphisms. The functor $K$ preserves all intersections of $\simCat/[n]$. \end{lemma}

\begin{proof} The easiest way to see that is to describe the monomorphisms in the intersection as decidable predicates on the domain of a map $\xi\of[m]\to[n]$. Maps $[m]\to\bool$ that equals $\true$ at least once characterize monomorphisms to $\xi$. The action of the endofunctor $(K_0,K_1)$ on the monomorphism has a parallel on predicates. For each $p\of[m]\to\bool$ the following equations define $p^K\of [m+\norm\xi]\to\bool$. 
\begin{align*}
p^K(\kappa(\vec x)) &= \forall i\of[m]-\set k.p(x_i) &
p^K(\lambda(x)) &= p(x)
\end{align*}
It all works out because $\forall i\of[m]-\set k.p(x_i)\land q(x_i)$ is equivalent to $(\forall i\of[m]-\set k.p(x_i))\land(\forall i\of[m]-\set k.q(x_i))$.
\end{proof}

\begin{definition}
For each $f\of X\to\horn_k[n]$ the functor $D\of \ambient\s/\horn_k[n]\to\ambient\s/\simplex[n]$ satisfies the following equations.
\begin{align*}
\base(\dom(D(f))) &= \sum_{\xi\of\base\simplex[n]}(\ambient\s/\horn_k[n])(K(\xi),f)\\
\dim(\xi,x) &= \dom\xi\\
(\xi,x)\cdot\phi &= (\xi\circ\phi,x\circ K(\phi))\\
D(f)(\xi,x) &= \xi
\end{align*}
On morphisms in $\ambient\s/\horn_k[n]$, the functor $D$ acts by composition: $Dm(\xi,x) = (\xi,m\circ x)$.
\end{definition}

If $\ambient$ has infinite colimits, the functor $D$ has a left adjoint $K'$ that satisfies $K'\simplex\simeq K$. We cannot rely on that property here. The functor $K$ does extend to finite colimits of simplices. Preservation of intersections implies that $K$ also applies to finite unions of simplices like $\horn_l[m]$. The reason $D$ preserves fibrations and their right lifting property, is that its partial left adjoint preserves the left lifting property. 

\begin{definition}[Face notation] We introduce a notation for monomorphisms in $\simCat/[n]$ or more accurately for the result of applying $\simplex$ to them. For $\xi\of[m]\to[n]$ and $\phi\of[m]\to\bool$ let $\face(\xi,p)\of \dom(\face(\xi,p))\to \simplex[m]$ satisfy
\begin{align*}
\base(\dom(\face(\xi,p))) &= \set{\alpha\of\base(\simplex[m])\middle|\forall i\of \dom(\alpha).p\circ\alpha(i)=\true}\\
\dim(\alpha) &= \dom(\alpha)\\
\alpha\cdot\chi &= \alpha\cdot\chi\\
\face(\xi,p)(\alpha) &= \alpha
\end{align*}
\[\xymatrix{
\bullet\ar[rr]^{\face(\xi,p)}\ar[dr] && \simplex[m]\ar[dl]^{\simplex(\xi)} \\
& \simplex[n]
}\]
\end{definition}

With the help of this notation, we describe commutative triangles of monomorphisms as follows.
\begin{align}
&\horn_l[m]\stackrel{\hat h}\to \simplex[m]\stackrel{\simplex(\xi)}\to \simplex[n] = \bigvee_{i\of [m]-\set l} \face(\xi,\function x{x\neq i})\\
&K(\horn_l[m])\stackrel{K(\hat h)}\to \simplex[m+\norm\xi]\stackrel{\simplex(K_0(\xi))}\to \simplex[n] = \bigvee_{j\of [m]-\set l} \face(K_0(\xi),v_j)\\
&h\ri(K(\horn_l[m]))\stackrel{h\ri(K(\hat h))}\to h\ri(\simplex[m+\norm\xi])\stackrel{h\ri(\simplex(K_0(\xi)))}\to \horn_l[n] = \\
&\quad\left(\bigvee_{i\of[n]-\set k} \face(K_0(\xi),u_i)\right)\wedge\left(\bigvee_{j\of [m]-\set l} \face(K_0(\xi),v_j)\right)=\\
&\quad\left(\bigvee_{\substack{i\of[n]-\set k\\j\of [m]-\set l}}\face(K_0(\xi),u_i\land v_j)\right)\\
&h\ri(\simplex[m+\norm\xi])\stackrel{h}\to \simplex[m+\norm\xi]\stackrel{K_0(\xi)}\to \simplex[n] =\left(\bigvee_{i\of[n]-\set k} \face(K_0(\xi),u_i)\right)
\end{align}
\begin{center}
where
\end{center}
\begin{align*}
u_i(x) &= (K_0\xi(x)\neq i) & 
v_i(\kappa(\xi)(\vec x)) &= (\xi(i)= k\vee x_{\xi(i)}\neq i)\\
v_i(\lambda(\xi)(x)) &= (x\neq i)&
u_i\land v_i(x) &= u_i(x)\land v_i(x)
\end{align*}


\begin{lemma} The monomorphism $h\ri K(\hat h)\of$
\[ \left(\bigvee_{\substack{i\of[n]-\set{k}\\j\of [m]-\set l}}\face(K_0(\xi),u_i\land v_j)\right)\to\left(\bigvee_{i\of[n]-\set{k}} \face(K_0(\xi),u_i)\right) \] 
has the left lifting property with respect to all fibrations.\label{left lifting property}\end{lemma}

\begin{proof}The class of morphisms with the left lifting property is closed under compositions and pushouts. The monomorphism $h\ri K(\hat h)$ belongs to this class because of this closure property.

Decompose $h\ri K(\hat h)$ as the inclusions of the following three subobjects.
\begin{align*}
A &=\dom(h\ri K(\hat h))=\left(\bigvee_{\substack{i\of[n]-\set{k}\\j\of [m]-\set l}}\face(K_0(\xi),u_i\land v_j)\right)\\
B &=\left(\bigvee_{j\of[m]-\set l}\face(K_0(\xi),u_{\xi(l)}\land v_j)\right)\vee \left(\bigvee_{i\of[n]-\set{k,\xi(l)}}\face(K_0(\xi),u_i)\right)\\
C &=\cod(h\ri K(\hat h))=\left(\bigvee_{i\of[n]-\set{k}} \face(K_0(\xi),u_i)\right)
\end{align*}

Decompose $A\to B$ as the inclusions of the following subobjects that satisfy $A_0=A$ and $A_{n+1}=B$.
\[ A_{j\of[n+1]} = A\vee \left(\bigvee_{i\of[j]-\set{k,j,\xi(l)}}\face(K_0(\xi),u_i)\right) \]
Each inclusion $A_j\to A_{j+1}$ is a pushout of the inclusion of $A_j\cap \face(K_0(\xi),u_{j+1})$ into $\face(K_0(\xi),u_{j+1})$, where
\begin{align*} A_j\cap \face(K_0(\xi),u_{j+1}) = &\left(\bigvee_{i\of[m]-\set l}\face(K_0(\xi),u_{j+1}\land v_{i})\right)\vee \\
&\left(\bigvee_{i\of[j+1]-\set{k,\xi(l)}}\face(K_0(\xi),u_i\land u_{j+1})\right)\end{align*}
The faces $\face(K_0(\xi),u_{j+1}\land v_{i})$ and $\face(K_0(\xi),u_i\land u_{j+1})$ all contain the point $\lambda(l)$ of $\simplex[m+\norm\xi]$ which means that lemma \ref{face completion} applies to the inclusion $A_j\cap \face(K_0(\xi),u_{j+1}) \to \face(K_0(\xi),u_{j+1})$. Hence $A\to B$ has the left lifting property.

If $\xi(l)=k$, then $B=C$ and the proof is complete. Suppose $\xi(l)=i\neq k$. First define the following maps $[m+\norm\xi]\to\bool$ for each $\vec p \of\product{i\of [n]-\set k}{\xi_i}$.
\begin{align*}
w_{\vec p}(\kappa(\vec q)) &= \true &
w_{\vec p}(\lambda(q)) &= ((\xi(q)=k\vee q\leq p_{\xi(q)})\land\xi(q)\neq \xi(l))
\end{align*}
Decompose $B\to C$ as the inclusions of the following subobjects that satisfy $B_0=B$ and $B_{m+\norm\xi+1}=C$.
\[ B_{j\of[m+\norm\xi+1]} = B \vee \left(\bigvee_{\kappa(\vec p)<j} \face(K_0(\xi),u_{\xi(l)}\land w_{\vec p})\right)\]

When $j\neq\kappa(\vec p)$ for any $\vec p \of \product{i\of [n]-\set k}{\xi_i}$, the inclusion $B_j\to B_{j+1}$ is the identity arrow, which satsifies the left lifting property trivially. 

Suppose $j\neq\kappa(\vec p)$. Each inclusion $B_j\to B_{j+1}$ is a pushout of the inclusion of $B_j\cap \face(K_0(\xi),u_{\xi(l)}\land w_{\vec p})$ into $\face(K_0(\xi),u_{\xi(l)}\land w_{\vec p})$, where
\begin{align*} B_j\cap \face(K_0(\xi),w_{\vec p}) = 
&\bigvee_{i\of [n]-\set{\xi(l),k}}\face(K_0\xi,u_i\land w_{\vec p})\vee\\
&\bigvee_{i\of [m]-\set l} \face(K_0\xi,v_i\land w_{\vec p})\vee\\
&\bigvee_{\kappa(\vec q)<j} \face(K_0\xi,w_{\vec q}\land w_j)
\end{align*}
Let $\vec p[l]\of \product{i\of [n]-\set k}{\xi_i}$ satisfy $p[l]_j = l$ if $\xi(l)=j$ and $p[l]_i=p_i$ otherwise. The point $\kappa(p[l])$ is a member of $\face(K_0\xi,u_i\land w_{\vec p})$ for all $i\of [n]-\set{\xi(l),k}$. It is also a member of $\face(K_0\xi,w_{\vec q}\land w_{\vec p})$ for all $\vec q\of \product{i\of [n]-\set k}{\xi_i}$ such that $\kappa(\vec q)<\kappa(\vec p)$. When $\kappa(\vec p[l])$ is not a member of $\face(K_0\xi,v_i\land w_{\vec p})$ for some $i$, then $\face(K_0\xi,v_i\land w_{\vec p})$ is a subobject of a face of $B_j\cap \face(K_0(\xi),\land w_{\vec p})$ that does contain $\kappa(\vec p[l])$.

Let $i\of[m]-\set l$. If not $v_i(\kappa(\vec p[l]))$ then by definition $\xi(i)\neq k$ and $(p[l]_\xi(i) = i)$. If $i$ is the least member of $\xi_{\xi(i)}$, then $\face(K_0\xi,v_i\land w_{\vec p})\subseteq \face(K_0\xi,u_{\xi(i)}\land w_{\vec p})$. If $\xi(i-1)=\xi(i)$, then $\kappa(\vec p[i-1])<\kappa(\vec p)$ and $\face(K_0\xi,v_i\land w_{\vec p})\subseteq \face(K_0\xi,W_{\vec p[i-1]}\land w_{\vec p})$.

Since $B_j\cap \face(K_0(\xi),\land w_{\vec p})$ is a unions of faces that have the point $\vec p[l]$ in common, lemma \ref{face completion} applies to the inclusion $B_j\cap \face(K_0(\xi),\land w_{\vec p})\to \face(K_0(\xi),\land w_{\vec p})$. Therefore $B_j\to B_{j+1}$, $B\to C$ and $h\ri K(\hat h)\of A\to C$ all have the left lifting property.
\end{proof}

\begin{lemma}[Face completion] Let $F$ be an inhabited decidable set of faces of $\simplex[p]$ which all have a point $e$ of $\simplex[p]$ in common. The inclusion $\bigcup F\to \simplex[p]$ has the left lifting property with respect to all fibrations. \label{face completion} \end{lemma}

\begin{proof} For all $j\of[p]$ let $F_j$ be the union of $F$ with the set of $j$-dimensional faces of $\simplex[p]$ which contain the edge $e$. Because $F$ is inhabited, $\bigcup F$ contains $e$ and therefore $F_0=F$. Because $\simplex[p]$ is a $p$-dimensional face of $\simplex[p]$ which contains $e$, $\bigcup F_p = \simplex[p]$. For $j>0$ let $S_j$ be the set of $j$-dimensional faces of $\bigcup F_j$ which are not already contained in $\bigcup F_{j-1}$. If a $j$-dimensional face $\face(\Sigma)$ of $\bigcup F_j$ opposes $e$, it is part of a higher dimensional face which is a member of $F$. Therefore each face $\face(\Sigma)\of S_j$ contains $e$. For this reason $\face(\Sigma)\cap \bigcup F_{j-1}$ is the horn whose central edge is $e$. The inclusion $\bigcup F_{j-1}\to\bigcup F_j$ is therefore the pushout of a coproduct of horn inclusions indexed over $S_j$. Therefore the inclusion has the left lifting property. Because composition preserves the left lifting property, $\bigcup F = F_0\to F_p = \simplex[p]$ has it too. 
\end{proof}

\end{document}