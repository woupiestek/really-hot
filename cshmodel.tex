\documentclass[csh.tex]{subfiles}
\begin{document}

\section{Model Structure}
This section outlines the proof of the main theorem of this section. We start with definition the core concepts.

\begin{definition} A \keyword{model structure} on a category $\cat C$ consist of three sets of morphism: the \emph{fibrations} $F$, the \emph{cofibrations} $C$ and the \emph{weak equivalences} $W$ which have the following properties. 
\begin{itemize}
\item The set $W$ satisfies \emph{2-out-of-3} which means that for every pair of morphisms $f\of X\to Y$ and $g\of Y\to Z$ of $\cat C$, if two out of $f$, $g$ and $g\circ f$ are in $W$ so is the third. 
\item The pairs of sets $(C,F\cap W)$ and $(C\cap W,F)$ are \emph{internal factorization systems}. 
\end{itemize}

Internal factorization systems are pairs of set of morphisms $(L,R)$ in a category with the following properties.
\begin{itemize}
\item A morphism $l$ belongs to $L$ if an only if it has the left lifting property with respect all to members of $R$.
\item The class $R$ has the dual property.
\item Every morphism factors as a member of $R$ following a member of $L$.
\end{itemize}

Members of $C\cap W$ are called \emph{acyclic cofibrations} and members of $F\cap W$ are called fibrations.
\end{definition}

In ambient we make the following selections. 

\begin{definition} A morphism $f\of X\to Y$ of category $\ambient\s$ is a \keyword{fibration} if it has the right lifting property with respect to the family of all \emph{horn inclusions}, which is defined as follows. The \emph{horn} $\horn_k[n]$ for $n\of\nno$ and $k\of[n]$ is the subobject of $\simplex[n]$ for which $\base\horn_k[n]$ consists of morphism of $\simCat$ to $[n]$ which miss at least one member of $[n]$ other different from $k$. The \emph{horn inclusion} is the monomorphism $\horn_k[n]\to\simplex[n]$. The \emph{family of horn inclusions} is the sum of all horn inclusions $\Sigma k\leq n\of\nno.\horn_k[n]\to \Sigma k\leq n\of\nno.\simplex[n]$ as it exists as a morphism of $\ambient\s$.

The lifting property that $f$ satisfies is defined in a slice category of $\ambient\s$. Let $I = \set{\tuplet{n,k}\of\nno\times\nno|k\leq n}$. There is a \emph{discrete simplicial set} $I\disc$ which satisfies $\base(I\disc) = \nno\times I$, $\dim\tuplet{n,x}=n$ and $\tuplet{n,x}\cdot\phi=\tuplet{n,x}$ for $\phi\of[m]\to[n]$. The family of horn inclusions is a morphism in $\ambient\s/I\disc$. So is the multiple $I\disc\times f\of I\disc\times X\to I\disc\times Y$ of $f$. The morphism $f$ is a fibration if $I\disc\times f$ has the right lifting property with respect to $\Sigma k\leq n\of\nno.\horn_k[n]\to \Sigma k\leq n\of\nno.\simplex[n]$.
\end{definition}

If $\ambient$ is the topos of sets the morphism that satisfy this definition are precisely \emph{Kan fibrations}. 

\begin{definition} A monomorphism $f\of X\to Y$ of category $\ambient\s$ is a \keyword{cofibration} if subobject of faces of $Y$ which are not in the image of $X$ is decidable. A \emph{face} of $Y$ is a $y\of\base Y$ such that if $\phi\of [\dim(y)]\to [\dim(y)]$ and $y\cdot \phi = y$, then $\phi = \id_{\dim(y)}$. Hence $f$ is a cofibration if there is a morphism $\psi\of\base Y\to\bool$ such that $\psi(y)=\true$ if an only if $y$ is a face and there is no $x\of \base X$ such that $f(x)=y$. 
\end{definition}

In the category of simplicial sets all monomorphism a cofibrations. Decidability comes for free there.

\begin{definition} A \keyword{weak equivalence} is a composition of an acyclic fibration following an acyclic cofibration. Here, an \emph{acyclic fibration} is a morphism that has the right lifting property with respect to all cofibrations; an \emph{acyclic cofibration} is a cofibration has the right lifting property with respect to all fibrations.
\end{definition}

We want our theorems to work in a $\Pi$-pretopos $\ambient$ that doesn't have infinite colimits. The small object argument uses such colimits to provide factorizations of morphisms that a model structure requires, so we cannot use it here. In fact, to get a model structure we retreat to a subcategory of $\ambient\s$.

\begin{definition} A \keyword{complex} is a tuple $\tuplet{\base X,\dim,\cdot,\filler}$ where $\tuplet{\base X,\dim,\cdot}$ is a simplicial object and $\filler$ a lifting operator for the morphism $\bang\of \tuplet{\base X,\dim,\cdot}\to 1$. A morphism of complexes is simply a morphism of simplicial objects. The \emph{category of complexes} and morphisms of complexes is $\ambient\s\f$.
\end{definition}

\hide{everything below was cut from somewhere else. because we changed the definitions, we need to check our language.

In particular: the fact that an acyclic cofibrations between fibrant objects is a kind of homotopy equivalence was taken as definition for some time.

}

\begin{theorem}[Model category]
With fibrations, weak equivalences and cofibrations defined as above $\ambient\s\f$ is a model category.
\label{model category}
\end{theorem}

\begin{proof}
Lemma \ref{factor1} shows that cofibrations and acyclic fibrations form a weak factorization system, considering that they satisfy a lifting property by definition \ref{Kan}. Lemma \ref{factor2} tells the same thing about fibrations and acyclic cofibrations. Lemma \ref{toot} below demonstrates that if two of $f$, $g$ and $f\circ g$ are weak equivalences, then all three are. These three requirements define a model structure.
\end{proof}

\begin{lemma}[2-out-of-3] Let $f\of X\to Y$ and $g\of Y\to Z$ be morphisms of $\ambient\s\f$. If any two of $f,g$ or $g\circ f$ are weak equivalences, then all three are. \label{toot}\end{lemma}

\begin{proof} Weak equivalences are closed under composition by lemma \ref{composition of weak equivalences}.

Let $g$ and $g\circ f$ be arbitrary weak equivalences. The morphism $f$ factors as an acyclic fibration $h\of W\to Y$ following a cofibration $k\of X\to W$ by proposition \ref{factor2}. Because weak equivalences are closed under composition (see \ref{composition of weak equivalences}), $g\circ h$ is a weak equivalence. The morphism $k$ is acyclic for the following reasons. Factor both $g\circ f$ and $g\circ h$ as acyclic fibrations following acyclic cofibrations, so $g\circ f = a\circ b$ and $g\circ h = c\circ d$. The lifting properties induce a morphism $l$ such that $l\circ b = d\circ k$ and $c\circ l = a$. Lemma \ref{shared retract 2} says that $l$ is a weak equivalence because $a$ and $c$ are acyclic fibrations. 
Because of closure under composition, the morphism $l\circ b = d\circ k$ is both a weak equivalence and a cofibration and hence an acyclic cofibration. Since $d$ and $d\circ k$ are an acyclic cofibrations, so is $k$ by lemma \ref{left cancellation}.
\[\xymatrix{
X\ar[r]_k\ar@/^2ex/[rr]^f\ar[d]_b & W\ar[r]_h\ar[d]^d & Y\ar[d]^g \\
\bullet\ar@/_2ex/[rr]_a\ar@{.>}[r]^l & \bullet\ar[r]^c & Z
}\]
Since $f = h\circ k$, $f$ is a weak equivalence.

The case where $f$ and $g\circ f$ are weak equivalences is dual to the case above and the reasoning is the same. Acyclic cofibrations satisfy lemma \ref{shared retract} where acyclic fibrations satisfy lemma \ref{shared retract 2}. Acyclic fibrations satisfy lemma \ref{right cancellation} where acyclic cofibrations satisfy lemma \ref{left cancellation}.

This means that weak equivalences indeed satisfy 2-out-of-3.
\end{proof}

\begin{lemma} Weak equivalences are closed under composition. \label{composition of weak equivalences}\end{lemma}

\begin{proof} Compositions of acyclic fibrations are acyclic fibrations and the same holds for acyclic cofibrations.
All compositions of weak equivalences are weak equivalences, if $g\circ f$ factors as an acyclic fibration following an acyclic cofibration for each acyclic cofibration $g$ and acyclic fibration $f$.

%transport of acyclics
By proposition \ref{factor1} $g\circ f=h\circ k$ for some acyclic fibration $h\of W\to Z$ and a cofibration $k\of X\to W$. Let $g'$ be the left inverse of $g$. Since $f \circ \id = g'\circ g\circ f= (g'\circ h)\circ k$ there is a morphism $k'$ such that $f\circ k' = g'\circ h$ and $k'\circ k = \id$, so $k$ has its own left inverse.
\[\xymatrix{
X\ar[d]_f \ar[r]_{k} \ar@/^2ex/[rr]^{\id} & W\ar[d]^h \ar[r]_{k'} & X\ar[d]^f\\
Y \ar[r]^{g} \ar@/_2ex/[rr]_{\id} & Z \ar[r]^{g'} & Y
}\]

Let $\phi$ be a homotopy between $\id_Z$ and $g\circ g'$ such that $\phi\circ (\id\times g)=g\circ\pi_1$.

There is a homotopy $\chi$ between $\id_W$ and $k\circ k'$ because of lemma \ref{triple lift}.
\begin{align*}
(\id, k\circ k')\circ (k+k) &= (k,k) = k\circ \pi_1\circ(c\times\id_X)\\
h\circ (k \circ \pi_1) &= (\phi\circ (\id_W\times h))\circ (\id_{\simplex[1]} \times k)\\
h\circ (\id, k\circ k') &= (\phi\circ (\id_W\times h))\circ (c\times \id_W)
\end{align*}
\[\xymatrix{
X+X\ar[d]_{c\times \id}\ar[r]^{k+k} & W+W\ar[d]_(.3){c\times \id}\ar[r]^(.6){(\id,k\circ k')} & W\ar[d]^h\\
\simplex[1]\times X\ar[r]_{\id\times k} \ar[urr]^(.3){k\circ \pi_1}  & \simplex[1]\times W\ar[r]_(.6){\phi\circ (\id\times h)} \ar@{.>}[ur]_\chi & Z
}\]
Because the homotopy satisfies $\chi\circ(\id_{\simplex[1]}\times k) = k\circ \pi_1$, definition \ref{acyclic cofibration} makes $k$ an acyclic cofibration.

\end{proof}

\begin{lemma} If $f\of X\to Y$, $g\of X\to Y$ and if $g$ and $g\circ f$ are acyclic cofibrations, then $f$ is an acyclic cofibration \label{left cancellation} \end{lemma}

\begin{proof} Let $k\of A\to B$ be a fibration and let $a\of X\to A$ and $b\of Y\to B$ satisfy $k\circ a=b\circ f$. Because $B$ is fibrant, there is a $b'\of Z\to B$ such that $b'\circ g = b$. Lifting properties also imply that there is an $a'\of Z\to A$ such that $a'\circ g\circ f = a$ and $k\circ a'= b'$. So $b'\circ g$ is a filler for $k\circ a=b\circ f$. By abstraction, there is a filler operator that lifts $f$ against all fibrations, and that makes it n acyclic cofibration.
\[\xy
(0,28)*+{X}="x",(14\halfrootthree,21)*+{A}="a",(0,14)*+{Y}="y",(14\halfrootthree,7)*+{B}="b",(0,0)*+{Z}="z"
\ar^a "x";"a" \ar_f "x";"y" \ar^k "a";"b" \ar|(.33)b "y";"b" \ar_g "y";"z"
\ar@{.>}_{b'} "z";"b" \ar@{.>}|(.67){a'} "z";"a"
\endxy\]
\end{proof}

\begin{lemma} If $f\of X\to Y$, $g\of X\to Y$ and if $f$ and $g\circ f$ are acyclic cofibrations, then $g$ is an acyclic fibration. \label{right cancellation}\end{lemma}

\begin{proof} This is the dual of of lemma \ref{left cancellation} and dual reasoning gives $g$ the right lifting property for all cofibrations. That makes $g$ an acyclic fibration. \hide{Next assume that $f$ and $g\circ f$ are acyclic fibrations. Let $k\of\cycle[n]\to\simplex[n]$ be the cycle inclusion. Let $a\of\cycle[n]\to Y$ and $b\of\simplex[n]\to Y$ satisfy $b\circ k = g \circ a$. Because $\cycle[n]$ is cofibrant, there is an $a'\of \cycle[n]\to X$ such that $f\circ a' = a$ and hence $(g\circ f)\circ a' = b\circ c$. There is a filler $b'\of\simplex[n]\to X$ for this commutative square. The morphism $f\circ b'$ is a filler for the square $b\circ k = g \circ a$. This proves $g$ is an acyclic fibration.

\[\xy
(14\halfrootthree,28)*+{X}="x", (0,21)*+{\cycle[n]}="v", (14\halfrootthree,14)*+{Y}="y", (0,7)*+{\simplex[n]}="w", (14\halfrootthree,0)*+{Z}="z"
\ar@{.>}^{a'} "v";"x" \ar|(.67)a "v";"y" \ar_k "v";"w"
\ar@{.>}|(.33){b'} "w";"x" \ar_b "w";"z"
\ar "x";"y" \ar "y";"z"
\endxy\]}
\end{proof}

\begin{lemma} If $f\of X\to Y$, $g\of X\to Y$ and if $g$ and $g\circ f$ are acyclic fibrations, then $f$ is a weak equivalence.\label{shared retract 2} \end{lemma}

\begin{proof} By proposition \ref{factor1}, $f$ factors as an acyclic fibration $h\of W\to Y$ following a cofibration $k\of X\to W$. Because $(g\circ f)\circ \id = (g\circ h)\circ k$ and $g\circ f$ is an acyclic fibration, $k$ has a left inverse $k'\of X\to W$ which satisfies $g\circ f\circ k' = g\circ h$. 
\[\xymatrix{
X\ar[d]_k \ar[r]^\id & X\ar[d]^{g\circ f}\\
W\ar[r]_{g\circ h} \ar@{.>}[ur]^{k'} & Z
}\]

Let $c\of 1+1\to\simplex[1]$ be the same cycle as above. There is homotopy $\phi$ between $\id_W$ and $k\circ k'$ because of lemma \ref{triple lift} and the following equations.
\begin{align*}
(\id,k\circ k')\circ(k+k) &= (k,k) = (k\circ\pi_1)\circ (c\times\id_X)\\
(g\circ h)\circ (k\circ \pi_1) &= (g\circ h\circ \pi_1)\circ(\id\times k)\\
(g\circ h)\circ (\id,k\circ k') &= (g\circ h\circ \pi_1)\circ(c\times\id_W)
\end{align*}
\[\xymatrix{
X+X \ar[d]_{c\times \id}\ar[r]^{k+k} & W+W\ar[d]_(.3){c\times\id}\ar[r]^(.6){(\id,k\circ k')} & W\ar[d]^{g\circ h}\\
\simplex[1]\times X \ar[r]_{\id\times k}\ar[urr]^(.3){k\circ\pi_1}& \simplex[1]\times W\ar[r]_{g\circ h\circ \pi_1}\ar@{.>}[ur]_\phi & Z
}\]
Because $\phi\circ (\id_{\simplex[1]}\times k)=k\circ \pi_1$, definition \ref{acyclic cofibration} says that $k$ is an acyclic cofibration. Therefore $f$ is a weak equivalence.\end{proof}

\begin{lemma} If $f\of X\to Y$, $g\of X\to Y$ and if $f$ and $g\circ f$ are acyclic cofibrations, then $g$ is a weak equivalence.\label{shared retract} \end{lemma}

\begin{proof} \hide{The morphism $g$ factors as a fibration $h\of W\to Z$ following an acyclic cofibration $k\of Y\to W$ by lemma \ref{factor2}. Lifting properties give $h$ a right inverse $h'$.
\[\xymatrix{
X\ar[r]^{k\circ f}\ar[d]_{g\circ f} & W\ar[d]^h\\
Z\ar[r]_\id \ar[ur]^{h'} & Z
}\]
Lemma \ref{triple lift} provides a homotopy $\phi$ between $\id_W$ and $h'\circ h$, because the following equations hold.
\begin{align*}
(\id,h'\circ h)\circ(k\circ f+k\circ f)&= (k\circ f,k\circ f) = (k\circ f\circ \pi_1)\circ(c\times \id_Y)\\
h\circ(k\circ f\circ\pi_1) &= g\circ f\circ\pi_1 = (h\circ\pi_1)\circ(\id_{\simplex[1]}\times (k\circ f))\\
h\circ(\id_W,h'\circ h) &= (h,h) = (h\circ\pi_1)\circ(c\times\id_W)
\end{align*}

\[\xymatrix{
X+X\ar[r]^{k\circ f+k\circ f}\ar[d]_{c\times \id} & W+W\ar[r]^(.6){(\id,h'\circ h)}\ar[d]_(.3){c\times\id} & W\ar[d]^h\\
\simplex[1]\times X\ar[r]_{\id\times (k\circ f)}\ar[urr]^(.3){k\circ f\circ\pi_1} & \simplex[1]\times W\ar[r]_(.6){h\circ \pi_1}\ar@{.>}[ur]_{\phi} & Y
}\]}

This lemma is dual to lemma \ref{shared retract 2}. The morphism $g$ factors as a fibration $h\of W\to Z$ following an acyclic cofibration $k\of Y\to W$ by lemma \ref{factor2}. The dual of the proof of lemma \ref{shared retract 2} doesn't show that $h$ is an acyclic fibration directly, but it does show that $h$ has a right inverse $h'\of Z\to W$ and that there is a homotopy $\phi\of\simplex[1]\times W\to W$ between $\id_W$ and $h'\circ h$.

Let $a\of I\to J$ be an arbitrary cofibration and let $i\of I\to W$ and $j\of J\to Z$ satisfy $j\circ a=h\circ i$. Lemma \ref{triple lift} deforms $h'\circ j$ into a filler. Let $c_i$ be the morphisms $1\to \simplex[0]$.
\begin{align*}
\phi\circ(\id\times i)\circ (c_1\times \id) &= h'\circ h\circ i = h'\circ j\circ a \\
h\circ h' \circ j &= j = j\circ \pi_1\circ(c_1\times\id) \\
j\circ \pi_1\circ (\id\times a) &= j\circ a\circ \pi_1 = h\circ i \circ \pi_1\\
h\circ \phi\circ(\id\times i) &= h \circ \pi_1 \circ (\id\times i) = h\circ i \circ \pi_1
\end{align*}

\[\xymatrix{
I\ar[r]^{a}\ar[d]_{c_1\times \id} & J\ar[r]^(.6){h'\circ j}\ar[d]_(.3){c_1\times\id} & W\ar[d]^h\\
\simplex[1]\times I\ar[r]_{\id\times a}\ar[urr]^(.3){\phi\circ(\id\times i)} & \simplex[1]\times J\ar[r]_(.6){j\circ \pi_1}\ar@{.>}[ur]_{\psi} & Z
}\]
The filler is $\psi\circ c_0$. By generalization, $h$ is an acyclic fibration and $f$ is a weak equivalence.
\end{proof}

\end{document}