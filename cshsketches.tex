\documentclass{tac}
\usepackage{amssymb, amsmath}
\usepackage[backend=bibtex,citestyle=authoryear-icomp]{biblatex}
\usepackage[all]{xy}
\usepackage{url}

\title{Constructive Simplicial Homotopy}
\author{Wouter Pieter Stekelenburg}
\copyrightyear{2015,2016,2017}
\address{Faculty of Mathematics, Informatics and Mechanics\\
University of Warsaw\\
Banacha 2\\
02-097 Warszawa\\
Poland}
\eaddress{w.p.stekelenburg@gmail.com}
\keywords{realizability, simplicial homotopy, Kan complexes}
\amsclass{03D80, 18G30, 18G55}

\newcommand\hide[1]{}
\newcommand\cat\mathcal
\newcommand\set[1]{\left\{#1\right\}}
\mathrmdef{id}
\mathrmdef{dom}
\mathrmdef{cod}
\newcommand\ri{^*}
\newcommand\N{\mathbb N}
\mathbfdef[nno]{N}
\newcommand\dual{^{\mathrm{op}}}
\mathbfdef[simCat]\Delta
\newcommand\s{^{\simCat\dual}}
\newcommand\bang{!}
\newcommand\of{:}
\newcommand\simplex\Delta
\newcommand\cycle{\partial\Delta}
\newcommand\horn\Lambda
\newcommand\f{_f}
\newcommand\tuplet[1]{\left\langle #1 \right\rangle}
\newcommand\true{\mathtt{true}}
\newcommand\false{\mathtt{false}}
\newcommand\bool{\mathtt{bool}}
\mathrmdef{nat}
\mathssbxdef{Ar}
\mathssbxdef{Ob}
\newcommand\pp{\mathbin\diamond}
\newcommand\norm[1]{\Vert #1 \Vert}
\newcommand\ka\kappa
\newcommand\la\lambda
\mathrmdef{face}
\mathrmdef{colim}
\newcommand\ex{_{\textrm{ex}}}
\newcommand\citep[1]{[\cite{#1}]}
\mathrmdef{dim}
\newcommand\base{\mathbf{U}}
\mathssbxdef[sub]{Sub}
\newcommand\ambient{\mathfrak A}
\mathrmdef{uni}
\newcommand\disc{_{\rm disc}}
\mathrmdef{filler}
\newcommand\traco\omega

\newcommand\product[2]{\Pi #1 \mapsto #2}
\newcommand\coproduct[2]{\Sigma #1 \mapsto #2}
\newcommand\function[2]{\lambda #1 \mapsto #2}

\newcommand\keyword[1]{\emph{#1}\label{#1}}


\begin{document}
We are working out exactly what a \emph{horn or cycle filler} operator is, just like we worked out what a simplicial set is, in hopes of making the proof clearer and more convincing.



The lifting property is more demanding because it has to take all cases into account.
I work in all kinds of slice categories to hide to gory details, and keep track of additional requirement:
For a particular $f\of X\to Y$ and the family of horns $h\of D\to C$.
\begin{align*}
P &= \set{ \tuplet{a,b}\of(D\to X)\times(C\to Y)| f\circ a = b \circ h }\\
\dim(a(\tuplet{\phi,k})) &=\max(\cod(\phi))\\
\dim(b(\tuplet{\phi,k})) &= \max(\cod(\phi))\\
a(\tuplet{\phi,k})\cdot\chi &= a(\tuplet{\phi\circ \chi,k}) \\
b(\tuplet{\phi,k})\cdot\chi &= b(\tuplet{\phi\circ \chi,k}) \\
\end{align*}
\begin{itemize}
\item The cycle case is simpler, because we can drop the $k$-index.
\item The filler satisfies the same restrictions on dimensionality and restrictions, plus $c\circ h = a$ and $f\circ c = b$.
\end{itemize}

We now get a simplified notion of what a filler operator for both families are, and can try to rewrite the proofs in order to build exactly such an operator.

Suppose we work out an official definition of horn and cycle fillers. What might happen?
The step from the filler to lifting cofibrations may be less obvious, requiring some updates of the proofs and definition.
Not the acyclic though, just the weakly invertible.

\newcommand\HLP{\mathrm{HLP}}
\begin{definition}
Let  Let $f\of X\to Y$ be an arbitrary morphism of $\ambient\s$. The object of horn lifting problems is 
\[\HLP(f) = \set{\tuplet{a,b}\of \base X^D\times \base Y^C \middle|\begin{array}{l}
  f\circ a = b\circ h,\\
  \dim(b\tuplet{\phi,k}) = \max(\dom\phi),\\
  b\tuplet{\phi,k}\cdot\chi = b\tuplet{\phi\circ \chi,k},\\
	a\tuplet{\phi,k}\cdot\chi = a\tuplet{\phi\circ \chi,k}
\end{array}}\]
Here $h\of D\to C$ is the inclusion of the following objects in $\ambient$.
\begin{align*}
C &= \coproduct{\phi\of\Ar(\simCat)}\cod(\phi) \\
D &= \set{\tuplet{\phi,k}\of C \middle| \exists i\of\cod(\phi)-\set k. \forall j\of\dom(\phi).\phi(j)\neq i}
\end{align*}
A \keyword{horn filler} is a morphism $c\of\HLP(f)\to \base X^C$ that satisfies the following equations. 
\begin{align*} c\tuplet{a,b}\circ h &= a & f\circ c\tuplet{a,b} &= b \end{align*}
A \emph{fibration} of $\ambient\s$ is a morphism with a horn filler.
\end{definition}

%Cycle fillers 
\newcommand\CLP{\mathrm{CLP}}
\begin{definition}
Let $f\of X\to Y$ be an arbitrary morphism of $\ambient\s$. The object of cycle lifting problems is 
\[\CLP(f) = \set{\tuplet{a,b}\of \base X^D\times \base Y^C \middle|\begin{array}{l}
  f\circ a = b\circ h,\\
  \dim(b(\phi)) = \max(\dom\phi),\\
  b(\phi)\cdot\chi = b(\phi\circ \chi),\\
  a(\phi)\cdot\chi = a(\phi\circ \chi)
\end{array}}\]
Here $h\of D\to C$ is the inclusion of the following objects in $\ambient$.
\begin{align*}
C &= \Ar(\simCat) \\
D &= \set{\phi\of C \middle| \exists i\of\cod(\phi).\forall j\of\dom(\phi).\phi(j)\neq i}
\end{align*}
A \keyword{cycle filler} is a morphism $c\of\CLP(f)\to \base X^C$ that satisfies the following equations. 
\begin{align*} c\tuplet{a,b}\circ h &= a & f\circ c\tuplet{a,b} &= b \end{align*}
A morphism  of $\ambient\s$ is \emph{contractible} if it has a cycle filler.
\end{definition}

Where do we have problems with generalizations?
\begin{itemize}
\item proposition on the factorization of $f$ as a contractible morphism following a cofibration.
\item the triple lifting property. Improved now.
\item if $f$ and $g\circ f$ are contractible, then so is $g$. Improved now.
\item fibrancy of the universal modest fibration
\end{itemize}

\section{Triple lifting property}
We derive this from the equality of contractible and the acyclic fibrations. The core argument is that $\tuplet{f\to g}$ is a fibration if $f\of A\to B$ is a cofibration and $g\of X\to Y$ is a fibration, and contractible if $g$ is contractible or $f$ is acyclic. So how to define the horn and cycle fillers?

\[ \tuplet{f\to g} = \function{x\of X^B}\tuplet{g\circ x,x\circ f}\of X^B \to Y^B\times_{Y^A} X^A \]

Another approach is to take the pushout product of the families and construct the desired fillers there.

Take push out product of the family of horns with the family of cycles, then find a filler.
If $f$ is contractible, take the pushout product of the family of cycles with itself, and find a filler for that too.
Then use adjoints etc. to prove the lemma.
We may be able to pull it of without products, however. The triple lift for the families is a special case form which we derive all the special cases.

How far would we get with just the family of cycles along an (acyclic) cofibration?

The ultimate point is that we can fill the pushout product by systematically using fillers for simplices.
That algorithmic reduction of a triple lifting problem to a horn or cycle lifting problem can be done inside $\ambient$ to show that elementary triple lifting problems have a solution if horn or cycle lifting problems have. There rest is the reduction of the general triple lifting problem to the product of the families.

The reduction uses monotone monics $[m + n]\to[m]\times[n]$. 
We start with a collection of $[m+n-1]$-dimensional faces in the pushout product. Unless $m=0$ or $n=0$ we have no $[m+n]$ dimensional faces however. So these need to be glued in, one by one.
These can be lexicographically ordered again, after which we ensure that they get filled.
I.e. we build a sequence $A_i\to A_{i+1}$ of approximations and show that each requires a well defined application of the preexisting filler operator.
Clearly, the result is a proof in the same style as the descent proof, with one important difference: we don't rely on saturation, but an a filler operator to do the job.

\paragraph{set up for triple lift}
We need to fill $\simplex[m]\times \cycle[n]\cup \horn_k[m]\times\simplex[n]$. The first subdivision is along $\set{k}\times \simplex[n]$, whose point we will use as tops of horns to fill. The second subdivision is along $[m]-\set{k}$. We include down sets into each new $m+n$ dimensional face we glue in, until we are completes. The last subdivision is along dimension, as it seems we need face completion to get the required horns.

The main point of course is that we have a construction and therefore a filler operator that works on arbitrary horns.

Can we focus on greater collections than individual simplicies? At least in the intermediate steps we can\dots

\paragraph{considerations}
What matters for the case c-c-af is that the pushout product is a cofibration, which requires no reductions.
The cases c-ac-f and ac-c-f are where the reduction to the family of horns is needed.
Still, we should be comfortable with just the case of a pushout product of the family with an arbitrary cofibrations.

\paragraph{subdivisions}
We want to systematically subdivide the problem of filling
\[ A = \simplex[m]\times \cycle[n]\cup \horn_k[m]\times\simplex[n] \to \simplex[m]\times \simplex[n] \]
Damn this is hard.

Each monic $[m+n]\to [m]\times [n]$ must be glued in at some stage. This gluing in possible if we have a horn.
That is a base point $l$ such that the composition with monics $[m+n-1]\to [m+n]$ that hit $l$ have already been included at an earlier stage.

One thing we know: composition of these monotone monic with the projections $\to[m]$ an $\to[n]$ are surjective. Why?
By induction over $m+n$. Case 0 is trivial. Otherwise the second last element must be lesser in one of the variables,
hence inside the $m+n-1$ dimensional subsquares. The induction hypothesis makes the compositions of the submonic surjective. This forces $f(m+n) = (m,n)$, answering when either of those are hit.

What do we start with?
\begin{itemize}
\item All monics $[m+n-1] \to ([m]-\set{i})\times[n]$, where $i\neq k$.
\item All monics $[m+n-1] \to [m]\times ([n]-\set{j})$
\item No monics $[m+n-1] \to ([m]-\set{k})\times[n]$
\end{itemize}
For each face we glue in, the intersection with the previous complex cannot be a cycle however.
Given a supporting point, there can be no monics that omit it.
This is why the spine $\set{k}\times[n]$ is the unique supply of supporting points.

Monotone monics cannot meander. The set of elements that intersect the spin is always a simple interval.

What worries me is that adding any monic necessarily adds all the monics that miss each point along the spine.

Suppose $M_j$ consists of monics that intersect the $k$-spine in $j+1$ places. Is this the proper selection?
Do we every get any monics for free? I.e. without explicitly gluing them in?

Let's try the most difficult situation and hope for the best.

$M_{i,j}$ where $i\leq j\of[n]$ is the set of monotone monics that hit the points $(k,i)$ to $(k,j)$.
For gluing in $f$ we pick $(k,i)$ as top. Now $f\circ d_l$ for $l\neq k+i$ should already be present, while
$f\circ d_{k+i}$ should not.

We look for the first point where $f(k+j+n)_1 > j$ and swap all those points!?
Nope, we need to subdivide one step further.

Better approach: the 'distance' of two monotone monics is the number of places they don't coincide.
There is a specific monic that we start with, and then we progressively add in monics at greater distances.
Note the law $f(k)_0+f(k)_1=k$, which may be even better than out surjectivity claims.

So let's try again: $f_0(l) = \tuplet{l,0}$ if $l\leq k$, $f_0(l) = \tuplet{k,l-k}$ if $l$ is between $k$ and $k+n$, and 
$f_0(l) = \tuplet{l-n,n}$ if $l\geq n+k$. We gluing the faces in order of their distance (a number of steps from the basis figure)
to this monic. That way, we can rely on the overlap to prove required faces are present.

The only problem is choosing the base point is such a way that the opposing face is absent. Can we always do this?

Let $f$ be glued in, and $g$ be different from in in point $p\of[m+n]$. For $g\circ \partial_q$ if $q\neq p$ we know those faces are available.
We also need a base point, however. Some place where $f_0$ and $g_0$ are both $k$.
Then we note that all opposites to where $f_0=k$ have been added however\dots
What am I saying? All subfaces of $g$ are different from $f$ except $g\circ \partial_p$!

In general, the surjectivity ensures we can use the $k$-spine to glue in any face. What we need is the opposite:
a proof that the face opposing the chosen base point has not already been glued in!

So for $f$ let the least $p$ such that $f(p)_0=k$ be the base point. We would hate for $f\circ\partial_p$ to be present, and there is exactly one face that could case this tragedy: $f + (p \mapsto \tuplet{k-1,f_1(p)+1})$. Its least point is $p+1$, so that gives us a criterion for ordering faces: the least points that hit $k$ 

The existing subfaces seem like causes for $f\circ\partial_p$ existence, but that requires either coordinate of $f(p) = \tuplet{k,x}$ not to occur anymore.
In the first case, that forces $f(p-1)_0=k-1$ and $f(p+1)_0=k+1$, which is in the missing $([m]-\set k)\times [n]$ area. In the second, $f(p-1) = \tuplet{k,x-1}$, which means that $p$ is not the least point, contradicting an assumption. We in fact hardly notice the preexisting faces.

The main part is then 'face completion': we have a base point and dimension by dimension glue in the necessary faces, noting that we already have all the $[m+n-2]$ ones already.

\paragraph{Setting up again}
More analysis: every monic can be seen as a sequence of increments in either vertical or horizontal directions. Hence the number of monics is the old familiar:
\[ \left(\begin{array}{c}m+n\\n\end{array}\right) \] 
Everywhere a monic 'turns a corner', we can invert that corner to find another monic. This is an adjacent face. 

The main thing to worry about are the faces of $([m]-\set k)\times [n]$. This makes certain faces $[m+n] \to [m]\times [n]$ dangerous,
precisely the faces that only hit $k$ in one point. Perhaps those are the only ones to look out for, or perhaps we risk chocking of intermediate faces the same way if we don't tread carefully. This is all very unclear, but the reason seems to be that there is no best solution, rather than that it is hard to pick the right ordering.

If we gradually diminish the number of places $p$ where $f_0(p)=k$, we gradually glue in lower dimensional monics of $([m]-\set k)\times [n]$, which makes it more likely that wanted faces are available.

A smarter choice of base point revolves are the inversion of the corner. Now we know that the faces we want are already there\dots
So the base point should move around a lot, rather than stick to the $k$-spine.

There is a clear starting point:
\begin{align*}
f(p) &= \tuplet{p,0} & p&\leq k\\
f(k+p) &= \tuplet{k,p} & p&\leq n\\
f(k+n+p) &= \tuplet{k+p,n} & p&\leq m-k
\end{align*}
Then the ordering depends on how many points a monic has in common with this one. We can always invert the least point of difference, when a monic's turn comes up.

All monics are going to have two points in common: $(0,0)$ and $(m,n)$.
If a monic $g$ has only three point in common with $f$, could the third point be outside of the $k$-spine? 
Nope: $g(p)=\tuplet{k,p-k}$ somewhere, and $f(p)=\tuplet{k,p-k}$.

How do we glue in $f$?
There are at most two faces not already present: $f\circ d_k$ and $f\circ d_{k+n}$. If $k=0$ or $k=m$ this reduces to 1, which is ideal.
Otherwise we can first glue in $f\circ d_{k+n}$, and then proceed. I am unsure if this step is unavoidable.

We order the set of monics in such a way that a monic appears earlier in the ordering, if it has more points in common with $f$.
It is more difficult than that though

Inverting a point of difference will not always bring us back all the way to $f$. 
The following measure of distance is probably better:
\[ d(f,g) = \sum_{p\of[m+n]}| f_0(p)-g_0(p) | \]
It has the property that each inversion in the right direction bring a morphism closer to $f$.

In order to glue in a new face, we need a base point and all $(n+m-1)$-dimensional faces not containing that point.

Some of these faces are present in less distant faces\dots
Vertical or horizontal stretches do not present a problem if we pick a base point on the $k$-spine, because they are present in the starting material!
The are only three options. The submonic omits a vertical point, a horizontal point or is part of a face closer to $f$. Distance therefore does the trick.

So the matter of the actual ordering remains.

The choice model makes us think about the places where vertical moves happen: $n$ places to pick out of $m+n$ ones.
We subsequently make these choices, picking one element above the other, in each step leaving enough points to work with.
We could also consider the game of increasing distances. Where do you choose to deviate form $f$.
That way we can create an ordering based on partial distance sums, or the list of pointwise distances.

Subtracting $f_0-g_0$ give a list of number that we can order lexicographically, using an ordering on numbers, e.g. based on $|n+\epsilon|$ where $|\epsilon|<0.5$.
An this gives the ordering which we will use for the faces.


\section{Triple lifting}
The triple lifting lemma is the work horse of constructive homotopy theory. The purpose of this section is to convince you that this lemma is valid in the internal language $\ambient$, because it is constructive and predicative to a sufficient amount.

\begin{lemma}[Triple lifting property] Let $f\of A\to B$ and $g\of C\to D$ be cofibrations and let $h\of X\to Y$ be a fibration. Let $a\of A\times D\to X$, $b\of B\times C\to X$ and $c\of B \times D\to Y$ satisfy $a\circ(\id_A\times g) = b\circ(f\times \id_C)$, $h\circ a=c\circ(f\times \id_D)$ and $h\circ b=c\circ(\id_B\times g)$. If one of $f$, $g$ or $h$ is acyclic, then there is a $d\of B\times D\to X$ such that $d\circ(f\times\id_D)=a$, $d\circ(\id_B\times g)=b$ and $h\circ d = c$.
\[\xy
(0,20)*+{A\times C}="AC",(25,20)*+{B\times C}="BC",(40,20)*+{X}="X",
(0,0)*+{A\times D}="AD",(25,0)*+{B\times D}="BD",(40,0)*+{Y}="Y"
\ar^{f\times\id} "AC";"BC"
\ar_{\id\times g} "AC";"AD"
\ar_{f\times\id} "AD";"BD"
\ar^{a} "AD";"X"
\ar|(.6){\id\times g} "BC";"BD"
\ar^(.6){b} "BC";"X"
\ar@{.>}_{d} "BD";"X"
\ar_(.6){c} "BD";"Y"
\ar^{h} "X";"Y"
\endxy\]
\label{triple lift}
\end{lemma}

\begin{proof} 
Cut down the variety of lifting problems with the following strategies.
\begin{itemize}
\item Using symmetry, derive the cases where $g$ is acyclic from those where $f$ is.
\item By definition, there is an equivalence between lifting acyclic cofibrations and having a filler operator for the family of horns. Use this to reduce the cases where $f$ is acyclic to the case where $f$ is the family of horns.
\item By lemma \ref{Reedy}, there is an equivalence between lifting cofibrations and having a filler operator for the family of cycles. Use this to reduce the cases where $g$ is an arbitrary cofibration to the case where $g$ is the family of cycles and do the same for $f$.
\end{itemize}
These reductions can be proved with the \emph{pullback power} construction and some diagram chasing.
\[\xy
(34,20)*+{X^C}="top",(0,10)*+{X^D}="left",(24,10)*+{\bullet}="middle",(44,10)*+{Y^C}="right",(34,0)*+{Y^D}="bottom"
\ar^{X^g} "left";"top" \ar@{.>}|(.6){h^g} "left";"middle" \ar_{h^D} "left";"bottom" \ar "middle";"bottom"
\ar "middle";"top" \ar^(.6){h^C} "top";"right" \ar_(.6){Y^g} "bottom";"right"
\endxy\]
The triple lifting problem is equivalent to the simple lifting problem of $f$ against $h^g$.
This leaves two tasks.
\begin{enumerate}
\item Prove that if $h$ has a filler operator for cycles, it has a triple filler operator for the cases where $f$ and $g$ are both the family of cycles.
\item Prove that if $h$ has a filler operator for horn, it can also handle all triple lifting problems involving the family of horns $f$ and the family of cycles $g$.
\end{enumerate}

In both cases there is a subobject of $B\times D$ that is the pushout of $f\times C$ and $A\times g$, because the underlying monomorphisms of $f$ and $g$ are decidable.
\[ \set{\tuplet{x,y}\of B\times D\middle| (x\of f(A)) \vee (y\of g(C)) }\]
Since pushouts preserve left lifting properties, a filler operator for the pushout product is sufficient.

For $m > 0$ and $n > 0$ none of the faces of $\simplex[m]\times \simplex[n]$ belong to either $\simplex[m]\times \cycle[n]$ or $\horn_k[m]\times\simplex[n]$ for any $k\of[m]$, because the faces are $m+n$-dimensional, and the faces of either $\simplex[m]$ or $\simplex[n]$ are not. The cases where $m=0$ or $n=0$ are trivial because $\simplex[0] \simeq 1$, $\cycle[0]\simeq 0$ and $\horn_0[0]$ does not exist. Since the pushout product is a countable sum of these inclusions, it is a cofibration. This settles the case where both $f$ and $g$ are the family cycle inclusions, because $h$ is an acyclic fibration.

The case where $f$ is the family of horn inclusions is more complicated. Fortunately, this case is worked out in the proof of lemma A.1 in \citep{DD&DIS11}, where \emph{inner anodyne} is what this paper calls acyclic, and \emph{box product} is pushout product. Note that all necessary properties of simplices are decidable. Since the pushout product of $f$ and $g$ is an acyclic cofibration.
\end{proof}
\hide{perhaps we should adopt the same terms as that paper.}


\hide{ 
The case for $f$ acyclic seems simpler, but perhaps pushout products are not as ubiquitous as I thought. I don't see how we can decide if a degenerate simplex belongs to the image of a cofibration. The problem is determining that $x\cdot \phi = x$ for some $\phi\of[\dim x]\to[\dim x]$, since $=$ is not always decidable. This in turn that pushout products may not be available for all cofibrations.

The trick involves regularity: the family of faces exists internally, and therefore so does its union.

Not good enough! We would need a union of faces inside. We can create a 'least pseudo-complement'. 
}



\end{document}

