\documentclass{tac}
\usepackage{amssymb, amsmath}
\usepackage[backend=bibtex,citestyle=authoryear-icomp]{biblatex}
\usepackage[all]{xy}
\usepackage{url}

\title{Constructive Simplicial Homotopy}
\author{Wouter Pieter Stekelenburg}
\copyrightyear{2015,2016,2017}
\address{Faculty of Mathematics, Informatics and Mechanics\\
University of Warsaw\\
Banacha 2\\
02-097 Warszawa\\
Poland}
\eaddress{w.p.stekelenburg@gmail.com}
\keywords{realizability, simplicial homotopy, Kan complexes}
\amsclass{03D80, 18G30, 18G55}

\newcommand\hide[1]{}
\newcommand\cat\mathcal
\newcommand\set[1]{\left\{#1\right\}}
\mathrmdef{id}
\mathrmdef{dom}
\mathrmdef{cod}
\newcommand\ri{^*}
\newcommand\N{\mathbb N}
\mathbfdef[nno]{N}
\newcommand\dual{^{\mathrm{op}}}
\mathbfdef[simCat]\Delta
\newcommand\s{^{\simCat\dual}}
\newcommand\bang{!}
\newcommand\of{:}
\newcommand\simplex\Delta
\newcommand\cycle{\partial\Delta}
\newcommand\horn\Lambda
\newcommand\f{_f}
\newcommand\tuplet[1]{\left\langle #1 \right\rangle}
\newcommand\true{\mathtt{true}}
\newcommand\false{\mathtt{false}}
\newcommand\bool{\mathtt{bool}}
\mathrmdef{nat}
\mathssbxdef{Ar}
\mathssbxdef{Ob}
\newcommand\pp{\mathbin\diamond}
\newcommand\norm[1]{\Vert #1 \Vert}
\newcommand\ka\kappa
\newcommand\la\lambda
\mathrmdef{face}
\mathrmdef{colim}
\newcommand\ex{_{\textrm{ex}}}
\newcommand\citep[1]{[\cite{#1}]}
\mathrmdef{dim}
\newcommand\base{\mathbf{U}}
\mathssbxdef[sub]{Sub}
\newcommand\ambient{\mathfrak A}
\mathrmdef{uni}
\newcommand\disc{_{\rm disc}}
\mathrmdef{filler}
\newcommand\traco\omega

\newcommand\product[2]{\Pi #1 \mapsto #2}
\newcommand\coproduct[2]{\Sigma #1 \mapsto #2}
\newcommand\function[2]{\lambda #1 \mapsto #2}

\newcommand\keyword[1]{\emph{#1}\label{#1}}


\begin{document}

\section{30/7/17}

There is a family of special fibration that withstand pulling back. The really hard part is connecting the descend to this family.

\paragraph{All problems}
The object of all problems is 
\begin{align*} 
P &= \set{\tuplet{\xi,k,l}|\xi\of\Ar(\simCat),k\of\dom(\xi),l\of\cod(\xi)}\\
p\tuplet{\xi,k,l} &= \tuplet{\dom(\xi),k}
\end{align*}
This is the one we want. The $p$ is the one we make the power over. The power is fine but the other object is much harder to do. The basis is still okay:
\[ q\tuplet{\xi,k,l} = \dom(\xi)+\norm\xi_l \]

The representables of $\ambient\s/\set\leq\disc$ are what I used as the key. That will work.

\section{28/7/17}
The global structure changes dramatically.

If $V$ is the universe, the morphism $\ambient\s/\nno\disc(\simplex_{\set\geq},V)\to \ambient\s/\nno\disc(\horn_{\set\geq},V)$ is a split epimorphism. 
The object $P=\ambient\s/\nno\disc(\horn_{\set\geq},V)$ is the object of lifting problems. By descending along $\horn_P\to \simplex_P$ we get where we want.
After we pin down what the descendant looks like, we just need to demonstrate that it is a modest assembly.
Here is where the real pain lies.

Fatten the simplex and then pull back. That is all we need to make things work.

Now the modesty is just in the way. All we really need is a grasp of the problem. 
All combinations.
Somehow just showing that horn inclusions are equivalences.

We make descend adjoint to a construction which is not the same as simply pulling back,
Although it comes close. Could we skip ahead and describe 

Nothing really changes about the structure of the proof.

\section{14/7/17}
Since $\ambient\s$ is sort of like a topos, we can represent its objects as coequalizers between sums of simplices. This representation allows us to extends any functor $K\of \simCat \to \simCat$ to a functor $\ambient\s\to\ambient\s$: the right Kan extension of $Y$ along $YK$ if $Y$ is the Yoneda embedding.
Technically we have a far larger class of lifting problems to address internally. We need a functor $\ambient\s/\simplex_{\set\leq}\to\ambient\s/\horn_{\set\leq}$, which has a right adjoint and which preserves cofibrations.

We can subdivide a little, and use the family of horn inclusion to define a model structure first.

\parargraph{recapitulate}
We need a lifting operator for the universe of modest objects. We realize it by descending fibrations along the family of horn inclusions. The descent operator must produce a fibration whose pull back is the original fibration. We turn this into a functor, which has a left adjoint, and then prove the left adjoint preserves cofibrations.

Ultimately we bring it down to a single family that is proven to be a family of cofibrations.

The descend construction itself matters very much and must be applicable to all modest fibrations.
Proving that the descended morphisms are fibrations is done by reducing the lifting problems of the descended morphisms to lifting problems of the original morphisms. This is what the left adjoint is useful for. Finally, 


For each $\xi\of [m]\to[n]$, $k\of[m]$ and $l\of[n]$ there is a strengthened cofibration 

\paragraph{strategy}
We want to put the family of horn inclusions central, so let's just rewrite the section that way, and see where we get stuck.





\section{30/6/17}
Left to do:
\begin{enumerate}
\item ensure that every proof of cofibrancy satisfies the new definition.
Done.

\item continue the other updates, namely the new family lifting property.
\item restructure the descend proof to be less hand-wavy as well.

\item make the proof that contractible morphisms are acyclic fibrations more readable.
\item surjective or epic in $\simCat$, make up your mind!
Epic is it.
\end{enumerate}


Technically $D\of \ambient\s/\simplex_{\set\leq} \to \ambient\s/\horn_{\set\leq}$.
The reason $D$ preserves cofibrations is that it is a 'partial' left adjoint $K$ that preserves acyclic cofibrations. It is in the wrong place to a 

The functor $D$ is indeed a Kan extension, but of a right adjoint to $K$.
That is how it preserves fibrations.

The problem object combines a morphism $\phi\of\simplex[m]\to\simplex[n]$ with horns at both simplices. Worse still, to prove that it has the lifting property, we need to split things up even further.

There is no Kan extension of $K$ along $\Delta$, but we can do every power and coequalizer.
These would be the enriched projectives and the things they cover.

Since we are working with enriched categories anyway, projective makes sense. The projectives of $\ambient\s$ in this sense and the objects that have a projective resolution permit the extension of $K$.

Projective resolutions of horns: the set of faces of codimension $1$ and a power of those of codimension $2$. Do these help in any way?

In any case, we can do the constructions with these objects,
and since the problem object is in here, we have enough to proof properties of $D$.

\paragraph{density}
Acyclic cofibrations are dense up to homotopy. I now expect no deeper relation between the descend construction of the paper and density or codensity monads than that.

The functor $(K_0,K_1)$ fails to be a monad because of the ordering we force on the supporting points. Without it, the reset of the construction doesn't work. But the ordering arbitrarily chosen to be the lexicographic one prevents a notion of bind.

Moving on: what about $D$? Left Kan extension of $K$ along $D$ anyone?

That makes it clearer why the projective thing would work out.
In that case the elements are 'nice colimits' of simplices, 
that can be replicated on the other side.

\paragraph{new setting}
Coproduct of categories over $\set\leq$. But then, we are considering simplicial object over other simplicial object, so there might be a more direct definition\dots

$D\of \ambient\s/\simplex_{\set\leq} \to \ambient\s/\horn_{\set\leq}$


\paragraph{projectives are cofibrants}
The category of projectives should be equivalent to the category $\cat P$ where objects are pairs $\tuplet{P,p\of P\to\nno}$ of objects and morphisms of $\ambient$, and where a morphism $\tuplet{P,p}\to \tuplet{Q,q}$ is a morphism $f\of P\to \times Q\times \Ar(\simCat)$ such that $f_1(x)\of [p(x)]\to [q(f_0(y))]$, i.e. like a tracked morphism in a sense.
Essentially, families of finite well orderings.

We can cover the family of horn inclusions with these, which is all we need.

Hey, notice something? Our cofibrations are projective morphisms in a suitably modified sense, just as intended. We have described cofibrant objects, which cover everything.

What we need however is the covering by regular epimorphisms, which may differ from that by acyclic fibrations. The 'latching' construction in the factorization proof. I get the feeling that the equivalence relation is already cofibrant. A tuplet $\tuplet{\epsilon_0,y_0,\epsilon_1,y_1}$ where $y_0\cdot\epsilon_0 = y_1\cdot\epsilon_1$, is nondegenerate if $(\epsilon_0,\epsilon_1)\of [a]\to[b]\times[c]$ is monic. Now we have an equivalence relation whose quotient is the original simplex.

I don't trust this. These structures aren't coproducts of simplices, like the projectives I look at before. So something is up.
Coverings with tensors of the family of simplices are everywhere though, so it may not be a great problem.

So $K$ does extends to all of $\ambient\s$. Every object is the quotient of a pseudoequivalence of cofibrant objects. The functor $K$ easily extends to cofibrants, and their quotients, hence all of $\ambient$. So $D$ has a proper left adjoint. This left adjoint has to send the family of horn inclusions to cofibrations.



\section{16/6/17}
What is the task that I have been postponing?
\begin{enumerate}
\item update the definition of cofibration once more, to say that the set of complementary faces covers the complement of the image.
\item ensure that every proof of cofibrancy also proves this extra condition.
\item continue the other updates, namely the new family lifting property.
\item prove anew that contractible morphisms are acyclic fibrations.
\item restructure the descend proof to be less hand-wavy as well.
\item epic vs surjective in $\simCat$--make up your mind.
\end{enumerate}

I tend to focus on the link between cofibrations and the saturated class of morphism generated from the family of cycle inclusions.
The recursive lifting constructions are a problem, however.

I keep missing what I really need, but degeneracy should be definite.

Suppose we definite degenerate as follows: $x=y\cdot\epsilon$ for a surjective $\epsilon$ such that $\epsilon\neq\id$.
Can we prove the current positive definition of face?
Suppose $x=x'\cdot\xi$ for some endomorphism $\xi$, but $x$ is nondegenerate. By em-factorization $x = x \cdot m(\xi)\cdot e(\xi)$, but now $\neg\neg(e(\xi)=\id)$, because of nondegeneracy, and $e(\xi) = \xi$ because equality between these morphisms is decidable.



\section{11/6/17}
Two approaches make sense. Add the condition that simplices are
generated or define acyclic cofibrations the hard way.
The latter doesn't make much sense.

We can glue in new faces. Therefore the condition that the decided
faces generate everything.

\section{5/6/17}
The lifting operator should be a family of morphism, otherwise
the piece-wise application doesn't make much sense.

Indeed, work with families of morphisms until the last step.

We have filler operators:
\[ f_n\of \hom(K_n,X)\times_{\hom(K_n,Y)}\hom(K_{n+1},Y) \to \hom(K_{n+1},X) \]
We define a new family of fillers using the following equations:
\begin{align*}
g_n &\of \hom(K_0,X)\times_{\hom(K_0,Y)}(\product{m\of\nno}\hom(K_m,Y))\to \hom(K_n,X)\\
g_0(a,b) &= a\\
g_{n+1}(a,b) &= f_n(g_n(a,b),b_{n+1})
\end{align*}
The $b$ must be a family as well and it probably must satisfy certain equations, like $b_0 = p\circ a$ and $b_{n+1}\circ c_n = b_n$.
So that is a factor.
Better yet $b\of K_\infty \to J$, etc.

The last step is a pushout to get to the cofibrations.

Rebuild the contractibles are acyclic proof. In need to reconsider the notation, but that can wait.

\paragraph{confusion}
I took for granted that the set of faces outside of the image of a cofibration is a set of generators, in the sense that each simplex is either part of the domain of the cofibration, or comes form one of the faces. Is it?

If we merely get a decidable subobject of generic simplices, do we get in trouble? Don't we also get the faces then?

Damage assessment:
\begin{enumerate}
\item we prove the decidability of the set of faces in order to show that acyclic fibrations are contractible.
\item the first factorization proof might actually become simpler
with a set of generators
\item the proof in the other direction is our problem now.
\item doesn't seem to come up futher down the road.
\end{enumerate}

So what would be the danger of switching to a decidable subject of generators?
So we loose decidability of genericity\dots do we need that?
Once again the uses are restricted to this section, where we are better of with simple generators!

A decidable set of generators. If we pull back a cofibration with this property, we get a new set of generators, hence no pain at all.

This is good.

I still feel like there is some danger. We cannot simply glue faces in and expect isomorphic results. I am afraid that some cofibrations are not going to be pushouts of powers of the family of cycles.

Why did I define cofibrations differently anyway?

Just adding the assumption that the set of faces outside of the image generates everything else corrects an oversight on my intuitions. It is probably provable that every simplex comes from a face using classical logic, by deriving a contradiction from the assumption that no restriction is nondegenerate.

The problem is equality. In a classical setting $j\cdot \xi = j$ is either true or false. This is a luxury we don't have here, so degeneracy is no decidable. This hurts us more often than expected: when we have a set of generators without the guarantee of non degeneracy, we can never get it back when we need it.

\section{4/6/17}
Let $c\of I\to J$ be a cofibration and let $\phi\of\base J\to\bool$ classify the subject of faces of $J$ that aren't faces of $I$.
Let $K\subseteq \nno\times J$ be the simplicial object that contain $\tuplet{n,j}$ if $\neq\phi(j\cdot\mu)$ for all monic $\mu$ with $\dom(\mu) > n$.
In other words, simplices are only admitted if all of their restrictions are either in $I$, degenerate or of dimension less than $n$.
Let $s\of K\to K$ be the map $\function{\tuplet{n,j}\of \nno\times K}\tuplet{n + 1,j}$ and use $b = \function i\tuplet{0,c(i)}\of I\to K$
The familiar part is proving the left lifting property for $\tuplet{b,s}\of I + K\to K$, as it is a pushout of a power of $\cycle_\nno \to \simplex_\nno$.
It feels stupid to need more than one pushout, but I am unsure what else to do.

The second part involves constructing a filler for $\function{\tuplet{n,i}\of \nno\disc\times I}\tuplet{n,c(i)}$ with induction.
The situation: $c=\tuplet{c_0,c_1}\of I + K\to K$ has the left lifting property. Now define $d\of\nno\disc\times I\to K$ recursively, by $d\tuplet{0,x} = c_0(x)$ and $d\tuplet{n + 1, x} = c_1(d(n,x))$. Then $d$ also has the left lifting property.

Suppose $f\of \hom(I+K,X)\times_{\hom(I+K,Y)}\hom(K,Y) \to \hom(K,X)$ is a filler operator for $c$ against $p$.
Define $g\of\hom(\nno\times I,X)\times_{\hom(\nno \times I,Y)}\hom(K,Y) \to \hom(K,X)$ as follows:
\begin{align*}
g(a,b) &= f(\tuplet{a_0,g(a_s,b)},b)\\
\end{align*}
That looks nice, but does it actually define a total function that satisfies the required equations?
The equation is correct, i.e. a function that satisfies it is a filler for $d$ against $p$. It does not define the function well, however.
This can be proved by induction I guess.

Assuming $\tuplet{n,j}\of\dom(g)$ for $n < m$, $\tuplet{m,j}\of\dom(g)$ because\dots this works out when $K$ is graded.

\section{2/6/17}
The goal is weighted coproducts. Now the discrete object are weighted coproducts of the terminal object, so that is a promising starting point.


The family lifting property is the normal lifting property in an alternative enrichment.

\paragraph{family lifting property}
The lifting property takes the enrichment as implicit parameter, but perhaps this should be an explicit parameter\dots
If there are a bunch of different enrichment, this might be a good idea.


I corrected the definition of fibration now. The next step involved adjusting the proves as well,
Especially 

\begin{itemize}
\item improve the definition of contractible as well.
\item use the new definition of fibrant and contractible in every proof.
\item replace coproducts with subobjects where possible.
\item track down where transfinite composition is used implicitly, make the argument explicit.
  * page 9, lemma 3.6 the end.	
\item try to get rid of transfinite compositions.
\end{itemize}

I updated the definition of lifting property with respect to a family. This permits a concrete way to write down certain proofs, which I am also introducing.
I do find that I remembered some forgotten aspects of earlier structures. As usual, my memory simplifies the proofs by erasing vital details, only hanging on to the general shape. Without these vital details some large scale restructuring becomes easy to imagine, but the chance that the new structure actually work becomes smaller.

I mean, I rediscovered the idea of an object of lifting problems, a vital detail that I didn't manage to put down clearly enough before, and gradually removed as I rewrote the paper. The same happened with transfinite compositions. I had them in a way that obscured their role in the central proofs. Now I miss them.

\section{26/5/17}

Transfinite compositions are not used explicitly anywhere, and don't seem like much of an addition.

Okay, suppose $f\of X\to X$ is a chain. Feels like the inclusion $X_0 \to X - f(X)$ should be the transfinite composition. Of course, $X - f(X)$ is not a simplicial set, unless we can adjust the restriction operator, but the existence of a transfinite composition suggest than we can.

Important change: rather than defining loose simplicial sets, I define the whole family in one go.
This may have impact on later parts of the paper.

Perhaps we better replace all coproducts with more direct definitions.

So now we have a longer to do list:
\begin{itemize}
\item repair found errors
\item don't use slice categories in the definition of enriched lifting\dots
\item replace coproducts with subobjects where possible.
\item track down where transfinite composition is used implicitly, make the argument explicit.
  * page 9, lemma 3.6 the end.
	
\item try to get rid of transfinite compositions.
\end{itemize}

Important insight: left adjoints preserve left lifting properties and right adjoint preserve right lifting properties. 
Hence the family lifting property reduces to the underlying lifting property!

What we needed comes from the enriched part of the definition! This is now very unclear.
Perhaps we need a lemma for this or something, but let's just see how far we get today.

As it stands, the text suggest that a lot of power comes from defining the lifting property in the slice category. It does not. 
There is an option to pull back morphism with left lifting property along \emph{discrete morphisms}, because of the enrichment. 
That is the power I wanted, and should be using explicitly throughout my proofs.

Another insight: the coproducts I seek are \emph{weighted coproducts}.

But something seems to be missing now: how to get weighted colimits of members of the family?

We are looking for a new proof of:
\begin{lemma} Contractible morphisms are acyclic fibrations.\end{lemma}
This seems a construction that actually breaks up the family filler operator internally.
We need to make the case that this is permitted somehow\dots
It doesn't follow from general rules about cofibrations. It must use discreteness somehow.

Actually, we should have argued that the enrichment is over $\ambient/\nno$.
That is where all problems seem to melt away.

Perhaps there is a trick that makes all individual inclusions retracts of the family. 
That might just be the most promising option.
The only trouble is $0\to 1$.


I have been needing a sum of hom objects, but got a product instead.
That is an error. However, when I have repaired the definition, then we have what we need everywhere.

This is applied stack semantics:
The slice $\ambient\s/I\disc$ is enriched over the slice $\ambient/I$: $\nat(X,Y)[i] = \nat(X[i],Y[i])$.

\paragraph{yet another def}

\begin{definition} A morphism $f\of X\to Y$ of $\ambient\s$ is a \keyword{fibration} it is has a \emph{pointwise} right lifting property with respect to the family of horn inclusions.

Let $\set\geq = \set{\tuplet{n,k}\of \nno\times\nno |n\geq k}$ in $\ambient$.
The simplicial set $\simplex_{\set\geq}$ satisfies:
\begin{align*}
\base(\simplex_{\set\geq}) &= \set{\tuplet{\phi,k}\of\Ar(\simCat)\times\nno| k\leq \cod(\phi) } \\
\dim\tuplet{\phi,k} &= \dom(\phi) \\
\tuplet{\phi,k}\cdot \xi &= \tuplet{\phi\circ\xi,k}
\end{align*}
The simplicial set $\horn_{\set\geq}$ satisfies:
\begin{align*}
\base(\horn_{\set\geq}) &= \set{(\phi,k)\of\Ar(\simCat)\times\nno\middle| 
\begin{array}{l}
k\leq \cod(\phi),\\
\exists i\of\nno.\left\{\begin{array}{l}i\leq \cod(\phi), i\neq k,\\ \forall j\of \dom(\phi).\phi(j)\neq i \end{array}\right. 
\end{array}
} \\
\dim\tuplet{\phi,k} &= \dom(\phi) \\
\tuplet{\phi,k}\cdot \xi &= \tuplet{\phi\circ\xi,k}
\end{align*}
The simplicial object $\horn_{\set\geq}$ is subobject of $\simplex_{\set\geq}$ by definition. 
The inclusion $\horn_{\set\geq}\to\simplex_{\set\geq}$ is the family of all horn inclusions.

The formula $\tuplet{\phi,k}\mapsto \tuplet{\dom(\phi),\cod(\phi),k}$ defines morphisms of simplices $\horn_{\set\geq}\to\set\geq\disc$ and $\simplex_{\set\geq}\to\set\geq\disc$ that commute with the inclusion, and make it a morphism in $\ambient\s/\set\geq\disc$. This category $\ambient\s/\set\geq\disc$ has an enrichment in $\ambient/\set\geq$, where $\nat(X,Y)$. If $\set\geq\ri(f)$ has the right lifting property with respect to the inclusion $\horn_{\set\geq}\to\set\geq\disc$ relative to the enrichment on $\ambient/\set\geq$.
\end{definition}

Perhaps this requires a new section, even if we use the notion only twice.

I suppose the family thing is quite clear and correct. But this enrichment stuff doesn't clear much up.
closer to the core, we could introduce a parametric variant of the enriched lifting property.

We could introduce the notion of lifting versus a discrete family from the start. 

Lifting discrete families as a core concept, to be introduced right after the enriched factorization system.

I know what I need, but as usual, it is a struggle to write down.

\section{28/4/17}
Dwelling on the contractible-cofibration factorization. The latching an matching elements evidently are part of fibrant and cofibrant replacements, and perhaps the proof becomes clearer if we somehow use the replacements first. I don't see it working however.

Plan: print out and check for mistakes; try to publish once again? At the same time, look for computer verification.


\section{24/3/17}
Two parts stay difficult to understand, because they generalize in the internal language.
\begin{itemize}
\item proposition on the factorization of $f$ as a contractible morphism following a cofibration.
\item fibrancy of the universal modest fibration
\end{itemize}

The trouble with the factorization must be that we need to show a filler operator for the family of cycles. Other than a rewording of the current text, I don't see much to improve here.

This whole part is based on Reedy's work, and it would be better to reflect that somehow.
I don't know how to do that yet.

\paragraph{Some observations on the relation with Reedy's work}
Reedy shows how a model structure on a small complete category can be lifted to functor categories. My starting point neither has a model structure, nor all colimits.
My construction combines the latching and the matching factorizations of the Reedy categories to yield only one of the factorizations of the model structure.
The latching part seem to ensure cofibrations, while the matching part ensures fillers.
The external induction hurts. Everything should be related to the internal natural number object.
Note that Reedy factorization is quite complicated, and the paper tries to skip to the result.

There is a factorization of morphisms in the ambient category based on decidable monomorphism following split epimorphisms.
Reedy's construction lifts this to the category of simplicial objects.
Cofibrations are decidable monomorphism lifted as Reedy defines.
Contractibles are split epi's lifted as Reedy defines.
It stands to reason that his proof could be useful, if it weren't for the external induction in it.

It is good to look up the definitions again, if only for the confirmation that the proof in the paper is not more complicated than necessary.

\section{10/3/17}
We are working out exactly what a \emph{horn or cycle filler} operator is, just like we worked out what a simplicial set is, in hopes of making the proof clearer and more convincing.

The lifting property is more demanding because it has to take all cases into account.
I work in all kinds of slice categories to hide to gory details, and keep track of additional requirement:
For a particular $f\of X\to Y$ and the family of horns $h\of D\to C$.
\begin{align*}
P &= \set{ \tuplet{a,b}\of(D\to X)\times(C\to Y)| f\circ a = b \circ h }\\
\dim(a(\tuplet{\phi,k})) &=\max(\cod(\phi))\\
\dim(b(\tuplet{\phi,k})) &= \max(\cod(\phi))\\
a(\tuplet{\phi,k})\cdot\chi &= a(\tuplet{\phi\circ \chi,k}) \\
b(\tuplet{\phi,k})\cdot\chi &= b(\tuplet{\phi\circ \chi,k}) \\
\end{align*}
\begin{itemize}
\item The cycle case is simpler, because we can drop the $k$-index.
\item The filler satisfies the same restrictions on dimensionality and restrictions, plus $c\circ h = a$ and $f\circ c = b$.
\end{itemize}

We now get a simplified notion of what a filler operator for both families are, and can try to rewrite the proofs in order to build exactly such an operator.

Suppose we work out an official definition of horn and cycle fillers. What might happen?
The step from the filler to lifting cofibrations may be less obvious, requiring some updates of the proofs and definition.
Not the acyclic though, just the weakly invertible.

\newcommand\HLP{\mathrm{HLP}}
\begin{definition}
Let  Let $f\of X\to Y$ be an arbitrary morphism of $\ambient\s$. The object of horn lifting problems is 
\[\HLP(f) = \set{\tuplet{a,b}\of \base X^D\times \base Y^C \middle|\begin{array}{l}
  f\circ a = b\circ h,\\
  \dim(b\tuplet{\phi,k}) = \max(\dom\phi),\\
  b\tuplet{\phi,k}\cdot\chi = b\tuplet{\phi\circ \chi,k},\\
	a\tuplet{\phi,k}\cdot\chi = a\tuplet{\phi\circ \chi,k}
\end{array}}\]
Here $h\of D\to C$ is the inclusion of the following objects in $\ambient$.
\begin{align*}
C &= \coproduct{\phi\of\Ar(\simCat)}\cod(\phi) \\
D &= \set{\tuplet{\phi,k}\of C \middle| \exists i\of\cod(\phi)-\set k. \forall j\of\dom(\phi).\phi(j)\neq i}
\end{align*}
A \keyword{horn filler} is a morphism $c\of\HLP(f)\to \base X^C$ that satisfies the following equations. 
\begin{align*} c\tuplet{a,b}\circ h &= a & f\circ c\tuplet{a,b} &= b \end{align*}
A \emph{fibration} of $\ambient\s$ is a morphism with a horn filler.
\end{definition}

%Cycle fillers 
\newcommand\CLP{\mathrm{CLP}}
\begin{definition}
Let $f\of X\to Y$ be an arbitrary morphism of $\ambient\s$. The object of cycle lifting problems is 
\[\CLP(f) = \set{\tuplet{a,b}\of \base X^D\times \base Y^C \middle|\begin{array}{l}
  f\circ a = b\circ h,\\
  \dim(b(\phi)) = \max(\dom\phi),\\
  b(\phi)\cdot\chi = b(\phi\circ \chi),\\
  a(\phi)\cdot\chi = a(\phi\circ \chi)
\end{array}}\]
Here $h\of D\to C$ is the inclusion of the following objects in $\ambient$.
\begin{align*}
C &= \Ar(\simCat) \\
D &= \set{\phi\of C \middle| \exists i\of\cod(\phi).\forall j\of\dom(\phi).\phi(j)\neq i}
\end{align*}
A \keyword{cycle filler} is a morphism $c\of\CLP(f)\to \base X^C$ that satisfies the following equations. 
\begin{align*} c\tuplet{a,b}\circ h &= a & f\circ c\tuplet{a,b} &= b \end{align*}
A morphism  of $\ambient\s$ is \emph{contractible} if it has a cycle filler.
\end{definition}

Where do we have problems with generalizations?
\begin{itemize}
\item proposition on the factorization of $f$ as a contractible morphism following a cofibration.
\item the triple lifting property. Improved now.
\item if $f$ and $g\circ f$ are contractible, then so is $g$. Improved now.
\item fibrancy of the universal modest fibration
\end{itemize}

\section{Triple lifting property}
We derive this from the equality of contractible and the acyclic fibrations. The core argument is that $\tuplet{f\to g}$ is a fibration if $f\of A\to B$ is a cofibration and $g\of X\to Y$ is a fibration, and contractible if $g$ is contractible or $f$ is acyclic. So how to define the horn and cycle fillers?

\[ \tuplet{f\to g} = \function{x\of X^B}\tuplet{g\circ x,x\circ f}\of X^B \to Y^B\times_{Y^A} X^A \]

Another approach is to take the pushout product of the families and construct the desired fillers there.

Take push out product of the family of horns with the family of cycles, then find a filler.
If $f$ is contractible, take the pushout product of the family of cycles with itself, and find a filler for that too.
Then use adjoints etc. to prove the lemma.
We may be able to pull it of without products, however. The triple lift for the families is a special case form which we derive all the special cases.

How far would we get with just the family of cycles along an (acyclic) cofibration?

The ultimate point is that we can fill the pushout product by systematically using fillers for simplices.
That algorithmic reduction of a triple lifting problem to a horn or cycle lifting problem can be done inside $\ambient$ to show that elementary triple lifting problems have a solution if horn or cycle lifting problems have. There rest is the reduction of the general triple lifting problem to the product of the families.

The reduction uses monotone monics $[m + n]\to[m]\times[n]$. 
We start with a collection of $[m+n-1]$-dimensional faces in the pushout product. Unless $m=0$ or $n=0$ we have no $[m+n]$ dimensional faces however. So these need to be glued in, one by one.
These can be lexicographically ordered again, after which we ensure that they get filled.
I.e. we build a sequence $A_i\to A_{i+1}$ of approximations and show that each requires a well defined application of the preexisting filler operator.
Clearly, the result is a proof in the same style as the descent proof, with one important difference: we don't rely on saturation, but an a filler operator to do the job.

\paragraph{set up for triple lift}
We need to fill $\simplex[m]\times \cycle[n]\cup \horn_k[m]\times\simplex[n]$. The first subdivision is along $\set{k}\times \simplex[n]$, whose point we will use as tops of horns to fill. The second subdivision is along $[m]-\set{k}$. We include down sets into each new $m+n$ dimensional face we glue in, until we are completes. The last subdivision is along dimension, as it seems we need face completion to get the required horns.

The main point of course is that we have a construction and therefore a filler operator that works on arbitrary horns.

Can we focus on greater collections than individual simplicies? At least in the intermediate steps we can\dots

\paragraph{considerations}
What matters for the case c-c-af is that the pushout product is a cofibration, which requires no reductions.
The cases c-ac-f and ac-c-f are where the reduction to the family of horns is needed.
Still, we should be comfortable with just the case of a pushout product of the family with an arbitrary cofibrations.

\paragraph{subdivisions}
We want to systematically subdivide the problem of filling
\[ A = \simplex[m]\times \cycle[n]\cup \horn_k[m]\times\simplex[n] \to \simplex[m]\times \simplex[n] \]
Damn this is hard.

Each monic $[m+n]\to [m]\times [n]$ must be glued in at some stage. This gluing in possible if we have a horn.
That is a base point $l$ such that the composition with monics $[m+n-1]\to [m+n]$ that hit $l$ have already been included at an earlier stage.

One thing we know: composition of these monotone monic with the projections $\to[m]$ an $\to[n]$ are surjective. Why?
By induction over $m+n$. Case 0 is trivial. Otherwise the second last element must be lesser in one of the variables,
hence inside the $m+n-1$ dimensional subsquares. The induction hypothesis makes the compositions of the submonic surjective. This forces $f(m+n) = (m,n)$, answering when either of those are hit.

What do we start with?
\begin{itemize}
\item All monics $[m+n-1] \to ([m]-\set{i})\times[n]$, where $i\neq k$.
\item All monics $[m+n-1] \to [m]\times ([n]-\set{j})$
\item No monics $[m+n-1] \to ([m]-\set{k})\times[n]$
\end{itemize}
For each face we glue in, the intersection with the previous complex cannot be a cycle however.
Given a supporting point, there can be no monics that omit it.
This is why the spine $\set{k}\times[n]$ is the unique supply of supporting points.

Monotone monics cannot meander. The set of elements that intersect the spin is always a simple interval.

What worries me is that adding any monic necessarily adds all the monics that miss each point along the spine.

Suppose $M_j$ consists of monics that intersect the $k$-spine in $j+1$ places. Is this the proper selection?
Do we every get any monics for free? I.e. without explicitly gluing them in?

Let's try the most difficult situation and hope for the best.

$M_{i,j}$ where $i\leq j\of[n]$ is the set of monotone monics that hit the points $(k,i)$ to $(k,j)$.
For gluing in $f$ we pick $(k,i)$ as top. Now $f\circ d_l$ for $l\neq k+i$ should already be present, while
$f\circ d_{k+i}$ should not.

We look for the first point where $f(k+j+n)_1 > j$ and swap all those points!?
Nope, we need to subdivide one step further.

Better approach: the 'distance' of two monotone monics is the number of places they don't coincide.
There is a specific monic that we start with, and then we progressively add in monics at greater distances.
Note the law $f(k)_0+f(k)_1=k$, which may be even better than out surjectivity claims.

So let's try again: $f_0(l) = \tuplet{l,0}$ if $l\leq k$, $f_0(l) = \tuplet{k,l-k}$ if $l$ is between $k$ and $k+n$, and 
$f_0(l) = \tuplet{l-n,n}$ if $l\geq n+k$. We gluing the faces in order of their distance (a number of steps from the basis figure)
to this monic. That way, we can rely on the overlap to prove required faces are present.

The only problem is choosing the base point is such a way that the opposing face is absent. Can we always do this?

Let $f$ be glued in, and $g$ be different from in in point $p\of[m+n]$. For $g\circ \partial_q$ if $q\neq p$ we know those faces are available.
We also need a base point, however. Some place where $f_0$ and $g_0$ are both $k$.
Then we note that all opposites to where $f_0=k$ have been added however\dots
What am I saying? All subfaces of $g$ are different from $f$ except $g\circ \partial_p$!

In general, the surjectivity ensures we can use the $k$-spine to glue in any face. What we need is the opposite:
a proof that the face opposing the chosen base point has not already been glued in!

So for $f$ let the least $p$ such that $f(p)_0=k$ be the base point. We would hate for $f\circ\partial_p$ to be present, and there is exactly one face that could case this tragedy: $f + (p \mapsto \tuplet{k-1,f_1(p)+1})$. Its least point is $p+1$, so that gives us a criterion for ordering faces: the least points that hit $k$ 

The existing subfaces seem like causes for $f\circ\partial_p$ existence, but that requires either coordinate of $f(p) = \tuplet{k,x}$ not to occur anymore.
In the first case, that forces $f(p-1)_0=k-1$ and $f(p+1)_0=k+1$, which is in the missing $([m]-\set k)\times [n]$ area. In the second, $f(p-1) = \tuplet{k,x-1}$, which means that $p$ is not the least point, contradicting an assumption. We in fact hardly notice the preexisting faces.

The main part is then 'face completion': we have a base point and dimension by dimension glue in the necessary faces, noting that we already have all the $[m+n-2]$ ones already.

\paragraph{Setting up again}
More analysis: every monic can be seen as a sequence of increments in either vertical or horizontal directions. Hence the number of monics is the old familiar:
\[ \left(\begin{array}{c}m+n\\n\end{array}\right) \] 
Everywhere a monic 'turns a corner', we can invert that corner to find another monic. This is an adjacent face. 

The main thing to worry about are the faces of $([m]-\set k)\times [n]$. This makes certain faces $[m+n] \to [m]\times [n]$ dangerous,
precisely the faces that only hit $k$ in one point. Perhaps those are the only ones to look out for, or perhaps we risk chocking of intermediate faces the same way if we don't tread carefully. This is all very unclear, but the reason seems to be that there is no best solution, rather than that it is hard to pick the right ordering.

If we gradually diminish the number of places $p$ where $f_0(p)=k$, we gradually glue in lower dimensional monics of $([m]-\set k)\times [n]$, which makes it more likely that wanted faces are available.

A smarter choice of base point revolves are the inversion of the corner. Now we know that the faces we want are already there\dots
So the base point should move around a lot, rather than stick to the $k$-spine.

There is a clear starting point:
\begin{align*}
f(p) &= \tuplet{p,0} & p&\leq k\\
f(k+p) &= \tuplet{k,p} & p&\leq n\\
f(k+n+p) &= \tuplet{k+p,n} & p&\leq m-k
\end{align*}
Then the ordering depends on how many points a monic has in common with this one. We can always invert the least point of difference, when a monic's turn comes up.

All monics are going to have two points in common: $(0,0)$ and $(m,n)$.
If a monic $g$ has only three point in common with $f$, could the third point be outside of the $k$-spine? 
Nope: $g(p)=\tuplet{k,p-k}$ somewhere, and $f(p)=\tuplet{k,p-k}$.

How do we glue in $f$?
There are at most two faces not already present: $f\circ d_k$ and $f\circ d_{k+n}$. If $k=0$ or $k=m$ this reduces to 1, which is ideal.
Otherwise we can first glue in $f\circ d_{k+n}$, and then proceed. I am unsure if this step is unavoidable.

We order the set of monics in such a way that a monic appears earlier in the ordering, if it has more points in common with $f$.
It is more difficult than that though

Inverting a point of difference will not always bring us back all the way to $f$. 
The following measure of distance is probably better:
\[ d(f,g) = \sum_{p\of[m+n]}| f_0(p)-g_0(p) | \]
It has the property that each inversion in the right direction bring a morphism closer to $f$.

In order to glue in a new face, we need a base point and all $(n+m-1)$-dimensional faces not containing that point.

Some of these faces are present in less distant faces\dots
Vertical or horizontal stretches do not present a problem if we pick a base point on the $k$-spine, because they are present in the starting material!
The are only three options. The submonic omits a vertical point, a horizontal point or is part of a face closer to $f$. Distance therefore does the trick.

So the matter of the actual ordering remains.

The choice model makes us think about the places where vertical moves happen: $n$ places to pick out of $m+n$ ones.
We subsequently make these choices, picking one element above the other, in each step leaving enough points to work with.
We could also consider the game of increasing distances. Where do you choose to deviate form $f$.
That way we can create an ordering based on partial distance sums, or the list of pointwise distances.

Subtracting $f_0-g_0$ give a list of number that we can order lexicographically, using an ordering on numbers, e.g. based on $|n+\epsilon|$ where $|\epsilon|<0.5$.
An this gives the ordering which we will use for the faces.


\section{Triple lifting}
The triple lifting lemma is the work horse of constructive homotopy theory. The purpose of this section is to convince you that this lemma is valid in the internal language $\ambient$, because it is constructive and predicative to a sufficient amount.

\begin{lemma}[Triple lifting property] Let $f\of A\to B$ and $g\of C\to D$ be cofibrations and let $h\of X\to Y$ be a fibration. Let $a\of A\times D\to X$, $b\of B\times C\to X$ and $c\of B \times D\to Y$ satisfy $a\circ(\id_A\times g) = b\circ(f\times \id_C)$, $h\circ a=c\circ(f\times \id_D)$ and $h\circ b=c\circ(\id_B\times g)$. If one of $f$, $g$ or $h$ is acyclic, then there is a $d\of B\times D\to X$ such that $d\circ(f\times\id_D)=a$, $d\circ(\id_B\times g)=b$ and $h\circ d = c$.
\[\xy
(0,20)*+{A\times C}="AC",(25,20)*+{B\times C}="BC",(40,20)*+{X}="X",
(0,0)*+{A\times D}="AD",(25,0)*+{B\times D}="BD",(40,0)*+{Y}="Y"
\ar^{f\times\id} "AC";"BC"
\ar_{\id\times g} "AC";"AD"
\ar_{f\times\id} "AD";"BD"
\ar^{a} "AD";"X"
\ar|(.6){\id\times g} "BC";"BD"
\ar^(.6){b} "BC";"X"
\ar@{.>}_{d} "BD";"X"
\ar_(.6){c} "BD";"Y"
\ar^{h} "X";"Y"
\endxy\]
\label{triple lift}
\end{lemma}

\begin{proof} 
Cut down the variety of lifting problems with the following strategies.
\begin{itemize}
\item Using symmetry, derive the cases where $g$ is acyclic from those where $f$ is.
\item By definition, there is an equivalence between lifting acyclic cofibrations and having a filler operator for the family of horns. Use this to reduce the cases where $f$ is acyclic to the case where $f$ is the family of horns.
\item By lemma \ref{Reedy}, there is an equivalence between lifting cofibrations and having a filler operator for the family of cycles. Use this to reduce the cases where $g$ is an arbitrary cofibration to the case where $g$ is the family of cycles and do the same for $f$.
\end{itemize}
These reductions can be proved with the \emph{pullback power} construction and some diagram chasing.
\[\xy
(34,20)*+{X^C}="top",(0,10)*+{X^D}="left",(24,10)*+{\bullet}="middle",(44,10)*+{Y^C}="right",(34,0)*+{Y^D}="bottom"
\ar^{X^g} "left";"top" \ar@{.>}|(.6){h^g} "left";"middle" \ar_{h^D} "left";"bottom" \ar "middle";"bottom"
\ar "middle";"top" \ar^(.6){h^C} "top";"right" \ar_(.6){Y^g} "bottom";"right"
\endxy\]
The triple lifting problem is equivalent to the simple lifting problem of $f$ against $h^g$.
This leaves two tasks.
\begin{enumerate}
\item Prove that if $h$ has a filler operator for cycles, it has a triple filler operator for the cases where $f$ and $g$ are both the family of cycles.
\item Prove that if $h$ has a filler operator for horn, it can also handle all triple lifting problems involving the family of horns $f$ and the family of cycles $g$.
\end{enumerate}

In both cases there is a subobject of $B\times D$ that is the pushout of $f\times C$ and $A\times g$, because the underlying monomorphisms of $f$ and $g$ are decidable.
\[ \set{\tuplet{x,y}\of B\times D\middle| (x\of f(A)) \vee (y\of g(C)) }\]
Since pushouts preserve left lifting properties, a filler operator for the pushout product is sufficient.

For $m > 0$ and $n > 0$ none of the faces of $\simplex[m]\times \simplex[n]$ belong to either $\simplex[m]\times \cycle[n]$ or $\horn_k[m]\times\simplex[n]$ for any $k\of[m]$, because the faces are $m+n$-dimensional, and the faces of either $\simplex[m]$ or $\simplex[n]$ are not. The cases where $m=0$ or $n=0$ are trivial because $\simplex[0] \simeq 1$, $\cycle[0]\simeq 0$ and $\horn_0[0]$ does not exist. Since the pushout product is a countable sum of these inclusions, it is a cofibration. This settles the case where both $f$ and $g$ are the family cycle inclusions, because $h$ is an acyclic fibration.

The case where $f$ is the family of horn inclusions is more complicated. Fortunately, this case is worked out in the proof of lemma A.1 in \citep{DD&DIS11}, where \emph{inner anodyne} is what this paper calls acyclic, and \emph{box product} is pushout product. Note that all necessary properties of simplices are decidable. Since the pushout product of $f$ and $g$ is an acyclic cofibration.
\end{proof}
\hide{perhaps we should adopt the same terms as that paper.}


\hide{ 
The case for $f$ acyclic seems simpler, but perhaps pushout products are not as ubiquitous as I thought. I don't see how we can decide if a degenerate simplex belongs to the image of a cofibration. The problem is determining that $x\cdot \phi = x$ for some $\phi\of[\dim x]\to[\dim x]$, since $=$ is not always decidable. This in turn that pushout products may not be available for all cofibrations.

The trick involves regularity: the family of faces exists internally, and therefore so does its union.

Not good enough! We would need a union of faces inside. We can create a 'least pseudo-complement'. 
}



\end{document}

