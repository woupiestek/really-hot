\documentclass{tac}
\usepackage{amssymb, amsmath}
\usepackage[backend=bibtex,citestyle=authoryear-icomp]{biblatex}
\usepackage[all]{xy}
\usepackage{url}

\title{Constructive Simplicial Homotopy}
\author{Wouter Pieter Stekelenburg}
\copyrightyear{2015,2016,2017}
\address{Faculty of Mathematics, Informatics and Mechanics\\
University of Warsaw\\
Banacha 2\\
02-097 Warszawa\\
Poland}
\eaddress{w.p.stekelenburg@gmail.com}
\keywords{realizability, simplicial homotopy, Kan complexes}
\amsclass{03D80, 18G30, 18G55}

\newcommand\hide[1]{}
\newcommand\cat\mathcal
\newcommand\set[1]{\left\{#1\right\}}
\mathrmdef{id}
\mathrmdef{dom}
\mathrmdef{cod}
\newcommand\ri{^*}
\newcommand\N{\mathbb N}
\mathbfdef[nno]{N}
\newcommand\dual{^{\mathrm{op}}}
\mathbfdef[simCat]\Delta
\newcommand\s{^{\simCat\dual}}
\newcommand\bang{!}
\newcommand\of{:}
\newcommand\simplex\Delta
\newcommand\cycle{\partial\Delta}
\newcommand\horn\Lambda
\newcommand\f{_f}
\newcommand\tuplet[1]{\left\langle #1 \right\rangle}
\newcommand\true{\mathtt{true}}
\newcommand\false{\mathtt{false}}
\newcommand\bool{\mathtt{bool}}
\mathrmdef{nat}
\mathssbxdef{Ar}
\mathssbxdef{Ob}
\newcommand\pp{\mathbin\diamond}
\newcommand\norm[1]{\Vert #1 \Vert}
\newcommand\ka\kappa
\newcommand\la\lambda
\mathrmdef{face}
\mathrmdef{colim}
\newcommand\ex{_{\textrm{ex}}}
\newcommand\citep[1]{[\cite{#1}]}
\mathrmdef{dim}
\newcommand\base{\mathbf{U}}
\mathssbxdef[sub]{Sub}
\newcommand\ambient{\mathfrak A}
\mathrmdef{uni}
\newcommand\disc{_{\rm disc}}
\mathrmdef{filler}
\newcommand\traco\omega

\newcommand\product[2]{\Pi #1 \mapsto #2}
\newcommand\coproduct[2]{\Sigma #1 \mapsto #2}
\newcommand\function[2]{\lambda #1 \mapsto #2}

\newcommand\keyword[1]{\emph{#1}\label{#1}}


\begin{document}
We are working out exactly what a \emph{horn or cycle filler} operator is, just like we worked out what a simplicial set is, in hopes of making the proof clearer and more convincing.



The lifting property is more demanding because it has to take all cases into account.
I work in all kinds of slice categories to hide to gory details, and keep track of additional requirement:
For a particular $f\of X\to Y$ and the family of horns $h\of D\to C$.
\begin{align*}
P &= \set{ \tuplet{a,b}\of(D\to X)\times(C\to Y)| f\circ a = b \circ h }\\
\dim(a(\tuplet{\phi,k})) &=\max(\cod(\phi))\\
\dim(b(\tuplet{\phi,k})) &= \max(\cod(\phi))\\
a(\tuplet{\phi,k})\cdot\chi &= a(\tuplet{\phi\circ \chi,k}) \\
b(\tuplet{\phi,k})\cdot\chi &= b(\tuplet{\phi\circ \chi,k}) \\
\end{align*}
\begin{itemize}
\item The cycle case is simpler, because we can drop the $k$-index.
\item The filler satisfies the same restrictions on dimensionality and restrictions, plus $c\circ h = a$ and $f\circ c = b$.
\end{itemize}

We now get a simplified notion of what a filler operator for both families are, and can try to rewrite the proofs in order to build exactly such an operator.

Suppose we work out an official definition of horn and cycle fillers. What might happen?
The step from the filler to lifting cofibrations may be less obvious, requiring some updates of the proofs and definition.
Not the acyclic though, just the weakly invertible.

\newcommand\HLP{\mathrm{HLP}}
\begin{definition}
Let  Let $f\of X\to Y$ be an arbitrary morphism of $\ambient\s$. The object of horn lifting problems is 
\[\HLP(f) = \set{\tuplet{a,b}\of \base X^D\times \base Y^C \middle|\begin{array}{l}
  f\circ a = b\circ h,\\
  \dim(b\tuplet{\phi,k}) = \max(\dom\phi),\\
  b\tuplet{\phi,k}\cdot\chi = b\tuplet{\phi\circ \chi,k},\\
	a\tuplet{\phi,k}\cdot\chi = a\tuplet{\phi\circ \chi,k}
\end{array}}\]
Here $h\of D\to C$ is the inclusion of the following objects in $\ambient$.
\begin{align*}
C &= \coproduct{\phi\of\Ar(\simCat)}\cod(\phi) \\
D &= \set{\tuplet{\phi,k}\of C \middle| \exists i\of\cod(\phi)-\set k. \forall j\of\dom(\phi).\phi(j)\neq i}
\end{align*}
A \keyword{horn filler} is a morphism $c\of\HLP(f)\to \base X^C$ that satisfies the following equations. 
\begin{align*} c\tuplet{a,b}\circ h &= a & f\circ c\tuplet{a,b} &= b \end{align*}
A \emph{fibration} of $\ambient\s$ is a morphism with a horn filler.
\end{definition}

%Cycle fillers 
\newcommand\CLP{\mathrm{CLP}}
\begin{definition}
Let $f\of X\to Y$ be an arbitrary morphism of $\ambient\s$. The object of cycle lifting problems is 
\[\CLP(f) = \set{\tuplet{a,b}\of \base X^D\times \base Y^C \middle|\begin{array}{l}
  f\circ a = b\circ h,\\
  \dim(b(\phi)) = \max(\dom\phi),\\
  b(\phi)\cdot\chi = b(\phi\circ \chi),\\
  a(\phi)\cdot\chi = a(\phi\circ \chi)
\end{array}}\]
Here $h\of D\to C$ is the inclusion of the following objects in $\ambient$.
\begin{align*}
C &= \Ar(\simCat) \\
D &= \set{\phi\of C \middle| \exists i\of\cod(\phi).\forall j\of\dom(\phi).\phi(j)\neq i}
\end{align*}
A \keyword{cycle filler} is a morphism $c\of\CLP(f)\to \base X^C$ that satisfies the following equations. 
\begin{align*} c\tuplet{a,b}\circ h &= a & f\circ c\tuplet{a,b} &= b \end{align*}
A morphism  of $\ambient\s$ is \emph{contractible} if it has a cycle filler.
\end{definition}

Where do we have problems with generalizations?
\begin{itemize}
\item proposition on the factorization of $f$ as a contractible morphism following a cofibration.
\item the triple lifting property
\item if $f$ and $g\circ f$ are contractible, then so is $g$. I put a more direct prove here.
\item fibrancy of the universal modest fibration
\end{itemize}

\section{Triple lifting property}
We derive this from the equality of contractible and the acyclic fibrations. The core argument is that $\tuplet{f\to g}$ is a fibration if $f\of A\to B$ is a cofibration and $g\of X\to Y$ is a fibration, and contractible if $g$ is contractible or $f$ is acyclic. So how to define the horn and cycle fillers?

\[ \tuplet{f\to g} = \function{x\of X^B}\tuplet{g\circ x,x\circ f}\of X^B \to Y^B\times_{Y^A} X^A \]

Another approach is to take the pushout product of the families and construct the desired fillers there.

Take push out product of the family of horns with the family of cycles, then find a filler.
If $f$ is contractible, take the pushout product of the family of cycles with itself, and find a filler for that too.
Then use adjoints etc. to prove the lemma.
We may be able to pull it of without products, however. The triple lift for the families is a special case form which we derive all the special cases.

How far would we get with just the family of cycles along an (acyclic) cofibration?

The ultimate point is that we can fill the pushout product by systematically using fillers for simplices.
That algorithmic reduction of a triple lifting problem to a horn or cycle lifting problem can be done inside $\ambient$ to show that elementary triple lifting problems have a solution if horn or cycle lifting problems have. There rest is the reduction of the general triple lifting problem to the product of the families.

The reduction uses monotone monics $[m + n]\to[m]\times[n]$. 
We start with a collection of $[m+n-1]$-dimensional faces in the pushout product. Unless $m=0$ or $n=0$ we have no $[m+n]$ dimensional faces however. So these need to be glued in, one by one.
These can be lexicographically ordered again, after which we ensure that they get filled.
I.e. we build a sequence $A_i\to A_{i+1}$ of approximations and show that each requires a well defined application of the preexisting filler operator.
Clearly, the result is a proof in the same style as the descent proof, with one important difference: we don't rely on saturation, but an a filler operator to do the job.
\end{document}

