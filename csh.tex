% LaTeX 2e document, TAC style, 36 pp, Xy-pic ver ?, MikTeX version ?

\documentclass{tac}
\usepackage{subfiles}
\usepackage{amssymb, amsmath, stmaryrd}
\usepackage[backend=bibtex,citestyle=authoryear-icomp]{biblatex}
\usepackage[all]{xy}
\usepackage{url}

\title{Constructive Simplicial Homotopy}
\author{Wouter Pieter Stekelenburg}
\copyrightyear{2015,2016,2017,2018}
\address{Faculty of Mathematics, Informatics and Mechanics\\
University of Warsaw\\
Banacha 2\\
02-097 Warszawa\\
Poland}
\eaddress{w.p.stekelenburg@gmail.com}
\keywords{realizability, simplicial homotopy, fibrant objects}
\amsclass{03D80, 18G30, 18G55}

\newcommand\hide[1]{}
\newcommand\cat\mathcal
\newcommand\set[1]{\left\{#1\right\}}
\mathrmdef{id}
\mathrmdef{dom}
\mathrmdef{cod}
\newcommand\ri{^*}
\mathbfdef[nno]{N}
\newcommand\dual{^{\mathrm{op}}}
\mathbfdef[simCat]\Delta
\newcommand\s{^{\simCat\dual}}
\newcommand\bang{!}
\newcommand\of{\mathord :}
\newcommand\simplex\Delta
\newcommand\cycle{\partial\Delta}
\newcommand\horn\Lambda
\newcommand\f{_f}
\newcommand\tuplet[1]{\left\langle #1 \right\rangle}
\newcommand\true{\mathtt{true}}
\newcommand\false{\mathtt{false}}
\newcommand\bool{\mathtt{bool}}
\mathrmdef{nat}
\mathssbxdef{Ar}
\mathssbxdef{Ob}
\newcommand\pp{\mathbin\diamond}
\newcommand\norm[1]{\Vert #1 \Vert}
\newcommand\ka\kappa
\newcommand\la\lambda
\mathrmdef{face}
\mathrmdef{colim}
\newcommand\ex{_{\textrm{ex}}}
\newcommand\citep[1]{[\cite{#1}]}
\mathrmdef{dim}
\newcommand\base{\mathbf{U}}
\mathssbxdef[sub]{Sub}
\newcommand\ambient{\mathfrak A}
\mathrmdef{uni}
\newcommand\disc{_{\rm disc}}
\mathrmdef{filler}
\newcommand\traco\omega

\newcommand\product[1]{\Pi #1 .}
\newcommand\coproduct[1]{\Sigma #1 .}
\newcommand\function[1]{\lambda #1 .}

\newcommand\keyword[1]{\emph{#1}\label{#1}}
\newcommand\icat\mathsf


\addbibresource{realizability}

\begin{document}

\begin{abstract} This paper develops the foundations of a constructive 
simplicial homotopy that takes the logical limits of the internal languages of 
elementary toposes in account. The aim is to develop new models of homotopy 
type theory in particular models that are based on simplicial objects of 
the effective topos and related categories.
\end{abstract}

\hide{
Three papers:
-simplicial homotopy
-complete categories [how they are preserved]
-the realizability model of HOTT [how to get a fibrant object out of a category]

Idea: reverse the order. definitions--theorem--lemmas. That way the purpose of 
the lemmas is set up from the start.
}

\maketitle

\section*{Introduction}
This paper grew out of an attempt to build a \emph{recursive realizability} 
model for \emph{homotopy type theory} by doing simplicial homotopy (see 
\citep{Hovey99,GJSHT}) in a \emph{realizability topos} (see \citep{MR2479466}) 
and following the example of \citep{KLV12}.
Realizability toposes make simplicial homotopy harder because they may lack 
infinite limits and colimits and their internal logic does not always satisfy 
the principle of the excluded middle or the axiom of choice. We face these 
challenges with the following tactics.
\begin{enumerate}
\item Replace the external notions of smallness used in the small object 
argument with an internal one. A dense full internal subcategory provides
a kind of infinite limit that the internal language can access, while giving a
similar limitation of size.
\item Limit the class of cofibrations. In classical simplicial homotopy theory 
every mono\-morphism is a cofibration. To keep classical arguments valid we 
demand that certain properties of the monomorphisms are decidable.
\item Strengthen the lifting property. Fibrations come equipped with a 
\emph{filler} that gives solutions for a family of basic lifting problems.
\end{enumerate}

\subsection*{Ambient category}
To avoid distracting peculiarities of realizability toposes, this paper works 
with a generic \emph{elementary topos} (see definition \ref{topos}). 
The exactness of toposes is not vital for the theory developed here, but 
since \emph{ex/lex completion} are homotopy categories for 1-dimensional 
simplices, I assume that there is no loss of generality in only considering 
exact examples. %cite someone?
The minimal requirements include local cartesian closure and a full 
dense internal subcategory. I believe that exact completions of 
suitable categories are all toposes, based on Menni's work. %cite Menni
The advantage is being able to work with the first order internal language of 
toposes.

\subsection*{Intended model} The \emph{category of assemblies}, which is the 
category of $\neg\neg$-separated objects of \emph{the effective topos} up to 
equivalence, has \emph{strongly} complete internal categories that are 
not posets, in particular the \emph{category of modest sets} \citep{MR1097022,
MR2479466,MR1023803}. \emph{Strongly complete} means that the externalization 
of the internal category is a complete fibred category. The category of 
assemblies is not exact, but the ex/lex completion 
preserves strongly complete internal categories. This exact completion is not 
the effective topos, but it is a kind of realizability topos and it is 
the intended ambient category in this paper \citep{MR1600009}. 

The topos with a complete internal category that is not a poset is interesting, 
because complete internal categories in Grothendieck toposes are necessarily 
posets. This fact is traditionally attributed to Peter Freyd.

In the ex/lex completion \emph{modest fibrations} are fibrations whose 
underlying morphisms are families of modest sets or quotients of such families 
by modest families of equivalence relations--see \citep{MR1097022,MR1023803,
MR2479466}. There is a \emph{universal modest fibration} (see definition 
\ref{universal modest fibration}) which is a \emph{univalent fibration} and 
hence a potential model of homotopy type theory.

\subsection*{Conclusion} There are definitions of fibration, cofibration and 
their acyclic counterparts (definition \ref{model structure}) in 
elementary toposes that make the category of fibrant objects a \emph{model 
category} (theorem \ref{model category}). For certain classes of fibrations 
\emph{universal fibrations} (definition \ref{universal modest fibration}) are 
automatically fibrant (theorem \ref{fibrant universe}).

\subsection*{Acknowledgments} 
I am grateful to the Warsaw Center of Mathematics and Computer Science for the 
opportunity to do the research leading to this paper. I am also grateful for 
discussions with Marek Zawadowski and the seminars on simplicial homotopy 
theory he organized during my stay at Warsaw University. Richard Garner, Peter 
LeFanu Lumsdaine and Thomas Streicher made helpful comments on early drafts 
of this paper.

\subfile{cshintro}

\subfile{cshmodel}

\subfile{cshfactor1}

\subfile{cshfactor2}

\subfile{cshwe}

\subfile{cshdescent}

\printbibliography

\end{document}

