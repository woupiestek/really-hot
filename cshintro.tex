\documentclass[csh.tex]{subfiles}
\begin{document}
\section{Internal simplicial objects}
Internal simplicial homotopy has a notion of simplicial object of $\ambient$ which does rely on external $\ambient$-valued presheaves, but on internal structures.

\begin{definition} A \keyword{simplicial object} $X$ of $\ambient$ is a triplet $\tuplet{\base X, \dim, \cdot}$ where $\base X$ is an object of $\ambient$, $\dim$ is a morphism $\base X\to \nno$. The operator $\cdot$ is has the following properties. Let $\Ar(\simCat)$ be the object of non decreasing functions between decidable, finite and inhabited initial segments of the natural numbers. The domain $\dom(\cdot)$ of $\cdot$ is the following object. 
\[ \set{\tuplet{x,\xi}\of \base X\times \Ar(\simCat)\middle| \dim(x) = \max(\cod(\xi))}\]
The codomain $\cod(\cdot)$ of $\cdot$ is $\base X$. The operator $\cdot$ satisfies the following equations.
\begin{align*}
\dim(x\cdot\xi) &= \max(\dom(\xi))\\
x\cdot \id &= x \\
x\cdot(\alpha\circ\beta)&=(x\cdot\alpha)\cdot\beta 
\end{align*}

A \emph{morphism of simplicial objects} $X\to Y$ is a morphism $f\of\base X\to\base Y$ which commutes with $\dim$ and $\cdot$. Together with simplicial objects, they form the category of simplicial objects and morphisms $\ambient\s$.

The category $\ambient\s$ is enriched over $\ambient$. The object of morphisms $X\to Y$ in $\ambient$ itself is $\nat(X,Y)$. It represents families of morphisms $X\to Y$, i.e. for each object $I$ of $\ambient$ there is a natural bijection between morphisms $I\to \nat(X,Y)$ and morphisms $I\times X\to Y$ which commute with $\dim$ and $\cdot$ and their $\id_I$-fold multiples.
\end{definition}

\begin{example} A simple kind of simplicial object is the \keyword{discrete} simplicial object. There is one for each object $I$ of $\ambient$: 
\[ I\disc = \tuplet{\nno\times I,\function{\tuplet{n,i}}n,\function{\tuplet{\tuplet{n,i},\phi}}i} \]
Note the $\function{x}{f(x)}$ notation for morphisms of $\ambient$ the paper uses.
\end{example}

This definition is consistent with the definition of initial presheaves in \citep{MR1300636}. The definition is valid because $\ambient$ is a $\Pi$-pretopos with a natural number object $\nno$. Decidable, finite and inhabited initial segments of the natural numbers have a classifier.
\[  \function{\tuplet{i,j}}j\of\set{\tuplet{i,j}\of\nno\times\nno\middle|i\leq j}\to \nno \]
The relation $\leq$ is decidable and comprehension on decidable predicates to define subobjects in $\ambient$, because there are pullbacks and classifiers of decidable subobjects $1\to \bool = 1+1$. Here $1$ is a terminal object and $+$ a binary coproduct.

Local Cartesian closure means that there is an object $A$ of morphisms between the initial segments. Local Cartesian closure also implies that there is a subobject of non decreasing morphisms $\Ar(\simCat)$.
\[ \product{f\of A,i,j\of\nno,x\of\set{y\of 1|i\leq j}}{\set{y\of 1|f(i)\leq f(j)}}\]
However, because $\ambient$ is also exact and extensive, it is a Heyting category, so the same subobject has a more familiar definition.
\[ \Ar(\simCat)=\set{f\of A|\forall i,j\of\nno.i\leq j \to f(i)\leq f(j)} \]

\begin{definition} The object $\Ar(\simCat)$ is the object of arrows of the internal \keyword{category of simplices} $\simCat$, whose object of objects $\Ob(\simCat)$ is $\nno$. Composition is defined as ordinary function composition.
\end{definition}

\begin{remark} We assume that all of the structure on $\ambient$ comes from functors. So several functors between categories related to $\ambient$ have right adjoints that give $\ambient$ its limits and exponentials and some functors have left adjoints that give $\ambient$ colimits. Definitions in terms of universal properties only define objects up to isomorphism. Our assumption ensures that there are functors that hit all the necessary isomorphism classes, when we only specify a functor up to isomorphism.
\end{remark}

\begin{definition} 
That $\ambient$ is a $\Pi$-\keyword{pretopos} means all of the following.

\begin{enumerate}
\item The category $\ambient$ is \emph{locally Cartesian closed}. This means that for each object $X$ of $\ambient$ each slice $\ambient/X$ is Cartesian closed.
\item The category $\ambient$ is \emph{extensive}, which means that is has finite coproducts and that $\ambient/(X+Y)$ is equivalent to $(\ambient/X)\times(\ambient/Y)$ for each pair of objects $X$ and $Y$ of $\ambient$.
\item The category $\ambient$ is \emph{exact} in the sense of Barr. This means that all weak groupoids (see below) have coequalizers and that those coequalizers are stable under pullback.
\end{enumerate}

A \keyword{weak groupoid} consists of a pair of objects $X_0$, $X_1$ and morphisms $r\of X_0\to X_1$ and $s,t\of X_1\to X_0$ such that $s\circ r= t\circ r = \id_{X_0}$ such that for each $x,y,z\of X_0$ if two of $\tuplet{x,y}$, $\tuplet{x,z}$ and $\tuplet{y,z}$ are in the image of $\tuplet{s,t}\of X_1\to X_0\times X_0$, the third is too. A \emph{coequalizer} for a weak groupoid is a coequalizer of $s$ and $t$.
\end{definition}

\begin{example} Every topos is a $\Pi$-pretopos so $\ambient$ can be any topos with a natural number object like the topos of sets. \end{example}

The category $\ambient$ is has a \emph{natural number object}. To keep track of the distinction between internal and external natural numbers, let the \emph{set} of natural numbers be $\N$ and use $i\in \N$ to indicate membership. The natural number object of $\ambient$ is $\nno$ and $i\of\nno$ means that $i$ is a morphism with codomain $\nno$.

This combination of properties implies the following.
\begin{enumerate}
\item The category $\ambient$ has all colimits and reindexing functors preserve them. 
\item The category $\ambient$ is a \emph{Heyting category}. 

Technically this means that for each object $X$ the poset of subobjects $\sub(X)$ is a Heyting algebra and that for each $f\of X\to Y$ the preimage map $f\ri\of\sub(Y)\to\sub(X)$ is a morphism of Heyting algebras. Moreover, for each morphism $f$, $f\ri$ has both adjoints--right adjoint $\forall_f$ and left adjoint $\exists_f$--and those adjoints satisfy the \emph{Beck-Chevalley condition} which says that quantification and substitution commute over pullback squares. If $f\circ g = h\circ k$ is a pullback square, then $f\ri\circ\exists_h=\exists_g\circ k\ri$.

Practically this means that $\ambient$ is a model for a constructive first order logic, hence that predicates define subobjects and that constructive theorems are valid in the internal language.
\end{enumerate}

\end{document}