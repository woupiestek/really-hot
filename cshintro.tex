\documentclass[csh.tex]{subfiles}
\begin{document}


\section{Introduction}\hide{dump assemblies--write about simplicial homotopy}
This paper grew out of an attempt to build a \emph{recursive realizability} model for \emph{homotopy type theory}. The intention is to interpret types as the homotopy types of the simplicial objects in the \emph{category of assemblies}. What makes this challenging, is that this category lacks some of the structure of the category of sets that \emph{classical homotopy theory} is build on. In the internal logic the principle of the excluded middle and the axiom of choice are false. The category of assemblies is not exact and lacks infinite limits and colimits. While there are \emph{generic monomorphisms}, they are not subobject classifiers. I offer the following solutions.
\begin{enumerate}
\item Take the exact completion of the category of assemblies. This is not the effective topos but another realizability topos.
\item Limit the class of cofibrations. In classical simplicial homotopy theory every mono\-morphism is a cofibration. We demand that certain properties of the monomorphisms are decidable.
\item Strengthen the lifting property. Fibrations come equipped with a \emph{filler operator} that specifies a solution for each of the basic lifting problems.
\item Build the homotopy category out of fibrant objects only. This way we don't need the infinite colimits that make the small object argument of classical simplicial homotopy work.
\end{enumerate}
To avoid the distracting peculiarities of the category of assemblies and its exact completion this paper works with a generic \emph{$\Pi$-pretopos} with a \emph{natural number object} instead of the category of assemblies itself.

The paper works out suitable definition of fibrations, cofibrations and their acyclic counterparts (definitions \ref{Kan}, \ref{cofibration}, \ref{acyclic cofibration}) and proves that they form a model structure on the category of Kan complexes (theorem \ref{model category}). The last section confirms that certain universal fibrations actually live in the category of Kan complexes (theorem \ref{fibrant universe}). 

\subsection{Related literature}
Simplicial homotopy is the homotopy of simplicial sets. It is equivalent to the homotopy of CW complexes which are a category of topological spaces \citep{Hovey99,GJSHT}. The intended model is the \emph{ex/lex completion} \citep{MR1600009} of category of assemblies defined in \citep{MR1097022,MR1023803,MR2479466}. This completion is not the famous \emph{effective topos}, but it is another realizability topos. 

\section{Preliminaries}% new idea: do simplicial homotopy, work out notation along the way.
This section surveys the categorical logic required to understand the rest of this paper.

\subsection{Set up}
The ambient category $\ambient$ where everything happens is a \emph{$\Pi$-pretopos} which means the following.

\begin{enumerate}
\item The category $\ambient$ is \emph{locally Cartesian closed}. This means that for each object $X$ of $\ambient$ each slice $\ambient/X$ is Cartesian closed.
\item The category $\ambient$ is \emph{extensive}, which means that is has finite coproducts and that $\ambient/(X+Y)$ is equivalent to $(\ambient/X)\times(\ambient/Y)$ for each pair of objects $X$ and $Y$ of $\ambient$.
\item The category $\ambient$ is \emph{exact} in the sense of Barr. This means that all weak groupoids (see definition \ref{weak groupoid}) have coequalizers, and that those coequalizers are stable under pullback.
\item The category $\ambient$ is has a \emph{natural number object}.
\end{enumerate}

\begin{definition} A \keyword{weak groupoid} consists of a pair of objects $X_0$, $X_1$ and morphisms $r\of X_0\to X_1$ and $s,t\of X_1\to X_0$ such that $s\circ r= t\circ r = \id_{X_0}$ such that for each $x,y,z\of X_0$ if two of $\tuplet{x,y}$, $\tuplet{x,z}$ and $\tuplet{y,z}$ are in the image of $\tuplet{s,t}\of X_1\to X_0\times X_0$, the third is too. A \emph{coequalizer} for a weak groupoid is a coequalizer of $s$ and $t$.
\end{definition}

We keep track of the distinction between internal and external natural numbers. The \emph{set} of natural numbers is $\N$, and since it is a set, $i\in \N$ indicates membership. The natural number object of $\ambient$ is $\nno$ and $i\of\nno$ means that $i$ is a morphism with codomain $\nno$.

This combination of properties implies that 
\begin{enumerate}
\item The category $\ambient$ has all colimits and that reindexing functors preserve them. 
\item The category $\ambient$ is a \emph{Heyting category}. This means that for each object $X$ the poset of subobjects $\sub(X)$ is a Heyting algebra and that for each $f\of X\to Y$ the preimage map $f\ri\of\sub(Y)\to\sub(X)$ is a morphism of Heyting algebras.  
\end{enumerate}

Using only this structure there is a category of internal simplicial objects $\ambient\s$ of $\ambient$ were most of the action in this paper happens. we use the following notation to help do this.

\begin{definition} Of internal categories the codomain, domain and identity operators are called $\cod$, $\dom$ and $\id$. When not talking about numbers, $0$, $1$, $+$ and $\times$ refer to initial and terminal objects and binary coproduct and product operators. The unique map to the terminal object is $\bang$ and projection maps from products to factors are $\pi_i$ for $i\in \N$.

Let $\cat C$ be an internal category. Its object of objects is $\Ob(\cat C)$ and its object of arrows is $\Ar(\cat C)$, so in this case $\cod$ and $\dom$ are morphisms $\Ar(\cat C)\to\Ob(\cat C)$ and $\id\of\Ob(\cat C)\to\Ar(\cat C)$. Let $\cat D$ be another internal category.

Every object $I$ of the $\ambient$ has an internal discrete category $I\disc$. For this category $\Ob(I\disc)=\Ar(I\disc)=I$ and $\cod=\dom=\id=\id_I$.

Let $[n]=\set{i\of\nno|i\leq n}$. The fact that $\ambient$ is a Heyting category justifies the use of set builder notation here. Let $\simCat$ be the internal category of non decreasing morphisms between inhabited initial segments of $\nno$--which exists by proposition \ref{simplex category}. 

 An \keyword{internal simplicial object} $X=(\base X,\dim,\cdot)$ is an object $\base X$ together with a morphism $\dim\of \base X\to\nno$ and an operator $\cdot$, whose domain is the following object.
\[ \set{\tuplet{x,\xi}\of \base X\times\Ar(\simCat)\middle| [\dim (x)]=\dom(\xi)} \]
The codomain of $\cdot$ is $\base X$. The operator $\cdot$ satisfies $(x\cdot\alpha)\cdot\beta=x\cdot(\alpha\circ\beta)$ and $x\cdot\id_{[\dim x]}=x$.  

A morphism of internal simplicial objects $X\to Y$ is a morphism $\base X\to \base Y$ that commutes with $\dim$ and $\cdot$. I.e. $f(x\cdot \xi)=f(x)\cdot\xi$ and $\dim(f(x))=\dim(x)$.

The category of internal simplicial objects and morphisms of $\ambient$ is $\ambient\s$.
\end{definition}\hide{ Is it feasible to rewrite the whole paper with these?}

In a simplicial object $X$, $\base X$ is the object of simplices, $\dim$ is the dimension of each simplex. The operator $\cdot$ records how simplices of different dimensions fit together. 

\begin{proposition} Every $\Pi$-pretopos with natural number object has an internal category of simplices. \label{simplex category} \end{proposition}

\begin{proof} A natural number object allows the definition of predicates by recursion, so the order relation of $\nno$ is a recursive function $\mathord\leq\of\nno\times \nno \to \bool$. This is all we need to define the internal category of simplices.
\begin{align*}
[n] &= \set{x\of\nno\middle| x\leq n}\\
\simCat([m],[n]) &= \set{f\of[m]\to[n]\middle|\forall x,y\of [m].x\leq y\to f(x)\leq f(y)}
\end{align*}
Universal quantification over $[m]$ reduces to a finite conjunction because $[m]$ is a finite object, which is why it is allowed above. 
\end{proof}


Kan fibrations are morphisms of simplicial sets that have the lifting property relative to the family of horn inclusions. This subsection focuses on defining lifting properties.

\begin{definition} For each pair of internal simplicial objects $X$, $Y$ let $\nat(X,Y)$ be the object of morphisms between them.

A morphism $f\of X\to Y$ of simplicial object has the \emph{right lifting property} with respect to a morphism $g\of I\to J$--and $g$ has the \emph{left} lifting property with respect to $f$--if the morphism $\tuplet{f_!,g\ri} = \tuplet{\nat(\id_J,f),\nat(g,\id_X))}$ 
which is the factorization of the span $\nat(\id_J,f)$, $\nat(g,\id_X)$ through the pullback cone of $\nat(\id_I,f)$ and $\nat(g,\id_Y)$
is a \emph{split} epimorphism.
\[\xy
(34,20)*+{\nat(I,X)}="top",(0,10)*+{\nat(J,X)}="left",(24,10)*+{\bullet}="middle",(44,10)*+{\nat(I,Y)}="right",(34,0)*+{\nat(J,Y)}="bottom"
\ar^{\nat(g,\id_X)} "left";"top" \ar@{.>}|(.6){\tuplet{f_!,g\ri}} "left";"middle" \ar_{\nat(\id_J,f)} "left";"bottom" \ar "middle";"bottom"
\ar "middle";"top" \ar^(.6){\nat(\id_I,f)} "top";"right" \ar_(.6){\nat(g,\id_Y)} "bottom";"right"
\endxy\]
A section of $\tuplet{f_\bang,g\ri}$ is a \emph{filler operator}.\label{lifting}
\end{definition}

A morphism can have a lifting property relative to a family or class of morphisms.

\hide{
Explain $I\ri\of\ambient\s\to (\ambient\s)^{I\disc}$ somewhere. It should be an internal simplicial object of the slice category. i.e. $(\ambient/I)\s$. 
}
\begin{definition}[Injective] An $I$-indexed-family of morphisms in $\ambient\s$ a morphism $a\of D\to E$ in $(\ambient\s)^{I\disc}$. Let $I\ri\of\ambient\s\to (\ambient\s)^{I\disc}$ be the diagonal functor. A morphism $f\of X\to Y$ is \emph{$a$-injective} if $I\ri f$ has the right lifting property with respect to $a$.
\end{definition}


\end{document}