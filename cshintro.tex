\documentclass[csh.tex]{subfiles}
\begin{document}
\section{Internal simplicial objects}
The \emph{ambient category} that provides the object and morphisms everything 
in this paper is constructed of, is an elementary topos $\ambient$ with natural 
number object $\nno$. 

\begin{definition} 
  That $\ambient$ is an elementary \keyword{topos} means all of the following.
  
  \begin{enumerate}
  \item The category $\ambient$ is \emph{locally cartesian closed}. This means 
  that for each object $X$ of $\ambient$ each slice $\ambient/X$ is cartesian 
  closed.
  \item The category $\ambient$ is \emph{extensive}, which means that is has 
  finite coproducts and that $\ambient/(X+Y)$ is equivalent to 
  $(\ambient/X)\times(\ambient/Y)$ for each pair of objects $X$ and $Y$ of 
  $\ambient$.
  \item The category $\ambient$ is \emph{exact} in the sense of Barr. This 
  means that all pseudo-equivalence relations (see below) have coequalizers and 
  that those coequalizers are stable under pullback.
  \item The category has a \emph{subobject classifier}, which is a monomorphism
  $t\of 1\to \Omega$ such that every monomorphism $m\of X\to Y$ is a pullback of
  $t$ along a unique morphism $\xi(m)\of Y\to \Omega$.
  \end{enumerate}
  
  A \keyword{pseudo-equivalence relation} consists of a pair of objects $X_0$, 
  $X_1$ and morphisms $r\of X_0\to X_1$ and $s,t\of X_1\to X_0$ such that 
  $s\circ r= t\circ r = \id_{X_0}$ and such that for each $x,y,z\of X_0$ if two 
  of $\tuplet{x,y}$, $\tuplet{x,z}$ and $\tuplet{y,z}$ are in the image of 
  $\function{x\of X_1}{\tuplet{s(x),t(x)}}\of X_1\to X_0\times X_0$, the third 
  is too. A \emph{coequalizer} for a pseudo-equivalence relation is a 
  coequalizer of $s$ and $t$.
\end{definition}

\begin{remark} A pseudo-equivalence relation is essentially a 1-dimensional 
  simplicial object. For that reason, the simplicial homotopy category doesn't
  change if some pseudo-equivalence relation has no coequalizer. Together with
  the subobject claissfier, exactness should play a minor role in the rest of
  this paper.
\end{remark}

Inside $\ambient$ the most relevant structure are \emph{internal diagrams} and
their duals \emph{internal presheaves}. These structures replace functors 
$\cat C\to \ambient$ and $\cat C\dual\to \ambient$ where $\cat C$ is a small 
category in classical treatments.

\begin{definition}
  For any category $\cat E$ with finite limits, an \keyword{internal category} 
  is a tuple
  $\icat C = \tuplet{\Ob(\icat C),\Ar(\icat C),\dom_{\icat C},\cod_{\icat C},\id_{\icat C},\circ_{\icat C}}$
  where $\Ob(\icat C)$ and $\Ar(\icat C)$ are objects of $\cat E$ and given that
  $C_2 = \set{\tuplet{f\of \Ar(\icat C),g\of \Ar(\icat C)}| \dom(f) = \cod(g) }$:
  \begin{align*}
  \dom_{\icat C} \of \Ar(\icat C)&\to \Ob(\icat C) &
  \cod_{\icat C} \of \Ar(\icat C) &\to \Ob(\icat C)\\
  \id_{\icat C} \of \Ob(\icat C)&\to \Ar(\icat C)&
  \circ_{\icat C} \of C_2&\to \Ar(\icat C)\\
  \dom_{\icat C}\circ\id_{\icat C} &= \id_{\Ob(\icat C)}&
  \cod_{\icat C}\circ\id_{\icat C} &= \id_{\Ob(\icat C)}\\
  \dom_{\icat C}(f\circ_{\icat C} g) &= \dom_{\icat C}(g)&
  \cod_{\icat C}(f\circ_{\icat C} g) &= \cod_{\icat C}(f)\\
  \id_{\icat C}(\cod_{\icat C}(g))\circ_{\icat C} g &= g &
  f\circ_{\icat C} \id_{\icat C}(\dom_{\icat C}(g)) &= f
  \end{align*}

  For $f\of X \to \Ob(\icat C)$, a \keyword{left action} of $\icat C$ is an 
  operator $\cdot\of Y \to X$ where \[Y = \set{\tuplet{g\of\Ar(\icat C),x\of X}|
  \dom(g) = f(x)}\] such that:
  \begin{align*}
    \id_{\cat C}(f(x))\cdot x &= x & 
    g \cdot (h\cdot x) &= (g\circ h)\circ x
  \end{align*}
  Dually, a \keyword{right action} of $\icat C$ on $f$ is an 
  operator $\cdot\of Z \to X$ where \[Z = \set{\tuplet{x\of X,g\of\Ar(\icat C)}|
  f(x) = \cod(g)}\] such that:
  \begin{align*}
    x\cdot\id_{\cat C}(f(x)) &= x & 
    (x\cdot g)\cdot h &= x\cdot(g\circ h)
  \end{align*}
  
  An \keyword{internal diagram} over $\icat C$ is the combination of an 
  $f\of X\to \Ob(\icat C)$ with a left action, while an \emph{internal presheaf}
  over $\icat C$ is the combination of an $f\of X\to \Ob(\icat C)$ with a right 
  action. Finally, a \emph{morphism} of diagrams or presheaves is a morphism that 
  commutes with the actions on the appropriate sides.
\end{definition}

The simplicial objects of thsi paper in are internal presheaves over the 
internal simplex category that can be fashioned from the natural number object.

\begin{definition} The internal category $\simCat$ satisfies 
  $\Ob(\simCat)=\nno$, $\Ar(\simCat)$ consist of non decreasing maps between
inhabited initial segments of $\nno$, domains, codomains, identities and 
compositions all canonical.

The \emph{category of internal presheaves} over $\simCat$ is denoted by
 $\ambient\s$.
\end{definition}

\begin{example} A simple kind of simplicial object is the \keyword{discrete} 
  simplicial object. There is one for each object $I$ of $\ambient$: 
\[ I\disc = \tuplet{\nno\times I,\function{\tuplet{n,i}}n,\function{\tuplet{\tuplet{n,i},\phi}}i} \]
Note the $\function{x}{f(x)}$ notation for morphisms of $\ambient$ the paper 
uses.
\end{example}

\begin{remark} Assume throughout this paper that all of the structure on 
  $\ambient$ comes from functors. Limits and exponentials come from right
   adjoints to functors between categories related to $\ambient$ and colimits 
   come from left adjoints. This assumption ensures that many functors only 
   specified up to isomorphism can be constructed.
\end{remark}

\hide{We need to talk about density comonads here, because it is a condition on
$\ambient$.}

Besides simplicial object we will be interested in a internal diagram
with specific properties assumed to be part of $\ambient$.

\begin{definition}
  %fully faithful
  Let $\cdot$ be a left action on $f\of X\to \Ob(\icat C)$ for some internal
  category $\icat C$. The left action $\cdot$ induces a map that shows how a
  diagram is a kind of functor:
  \[ \function g g\cdot\of\Ar(\icat C)\to\set{\tuplet{c,d,e}| c,d\of\Ob(\icat C), e\of X_c\to X_d} \]
  A diagram is therefore \emph{full} if this map is an epimorphism, 
  \emph{faithful} if this map is a monomorphism and \emph{fully faithful} if it
  is an isomorphism.

  %dense
  The \keyword{codensity monad} $M$ for a diagram $d\of D\to\Ob(\icat C)$ is the 
  left Kan extension of a diagram along itself. 
  \[M(X) = \colim_{D_c\to X} D_c\]
  It gives a best approximation of an object of $\ambient$ in terms of the 
  fibres of the diagram. If $M$ is isomorphic to the identity functor, the
  diagram $d$ is dense. %lot of handwaving, but how could this be misunderstood?

  %filtered
  A \emph{filtered category} is a category is a category in which every 
  finite diagram has a cocone. A \emph{filtered diagram} is a diagram over a
  filtered category. Both notions have internal counterparts which I won't spell
  out here.
\end{definition}

\begin{example} In any Grothendieck topos, there is a a fully faithful and dense
  diagram that indexes the representable functors. Some conditions on the 
  underlying site that are always satisfied by the ones that come form 
  topological spaces ensure that this diagram is filtered.
\end{example}

\begin{remark} %make this something to point back to.
An important assumption we make about $\ambient$ is that it has a 
dense filtered fully faithful internal diagram that contains all the objects 
of non decreasing morphisms between inhabited initial sections of the natural 
number objects. The diagram provides an initial alternative to smallness and 
this extra condition will eventually ensure that the generic cofibrations have 
`small' codomains.
\end{remark}

\begin{remark} The category of modest objects, or its ex/lex completion is not
  assumed to take on the role of dense fully faithful filtered diagram in the
  intended model, even though that would work. The modest sets are
  interesting because of their completeness, which is an unrelated rarer
  quality.
  As the example shows, the diagram makes an elementary topos more like a
  Grothendieck topos, but realizability toposes have other ways of meeting the
  same condition.
\end{remark}

\end{document}