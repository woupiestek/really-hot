\documentclass[csh.tex]{subfiles}
\begin{document}
\section{Cofibrations}
This section show that the cofibrations from definition \ref{cofibration} are part of an enriched factorization system (see definition \ref{enriched factorization system}) on all of $\ambient\s$ instead of just on the subcategory of complexes.

\begin{lemma} Cofibrations and acyclic fibrations form an enriched factorization system on $\ambient\s$. \label{factorization system 1} \end{lemma}

\begin{proof} Proposition \ref{factor1} shows that every morphism factors as a \emph{contractible morphism} (see definition \ref{contractible}) following a cofibration. Acyclic fibrations are examples of contractible morphisms (see lemma \ref{acyclic means contractible}). Lemma \ref{Reedy} shows that contractible morphisms are acyclic fibrations. Hence contractible morphisms and acyclic fibrations are the same class of morphism and this class form the right class of a factorization system with cofibrations on the left (see definition \ref{enriched factorization system}).
\end{proof}

\begin{definition} Let $\nno\disc\ri$ be the constant family functor $\ambient\s\to\ambient\s/\nno\disc$. 
A morphism $f\of X\to Y$ is \keyword{contractible} if $\nno\ri(f)$ has the $\ambient/\nno$-enriched right lifting property with respect to the \emph{family of cycle inclusions}, which is defined as follows.

The simplicial sets $\simplex_\nno$ and $\cycle_\nno$ satisfy:
\begin{align*}
\base(\simplex_\nno) &= \Ar(\simCat) \\
\base(\cycle_\nno) &= \set{\phi\of\Ar(\simCat)\middle|\exists i\of\nno.i\leq \cod(\phi),\forall j\of \dom(\phi).\phi(j)\neq i} \\
\dim(\phi) &= \dom(\phi) \\
\phi\cdot \xi &=\phi\circ\xi
\end{align*}
By these definitions $\cycle_\nno$ is a subobject of $\simplex_\nno$. The formula $\function{\phi}\tuplet{\dom(\phi),\cod(\phi)}$ defined morphisms of simplices $\cycle_\nno\to \nno\disc$ and $\simplex_\nno\to \nno\disc$ that commute with the inclusion, which is therefore a monomorphism of $\ambient\s/\nno\disc$. This inclusion is the family of cycle inclusions.
\end{definition}

\begin{lemma} Every acyclic fibration is a contractible morphism. \label{acyclic means contractible} \end{lemma}

\begin{proof}
Let $f\of X\to Y$ be an arbitrary acyclic fibration, let $k\of\cycle_\nno\to\simplex_\nno$ be the family of cycle inclusions and 
let $\hom_\nno\of(\ambient\s/\nno\disc\dual\times \ambient\s/\nno\disc)\to \ambient/\nno$ be the enrichment of $\ambient\s/\nno\disc$ over $\ambient\nno$.
The category $\ambient/\nno$ has an object that represents all lifting problems involving cycle inclusions and $f$.
\[ p\of P\to \nno = \set{\tuplet{a,b}\of \hom_\nno(\cycle_\nno,\nno\ri(X))\times \hom_\nno(\simplex_\nno,\nno\ri(Y)) \middle|f\circ a = b\circ k} \]

The simplicial sets $\simplex_P$ and $\cycle_P$ satisfy:
\begin{align*}
\base(\simplex_P) &= \set{ \tuplet{\phi,x}\of\Ar(\simCat)\times P | \cod(\phi) = p(x) } \\
\base(\cycle_P) &= \set{\tuplet{\phi,x}\of\base(\simplex_P)\middle|\exists i\of\nno.i\leq \cod(\phi),\forall j\of \dom(\phi).\phi(j)\neq i} \\
\dim\tuplet{\phi,x} &= \dom(\phi) \\
\tuplet{\phi,x}\cdot \xi &=\phi\circ\xi
\end{align*}

The inclusion $k_P\of\cycle_P\to\simplex_P$ is a cofibration for the following reasons. The simplex $\tuplet{\phi,x}\of\simplex_P$ is a face if $\phi$ is monic, outside of $\cycle_P$ if $\phi$ is epic, so $\phi = \id_[n]$ for some $n\of \nno$. This is a decidable property as required. 

For each $n\of\nno$ the members of the fibre $P[n]$ of $p$ over $n$ are pairs $\tuplet{a,b}$ where $a\of\cycle[n]\to X$ and $b\of \simplex[n]\to Y$ satisfy $f\circ a = b \circ k_n$. Here $\cycle[n]$ and $\simplex[n]$ are fibres over $n$ of the families. Define $a_P\of \cycle_P\to X$ and $b_P\of \cycle_P\to X$ as follows.
\begin{align*}
a_P\tuplet{\phi,\tuplet{a,b}} &= a(\phi)\\
b_P\tuplet{\phi,\tuplet{a,b}} &= b(\phi)
\end{align*}
This way $f\circ a_P = b_P\circ k_P$, and since $f$ is an acyclic fibration, there is a $c_P\of \simplex_P\to X$ such that $f\circ c_P = b_P$ and $c_P\circ k_P = a_P$. This is the key.

Back in $\ambient/\nno$, $c_P$ induces a morphism $p\to\hom_\nno(\simplex_\nno,\nno\ri(X))$, which is a section to the canonical morphism $\function c\tuplet{\nno\ri(f)\circ c,c\circ k}$ and hence a filler for $k$ against $\nno\ri(f)$. Therefore $f$ is contractible.
\end{proof}

\begin{proposition} Every morphism factors as a contractible morphism following a cofibration. \label{factor1} \end{proposition}

\begin{proof}
Let $\simCat^+$ be the subcategory of monomorphisms and $\simCat^-$ that of epimorphisms of $\simCat$ (see definition \ref{category of simplices}). For each morphism $\phi$ of $\simCat$ let $m(\phi)$ be the monic and $e(\phi)$ the epic factors.

The first step is to cover $Y$ with another simplicial set $LY$ where degeneracy is decidable. Define the simplicial set $LY$ as follows.
\begin{align*}
\base(LY) &= \set{\tuplet{\epsilon,y}\of \Ar(\simCat^-)\times Y\middle| \cod(\epsilon)=\dim(y)}\\
\dim(\tuplet{\epsilon,y}) &= \dom(\epsilon)\\
\tuplet{\epsilon,y}\cdot \phi &= \tuplet{e(\epsilon\circ \phi),y\cdot m(\epsilon\circ\phi)}
\end{align*}
A pair $\tuplet{\epsilon,y}$ is a face of $LY$ if and only if $\epsilon=\id$ and equality is a decidable property of morphisms in $\simCat$.
Let $l_Y\of LY\to Y$ satisfy $l_Y\tuplet{\epsilon,y}=y\cdot\epsilon$.

\hide{ $l_Y$ may be an equivalence, but not an acyclic fibration in a constructive way }

The second step glues simplices from $X$ and $LY$ together.
The pair $(f,l_Y)$ stands for the morphism $X+LY\to Y$ that satisfies $(f,l_Y)(x)=f(x)$ for $x\of X$ and $(f,l_Y)(y) = l_Y(y)$ for $y\of LY$.
Let $Z[n]\subseteq \product{i\of[n]}{(X[i]+LY[i])^{\simCat^+([i],[n])}}$ (where $\simCat^+([i],[n])$ is the object of monomorphisms $[i]\to [n]$) consist of elements $z$ which satisfy the following conditions for all $\alpha\of [j]\to [k]$ and $\beta\of [i]\to[j]$ in $\simCat^+$.
\begin{enumerate}
\item $(f,l_Y)(z(\alpha\circ\beta)) = (f,l_Y)(z(\alpha))\cdot\beta$;
\item if $z(\alpha)\of X$ then $z(\alpha\circ\beta)\of X$;
\item if $z(\alpha)=\tuplet{\epsilon,y}\of LY$ and $\epsilon\circ\beta$ is not monic, then $z(\alpha\circ\beta) = \tuplet{\epsilon',y'}\of LY$ and $\epsilon'$ factors through $\epsilon\circ\beta$.
\end{enumerate}
For $\phi\of[m]\to [n]$ and $z\of Z[n]$, let $z\cdot\phi$ satisfy the following equation for all $\alpha$ of $\simCat^+$.
\[ (z\cdot\phi)(\alpha) = z(m(\phi\circ\alpha))\cdot e(\phi\circ \alpha) \]
By these definitions, $Z=(\coproduct{n\of\nno}{Z[n]},\function{\tuplet{n,z}}n,\cdot)$ is a simplicial object.

Let $g\of X\to Z$ satisfy $g(x)(\alpha) = x\cdot\alpha$. It is a cofibration because a $z\of\base Z$ is nondegenerate and outside of the image of $g$ if and only if $z(\id_{[\dim(z)]}) = \tuplet{\id_{[\dim(z)]},y}$ for some $y\of Y[\dim(z)]$. The definition of $LY$ ensures that elements with this property are nondegenerate while condition 3 ensures that if $z(\id)$ is degenerate, then so is all of $z$.

Let $h\of Z\to Y$ satisfy $h(z) = (f,l_Y)(z(\id_{[\dim(z)]}))$. This morphism is contractible thanks to the following filler construction.
Let $k\of\cycle_\nno\to\simplex_\nno$ be the family of cycle inclusions and let $a\of \cycle_\nno\to Z$ and $b\of \simplex_\nno\to Y$ satisfy $h\circ a = b\circ k$. Since $k$ is essentially the the subobject of nonsurjective morphisms of $\simCat$ a filler $c$ can be defined as follows.
Let $c\of \simplex_\nno\to Z$ satisfy $c(\phi)(\mu) = \tuplet{\phi,b(\mu)}$ if $\phi$ is surjective and $c(\phi) = a(\phi)$ for $\phi$ in the image of $k$. By this definition $c\circ k = a$ and $h\circ c = b$.\end{proof}\hide{ Codomains of contractibles may have a distance function, that gives a lower bound to the number of compositions required to to reach it. I don't have time to work that out and check it right now.}

\begin{remark} The construction is the proof of lemma \ref{factor1} are related to Reedy model structures \citep{Reedy74}. In the ambient category $\ambient$ morphisms factor as split epimorphism following decidable monomorphisms thanks to coproducts. Cofibrations are lifted versions of decidable monomorphisms and contractible fibrations are lifted versions of split epimorphisms and they form a factorization system on $\ambient\s$ by a similar construction. Unfortunately, external induction is not strong enough to proof things in the internal language of $\ambient$.
\end{remark}

\begin{lemma} Contractible morphisms are acyclic fibrations.\label{Reedy}\end{lemma}

\begin{proof}
Suppose $c\of I\to J$ is a cofibration and $F$ is the decidable subobject of faces outside the image of $c$.
Define the following $\nno$-indexed family of subobjects of $J$.
\begin{align*} 
K &= (\nno\disc\times \set{c(i)| i\of I})\cup\set{ \tuplet{n,j}\of \nno\disc\times J \middle| \exists f\of F_{<n}, \xi\of\Ar(\simCat).f\cdot \xi = j }\\
&\textrm{where } F_{<n} = \set{f\of F|\dim(f) < n}
\end{align*}
For every $n\of \nno$ and all $j\of J$ with $\dim(j) < n$, $\tuplet{n,j}\of K$ as shown by the following induction argument. The base case $n=0$ is trivial because $\set{j\of J|\dim(j)<0}$ is empty. Assume $n = m+1$, $j\of J$ and $\dim(j)=m$ and assume that all $\tuplet{m,j'}\of K$ for all $j'\of J$ with $\dim(j') \leq m$. By definition of cofibration, either $j\of F$, $j\of \set{c(i)|i\of I}$ or $j=j'\cdot\epsilon$, where $\epsilon$ is epic and different from $\id$. The first two directly imply $\tuplet{n,j}\of J$. In the third, $\dim(j') < m$ because of $\epsilon$. Therefore $\tuplet{m,j'}$ and hence $\tuplet{m,j}\of K$. Since $F_{<m}\subseteq F_{<n}$, $\tuplet{n,j}\of K$ as well.

Let $f\of X\to Y$ is any morphism of $\ambient\s$. The successor function $s\of \nno\to\nno$ induces a endofunctor $s\ri$ of $\ambient\s/\nno\disc\to$ and a natural transformation $\sigma_X\of X \to s\ri(X)$. If $\nno\ri(f)$ has the right lifting property with respect to$\sigma_K$, then $f$ has the right lifting property with respect to $c\of I\to J$ for the following reasons. A filler operator for is an $\nno$-indexed morphism 
$p_n \of \hom(K_n,X)\times_{\hom(K_n,Y)}\hom(K_{n+1},Y) \to \hom(K_{n+1},X)$ that satisfies
$p_n\tuplet{a,b}\circ \sigma_{K,n} = a$ and $f\circ p_n\tuplet{a,b} = b$ for all $\tuplet{a,b}$ in its domain.
Here $K_n$ stands for the fibre of the projection $\function{\tuplet{i,j}}i\of K\to\nno\disc$ over the point $n\of \nno$.
Define $q_n\of\hom(K_0,X)\times_{\hom(K_0,Y)}\hom(J,Y)\to \hom(K_n,X)$ as follows:
\begin{align*}
q_0\tuplet{a,b} &= a\\
q_{n+1}\tuplet{a,b} &= p_n(q_n\tuplet{a,b},b\circ (\function{\tuplet{i,j}}j))
\end{align*}
Since every $j\of J$ is in $K_n$ for some $n\of\nno$ and $q_{n+1}\tuplet{a,b}(j) = q_n\tuplet{a,b}(j)$ if $j\of K_n$, the join of the $q_n$ is a morphism $q_\omega\of\hom(K_0,X)\times_{\hom(K_0,Y)}\hom(J,Y)\to \hom(J,X)$. This morphism satisfies $f\circ q_\omega\tuplet{a,b} = b$.
The morphism $c\of I \to K_0 = \set{c(i)|i\of I}$ is an isomorphism since cofibrations are monic. For all $a\of I\to X$ and $b\of J\to Y$ such that $f\circ a = b\circ c$, $q_\omega\tuplet{a\circ c^{-1},b}\circ c = a\circ c^{-1}\circ c = a$. Hence $\function{\tuplet{a,b}}q_\omega\tuplet{a\circ c^{-1},b}$ is a filler for $c$ against $f$.
\hide{Notation may need some work, but the argument is now correct}

The simplicial sets $\simplex_F$ and $\cycle_F$ satisfy:
\begin{align*}
\base(\simplex_F) &= \set{ \tuplet{\phi,x}\of\Ar(\simCat)\times F | \cod(\phi) = \dim(x) } \\
\base(\cycle_F) &= \set{\tuplet{\phi,x}\of\base(\simplex_F)\middle|\exists i\of\nno.i\leq \cod(\phi),\forall j\of \dom(\phi).\phi(j)\neq i} \\
\dim\tuplet{\phi,x} &= \dom(\phi) \\
\tuplet{\phi,x}\cdot \xi &=\phi\circ\xi
\end{align*}
The morphism $\sigma_K$ is a pushout of the inclusion $k_F\of\cycle_F\to\simplex_F$, along $\function{\tuplet{\phi,f}}\tuplet{\dim(f),f\cdot \phi}$, for the following reasons. Let $p\of K \to X$ and $q\of \simplex_F\to X$ satisfy $p\circ \function{\tuplet{\phi,f}}\tuplet{\dim(f),f\cdot \phi} = q\circ k_F$. 
The factorization $\tuplet{p,q}\of s\ri(K)\to X$ of $p$ and $q$ through $s\ri(K)$ satisfies $\tuplet{p,q}\circ \sigma_K = p$ and $\tuplet{p,q}\circ \function{\tuplet{\phi,f}}\tuplet{\dim(f)+1,f\cdot \phi} = q$. 
For each $\tuplet{n,j}\of s\ri(K)$, $j = c(i)$ for $i\of I$ or $j = f\cdot \phi$ for $f\of F$ with $\dim(f)\leq n$. If $j=c(i)$ then $\tuplet{p,q}(\tuplet{n,j}) = p(\tuplet{n-1,j})$. If $j = f\cdot \phi$ then $n=\dom(\phi)+1$ and $\tuplet{p,q}(\tuplet{n,j}) = q(\tuplet{\id,f})\cdot\phi$, which demonstrates existence. 
If $c(i)=f\cdot\phi$, then $\phi$ cannot be an epimorphism, since epimorphisms have right inverses, and that would put $f$ in the image of $c$ where it is not by definition \ref{cofibration}. Hence $\tuplet{\phi,f}\of\cycle_F$ and $p(\tuplet{\dom(\phi),c(i)})=q(\tuplet{\phi,f}) = q(\tuplet{\id,f})\cdot\phi$. This proves uniqueness and existence of $\tuplet{p,q}$.

If $f\of X\to Y$ is any morphism of $\ambient$, then $\nno\ri(f)$ has the right lifting property with respect to $\sigma_K$ if it has the left lifting property with respect to $k_F$.

The last step is the lifting property with respect to $k_F$. The morphism $k_F$ is the \keyword{tensor} of $k_\nno\of\cycle_\nno\to\simplex_\nno$ for the object $\dim\of F\to\nno$ of $\ambient/\nno$. This means $F\times \hom(X,Y) = \hom(F\otimes X,Y)$.
The power construction preserves split epimorphisms and hence fillers. Therefore $\nno\ri(f)$ has a filler for $k_F$ it it has one for $k_\nno$.

So if $f$ is contractible, $\nno\ri(f)$ has the right lifting property with respect to all tensors of $k_\nno$, their pushouts and all the induced transfinite compositions. These include all cofibrations.
\end{proof}

\end{document}