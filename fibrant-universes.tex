\documentclass{tac}
\usepackage{amssymb, amsmath, stmaryrd}
\usepackage[backend=bibtex,citestyle=authoryear-icomp]{biblatex}
\usepackage[all]{xy}
\usepackage{url}

\title{Fibrant universes---a combinatorial proof}
\author{Wouter Pieter Stekelenburg}
\copyrightyear{2019}
\eaddress{w.p.stekelenburg@gmail.com}
\keywords{simplicial homotopy, fibrant objects}
\amsclass{03D80, 18G30, 18G55}
\addbibresource{realizability}

%structural helpers
\newcommand\hide[1]{}

%symbols
\newcommand\cat{\mathcal}
\newcommand\icat{\mathsf}
\newcommand\of{\mathord:}
\newcommand\set[1]{\left\{#1\right\}}
\newcommand\dual{^\circ}
\newcommand\nno{\mathbb{N}}
\newcommand\ri{^*}

\mathrmdef{Ar}
\mathrmdef{Ob}
\mathrmdef{id}
\mathrmdef{dom}
\mathrmdef{cod}
\newcommand\colim{\mathop{\mathrm{colim}}}
\newcommand\tuplet[1]{\left\langle{} #1 \right\rangle}
\newcommand\pulled{\ar@{}[dr]|(.2)\lrcorner}

\newcommand\simplex[1]{\Delta^{#1}}
\newcommand\horn[2]{\Lambda^{#1}_{#2}}
\newcommand\cycle[1]{\partial\Delta^{#1}}
%yoneda embedding of the simplex category. needs work.
\newcommand\Y{\Delta}

\begin{document}
\begin{abstract} This paper presents a combinatorial proof that Kan fibrations
  descend along acyclic cofibration. This indirectly shows that certains
  universes of fibrations have a fibrant codomains.
\end{abstract}

\maketitle

\section{Introduction}
This paper grew out of attempts to develop simplicial homotopy inside
realizability toposes to find interesting models of homotopy type theory, in
particular the category of \emph{modest Kan complexes} in the category of 
\emph{simplicial assemblies}.\hide{todo: refer to a point in the paper where 
these are explained}
Unfortunately, classical simplicial homotopy theory doesn't work in 
realizability toposes, because the classical theory is rife with arguments that 
rely on classical logic and the completeness of various large categories.
\hide{this could also do with explanation or citations} There are a growing 
number of approaches that could help,\hide{cite! Garner, Swan, Gambino, Henry, 
Stattler, Szumilo\dots} but demonstrating that these work in realizability 
toposes is more work than I am willing and able to do (having left academia). 
Instead I aim to make a modest contribution to constructive simplicial homotopy 
by showing a constructive solution to a specific problem that apparently doesn't
have one yet.

\mathrmdef{sSet}
\section{The problem}

\begin{proposition}
Let \( h^n_k\of \horn{n}{k} \to \simplex{n} \) be a horn inclusion in the category of
simplicial sets and let \( f\of X\to \horn{n}{k} \) be a Kan complex. There is a
Kan complex \( Df\of X \to \simplex{n} \) such that \( f \) is the pullback of 
\( Df \) along \( h^n_k \). Moreover, there is a functor 
\( D\of \sSet/\horn{n}{k}\to \sSet/\simplex{n} \) that preserves Kan complexes, 
with a left adjoint \( A \) that preserves acyclic cofibrations, such that 
\( (h^n_k)\ri \circ D\simeq \id \).
\[\xymatrix{
X\ar[d]_f\ar@{.>}[r]\pulled & DX \ar@{.>}[d]^{Df}\\
\horn{n}{k} \ar[r]_{h^n_k} & \simplex{n}
}\]
\end{proposition}
\hide{ 
  lots of undefined terms here\dots 

  Jaap would interupt me before the first sentence ended, but we could go over
  the context and the definitions down here, using the informal decription
  above as a guide.

  Maybe I could add a `bear with me' on top.
}
\hide{
  Functors \( D^n_k \) and \( A^n_k \)? 
}
\begin{proof}
Definition~\ref{descend functor} describes the functor, the 
adjunction is demonstrated at Lemma~\ref{adjunction} and 
Lemma~\ref{inverse} shows that the functor is inverse.
\end{proof}

\section{The ascend functor}
The solution starts with the left adjoint 
\( A\of \sSet/\simplex{n} \to \sSet/\horn{n}{k} \). This functor is similar 
enough to reindexing along \( h\ri \) to allow its right adjoint to satisfy 
\( h\ri\circ D = \id \), but it should also preserve acyclic cofibrations. 

It is instructive to first understand why \( h\ri \) doesn't preserve all 
acyclic cofibrations. The problem boils down to the simplex
\( \delta^n_k\of \simplex {n-1}\to\simplex{n} \) \hide{\( \delta^n_k \)?},
whose pullback along \( h \) is a cycle 
\( h\ri(\delta^n_k)\of \cycle {n-1}\to\horn{n}{k} \).
\[\xymatrix{
\cycle {n-1}\ar[d]\ar[r]\pulled & \simplex {n-1} \ar[d]^{\delta^n_k}\\
\horn{n}{k} \ar[r]_h & \simplex{n}
}\]
All the other simplices \( \simplex {n-1}\to\simplex{n} \) are preserved in the pullback
and therefore acyclic cofibrations over these simplices are preserved as well,
but those over \( \delta^n_k \) get cycles cut into them.

The functor \( A \) solves this by applying \( h\ri \) after inflating simplices over 
\( \delta^n_k \) in a way that makes their pullbacks acyclic, while keeping other
simplices untouched. For any \( f\of X\to \Delta_n \) the inflation involves glueing
a copy of the exponential \( f^{\delta^n_k} \) to the fibre of \( f \) over the edge 
\( k \), together with a lot of bridging simplices.

The construction requires some bookkeeping, to deal with the structure of the
simplex category. \hide{simplex category?}

Firstly, things get a lot easier with an \emph{`empty simplex'} \( [-1] \), an 
initial object for the simplex category. This new simplex category is known as 
the \emph{augmented simplex category}. For each simplical set \( X \) we let 
\( X[-1] \) be a terminal object---i.e.\ a singleton or one element set---so the 
required restrictions maps to \( X[-1] \) are all uniquely determined.

Secondly, the members of the inflated simplicials sets are simplices that
serve as paths between a `source' simplex of \( f \) and a `target' simplex of 
\( f^{\delta^n_k} \). The action of the simplex category on these paths
is defined in term of the action on sources and targets, but this requires
spliting morphisms \( \phi \) of the simplex category into two parallel parts
\( \phi_0 \) and \( \phi_1 \).

\newcommand\source{\mathbf{s}}
\newcommand\target{\mathbf{t}}
Let \( m\of\nno \), \( \xi\of[m]\to[n] \), \( y\of \nno \). For 
each \( \phi\of[a]\to[m + y + 1] \) let:
\begin{align}
  \label{max0} m_0 &= \max(\set{-1}\cup\set{i\of[a]\middle|
    \phi(i)\leq m, \xi(\phi(i)) < k
  })\\
  \label{max1} m_1 &= \max(\set{-1}\cup\set{i\of[a]\middle|
    \phi(i) > y, \xi(\phi(i)-y-1)\leq k
  })\\
  \label{source0} \source(\xi, y, \phi)&\of
    [a - m_1 + m_0]\to[n]\\
  \label{source1} \source(\xi, y, \phi)(i)&= 
    \left\{
      \begin{array}{rl}
        \phi(i) &\textrm{if } i \leq m_0 \\
        \phi(i + m_1 - m_0) - y - 1 &\textrm{if } i > m_0
      \end{array}
    \right. \\
  \target(\xi, y, \phi)&\of\label{target0} 
    [m_1 - m_0 - 1]\to[n]\\
  \target(\xi, y, \phi)(i) &=\label{target1} 
    \phi(i + m_0 + 1) - \phi(m_0 + 1)
\end{align}

With these helpers, the \emph{ascend functor} can be defined.
\begin{definition}
  For each \( f\of X\to \simplex{n} \), \( AX \) is the 
  presheaf over the augmented simplex category that
  satisfies the following equations.
  \begin{align} 
    \label{AX1} AX[n] &= \set{
      \tuplet{x,y}\of\sum_{k+l=n-1} X[k]\times X^{\delta^n_k}[l]
    \middle|
      f(x)\of \horn{n}{k}
    }\\
    \label{AX2} \tuplet{x,y}\cdot\phi &= \tuplet{
      x\cdot\source(f(x),\dim(y),\phi),
      y\cdot\target(f(x),\dim(y),\phi)
    }
  \end{align}

  Let \( Af\of AX\to \horn{n}{k} \) satisfy the following.
  \[Af\tuplet{x,y}\of[\dim(x)+\dim(y)+1]\to[n]\]
  \begin{multline}
    \label{Af} Af\tuplet{x,y}(i) = \\
    \begin{array}{rl}
      f(x)(i) &
      \textrm{ if } i\leq\dim(x), f(x)(i)<k \\
      f(x)(i-\dim(y)-1) &
      \textrm{ if } i > \dim(y), f(x)(i-\dim(y)-1)\geq k\\
      k &
      \textrm{ otherwise}
    \end{array}
  \end{multline}
  
  For each \( m\of Y\to X \), let \( Am\of AY \to AX \) satisfy: 
  \begin{equation} \label{Am}
    Am\tuplet{y_0,y_1} = \tuplet{m(y_0),m\circ y_1}
  \end{equation}

  This defines the \emph{ascend functor} \(
    A\of \sSet/\simplex{n} \to \sSet/\horn{n}{k}
  \)
\end{definition}

Let's check the correctness of this definition.

\begin{lemma} The ascend functor \(A\) is a functor 
\(\sSet/\simplex{n} \to \sSet/\horn{n}{k}\). \end{lemma}

\begin{proof} 
  Let \(f\of X\to \simplex{n} \). The condition
  \(f(x)\of\horn{n}{k}\) in (\ref{AX1}) ensures that 
  \(f(x)\) is not onto \([n]-\set{k}\). By 
  (\ref{Af}) \(Af(x)\) isn't onto either. Hence
  \(Af\of AX\to \horn{n}{k} \) as far as the underlying sets are
  concerned.

  That \(AX\) is valued in simplical sets follow from (\ref{AX2}),
  (\ref{source1}) and (\ref{target1}). The latter two explain how
  morphisms are split up to act on the members of the tuplet,
  and this is done in a way that commutes with composition and
  identity so \(\cdot \) is a proper action of the (augmented)
  simplex category on \(AX\).

  Let \(m\of Y \to X\), so it forms a morphism 
  \(f\circ m \to m\) in \(\sSet/\simplex{n}\). Equation
  (\ref{Am}) makes it straighforward that \(A\) preserves
  identitiies and compositions, and crosschecking with 
  (\ref{AX2}) shows that \(A\) commutes with \(\cdot \).
\end{proof}

\hide{
  has right adjoint (which hence preserves fibrations)
  which is right inverse of reindexing
}
\section{Preservation of acyclic cofibrations}
This section covers the most complicated part.
Essentially, \( f^{\delta^n_k} \) is vulnerable to the pullback, 
so \( A \) creates a backup for this part over the top edge of 
the horn \( \horn{n}{k} \). The pullback does not
cut a cycle into the inflated \( f^{\delta^n_k} \), which
therefore remains acyclic, as demonstrated below.

\begin{lemma}
  Let \(\xi\of[m]\to[n]\) in the simplex category and let \(
    \Y(\xi)\of \simplex{m} \to \simplex{n} \) be its Yoneda embedding.
  Let \(h^m_l \of \horn{m}{l}\to\simplex{m}\) be the
  horn inclusion. Then \( A(\Y(\xi)\circ(h^m_l))\to A(\Y(\xi)) \) is an acyclic
  cofibration.
\end{lemma}

\begin{proof}
  The strategy here is to show that \( A(\Y(\xi)\circ(h^m_k))\to A(\Y(\xi)) \) 
  is the composition of a series of pushouts of sums of horn inclusions \(
    H_i\to H_{i+1} \), where \( H_0 = A(\Y(\xi)\circ(h^m_k)) \).
  This clearly produces an acyclic cofibration.
  Each \(H_{i+1}\) is generated by picking faces \(a\) of \(A(\Y(\xi))\) for 
  which \(a\cap H_i\to H_a\) is a horn inclusion. Because all faces of 
  \(A(\Y(\xi))\) get glued in along the way, \(H_j=A(\Y(\xi))\) from some 
  \(H_j\) on, proving that  \( A(\Y(\xi)\circ(h^m_k))\to A(\Y(\xi)) \) is the
  constructed acyclic cofibration, and hence proving the lemma.
  This leaves the question of which faces of  \(A(\Y(\xi))\) to include at every 
  step.
  
  Let's peek inside. Simplices of \( A(\Y(\xi)) \) are pairs \(\tuplet{x,y}\)
  where \(x\of[a]\to [m]\) is a nondecreasing map and \(
    y\of [b]\to{[m]}^{[n-1]}\) subject to the restrictions that both are 
  nondecreasing map (both variables in case of \(y\)), \(
    \xi\circ x\of \horn{n}{k}\)---i.e. \(x\) misses some edge different from 
  \(k\)---and \(\xi\circ y(i) = \delta^n_k \) for all \( i\of [b] \). Among 
  these, faces are the simplices for which both \(x\) and \(y\) are injective 
  functions.

  The set \( A(\Y(\xi)\circ(h^m_l)) \) can be considered to be the subset of 
  \( A(\Y(\xi)) \) of those pairs \(\tuplet{x,y}\) that miss at least one element
  \(p\) from \([m]\) different from the top \(l\) both with \(x\) and \(y\). 
  For \(y\) this means \(y(i)(j)\neq p\) for all \(i\of[b]\) and \(j\of[n-1]\).

  When selecting faces to include into \(H_i\) we have to be careful not to
  accidentally create cycles, because they form boundaries of faces that
  still need to be glued in. For this purpose, consider the following two
  collection of faces with suggested top edges.
  \begin{enumerate}
    \item The collection \(P\) of faces \( \tuplet{x,y} \) such that \( x \) 
    reaches \(l\) but \(\xi\circ x\) misses some element in 
    \([n]-\set{k,\xi(l)}\).
    \item For \( \tuplet{x,y} \) such that 
    \( \xi\circ x \) is surjective on \([n]-\set{k,\xi(l)}\), 
    let \(l'(x)\of {[m]}^{[n-1]}\) be the unique edge that reaches 
    the maximal value of \(x\) in each fibre of \(\xi \) except the one over
    \(\xi(l)\) where its value is \(l\).
    The collection \(Q\) consist of faces \( \tuplet{x,y} \) such that 
    \( \xi\circ x \) is surjective on \([n]-\set{k,\xi(l)}\)
    and where \(y\) reaches \(l'(x)\).
  \end{enumerate}

  Let's first check that the faces of \(P\cup Q\) cover all the faces in 
  the complement \( A(\Y(\xi)) - A(\Y(\xi)\circ(h^m_k)) \). 
  First, note that \(\xi\circ x\) can never be onto \([n]-\set{k}\) because 
  \(A(\Y(\xi))\) is a simplicial set over \(\horn{n}{k}\).
  For a face \(\tuplet{x,y}\) of the complement \(xi\circ x\) either is or isn't 
  onto \([n]-\set{k,\xi(l)}\). %the classical argument!
  If it is onto, then extend \(y\) with the edge \(l'(x)\), to get an 
  element in \(Q\).
  If it isn't, then extending \(x\) with \(l\) won't make it so, and this gives 
  an element in \(P\). For this reason, it is sufficient to show the presence of 
  faces in \(P\cup Q\) to prove the lemma.

  Simply adding faces based on dimension still doesn't work out, but a similar
  metric does. Put a \emph{price} on each edge in \([m]\) and \({[m]}^{[n-1]}\):
  \begin{enumerate}
    \item In \([m]\) the top edges, i.e. \(l\) and all \(i\of [m]\) such 
    that \(\xi(i) = k\) are worth 1;
    \item In \([m]\) all the other edges are worth 2;
    \item In \({[m]}^{[n-1]}\) all edges are worth 1.
  \end{enumerate}
  The price of a face is simply the sum of the prices of the edges it contains. 
  Faces from both \(P\cup Q\) are included in \(H_i\) if they cost at 
  most \(i\).

  For each face \(\tuplet{x,y}\of P\)
  all the boundary faces \(\tuplet{x,y}\cdot\delta_i\) except the 
  one that misses \(l\) is in \(P\) and they cost either \(i\) or \(i-1\)
  depending on the price of the removed point. Hence those faces are part of
  \(H_i\) and \(\tuplet{x,y}\cap H_i \to \tuplet{x,y}\)---where the element is
  reinterpreted as the subobject it generates---is a horn inclusion.
  
  If \(\tuplet{x,y}\of Q\), then the boundary faces 
  \(\tuplet{x_i,y_i} = \tuplet{x,y}\cdot\delta_i\) fall in several different
  categories that demonstrate why some edges cost more than others. 
  \begin{itemize}
  \item Those that remove an edge of \(y\)--except \(l'(x)\)--are members of 
  \(Q\) 
  that cost one less. 
  \item Those that remove an edge of \(x\) in the fibre of \(\xi \) over \(k\) 
  are also 
  members of \(Q\) that cost one less. Those that remove an edge of \(x\) in 
  the fibre of \(\xi \) over some other member come in three kinds.
  \item Those that remove an element that differs from the maximal element of 
  their 
  fibre have \(l'(x_i) = l'(x)\), which means they are members of \(Q\) that 
  simply cost two less. 
  Those that remove the maximal element of their fibre either empty the fibre
  or not. 
  \item In the empty case \(\xi\circ x_i\) is not onto, and adding \(l\) results
  in a member of \(P\) that still costs one less.
  \item In the non empty case, adding \(l'(x_i)\) results in a member of \(Q\)
  that still costs one less.
  \end{itemize}
  So if \(\tuplet{x,y}\) is a member of \(Q\) that costs \(i+1\), then
  \(\tuplet{x,y}\cap H_i \to \tuplet{x,y}\) is a horn inclusion as well.
  That completes the proof that \(A\) peserves acyclic cofibrations.
\end{proof}

\section{The adjunction}
A lot of nice properties of \(A\) follow come from the fact 
that it has a right adjoint, so let's look at that adjoint
now.

\begin{definition}\label{descend functor}
Let \(f\of X\to \horn{n}{k} \) and define 
\(Df\of DX\to \simplex{n}\) as follows.
\begin{align}
  \label{DX1} DX &= \set{\tuplet{
    \xi\of [m]\to [n],
    y \of \sSet/\horn{n}{k}(A(\Y(\xi)), f) 
  }\middle|m\of\nno}\\
  \label{DX2} \tuplet{\xi,y}\cdot\phi &= \tuplet{
    \xi\circ \phi,
    y\circ A(\Y(\phi))
  }\\
  \label{Df} Df \tuplet{\xi,y} &= \xi
\end{align}
Here we take advantage of the fact that \(\sSet \) is locally 
small.

For \(m\of Y\to X\), let \(Dm\of D(f\circ m)\to D(f)\) satisfy:
\begin{equation}
  \label{Dm} Dm \tuplet{\xi,y} = \tuplet{\xi,m\circ y}
\end{equation}

This defines the \emph{descend functor}
\(D\of \sSet/\horn{n}{k} \to \sSet/\simplex{n}\).
\end{definition}

Let's first check whether the \(D\) is a functor.

\begin{lemma} The \emph{descend functor}
  \(D\) is a functor \(\sSet/\horn{n}{k} \to \sSet/\simplex{n}\).
\end{lemma}

\begin{proof}
  Let \(f\of X\to \horn{n}{k}\). Equation (\ref{DX1}) clearly
  shows a dependent product indexed over the underlying set of
  \(\simplex{n}\) and (\ref{Df}) clearly commutes with the
  action of the simplex category, as defined in (\ref{DX2}) for
  \(DX\). Hence the object map is good.

  Let \(m\of Y\to X\) be a morphism of simplicial sets, so
  \(m\) is also a morphism \(f\circ m\to f\). Equation 
  (\ref{Dm}) implies that \(D\) commutes with identities
  and composition, just as a functor should.
\end{proof}

\begin{lemma} The descend functor \(D\) is right adjoint to the
  ascend functor \(A\). \label{adjunction}
\end{lemma}

\begin{proof} Adjoints are unique up to isomorphism and any 
  right adjoint \(R\) to \(A\) must satisfy:
  \[ \sSet(\Y(\xi), Rf)/\simplex{n}\simeq
     \sSet(A\Y(\xi),f)/\horn{n}{k} \]
  For \(R=D\) this natrual equivalence follows straight from 
  (\ref{DX1}) and (\ref{Df}). Hence \(D\) is right adjoint to 
  \(A\).
\end{proof}

This adjunctions has far reaching consequences.
\begin{lemma} The descend functor preserves fibrations,
  the ascend functor all acyclic cofibrations.
\end{lemma}

\begin{proof} This is a matter of transposing lifting
  problems between the categories.
\end{proof}

This ultimately amounts to a proof that the homotopy categories
over the weakly equivalent spaces \(\simplex{n}\) and 
\(\horn{n}{k}\) are equivalent, as they should be.
We explore a different path here. %here
%this doesn't really belong in the adjunction section.

\begin{lemma} The descend functor is right inverse to reindexing
  along the horn inclusion 
  \(h^n_k\of\horn{n}{k}\to\simplex{n}\).
  \[(h^n_k)\ri\circ D \simeq \id \] \label{inverse}
\end{lemma}

\begin{proof} 
  For \(\xi\of\horn{n}{k}\), \({\Y(\xi)}^{\delta^n_k}\) is
  empty because \(\xi \) cannot be onto \([n]-\set{k}\).
  Therefore, by (\ref{AX1}), 
  \(A\Y(\xi) \simeq (h^n_k)\ri(\Y(\xi))\). In turn this means
  that for \(f\of X\to \simplex{n}\), the fibre of \(Df\) over
  \(\xi \)---which is isomorphic to 
  \(\sSet((h^n_k)\ri(\Y(\xi)),f)\) by (\ref{DX1})---is
  isomorphic to the fibre of \(f\) over \(\xi \) by the Yoneda
  lemma.
\end{proof}



\end{document}