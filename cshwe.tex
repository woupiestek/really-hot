\documentclass[csh.tex]{subfiles}
\begin{document}
\section{Weak Equivalences}
This section shows that the weak equivalences of definition \ref{weak equivalence} satisfy the two out of three property (see definition \ref{model structure}).

\begin{lemma}[2-out-of-3] Let $f\of X\to Y$ and $g\of Y\to Z$ be morphisms of $\ambient\s\f$. If any two of $f,g$ or $g\circ f$ are weak equivalences, then all three are. \label{toot}\end{lemma}

\begin{proof} Weak equivalences are closed under composition by lemma \ref{composition of weak equivalences}.

Let $g$ and $g\circ f$ be arbitrary weak equivalences. The morphism $f$ factors as an acyclic fibration $h\of W\to Y$ following a cofibration $k\of X\to W$ by proposition \ref{factor2}. Because weak equivalences are closed under composition, $g\circ h$ is a weak equivalence. The morphism $k$ is acyclic for the following reasons. Factor both $g\circ f$ and $g\circ h$ as acyclic fibrations following acyclic cofibrations, so $g\circ f = a\circ b$ and $g\circ h = c\circ d$. The lifting properties induce a morphism $l$ such that $l\circ b = d\circ k$ and $c\circ l = a$. Lemma \ref{shared retract 2} says that $l$ is a weak equivalence because $a$ and $c$ are acyclic fibrations. 
Because of closure under composition, the morphism $l\circ b = d\circ k$ is both a weak equivalence and a cofibration and hence an acyclic cofibration. Since $d$ and $d\circ k$ are acyclic cofibrations, so is $k$ by lemma \ref{left cancellation}.
\[\xymatrix{
X\ar[r]_k\ar@/^2ex/[rr]^f\ar[d]_b & W\ar[r]_h\ar[d]^d & Y\ar[d]^g \\
\bullet\ar@/_2ex/[rr]_a\ar@{.>}[r]^l & \bullet\ar[r]^c & Z
}\]
Since $f = h\circ k$, $f$ is a weak equivalence.

The case where $f$ and $g\circ f$ are weak equivalences is dual to the case above and the reasoning is the same. Compositions of acyclic cofibrations are weak equivalence by lemma \ref{shared retract} where compositions of acyclic fibrations are by lemma \ref{shared retract 2}. Acyclic fibrations satisfy lemma \ref{right cancellation} where acyclic cofibrations satisfy lemma \ref{left cancellation}.

This means that weak equivalences indeed satisfy 2-out-of-3.
\end{proof}

\begin{lemma} Weak equivalences are closed under composition. \label{composition of weak equivalences}\end{lemma}

\begin{proof} Compositions of acyclic fibrations are acyclic fibrations and the same holds for acyclic cofibrations.
So all compositions of weak equivalences are weak equivalences, if $g\circ f$ factors as an acyclic fibration following an acyclic cofibration for each acyclic cofibration $g$ and acyclic fibration $f$ that can be composed.

By proposition \ref{factor1} $g\circ f=h\circ k$ for some acyclic fibration $h\of W\to Z$ and a cofibration $k\of X\to W$. To show that the cofibration is acyclic, the rest of this proof shows that $k$ is weakly invertible (see definition \ref{weakly invertible}) because $g$ is. That acyclic cofibrations are weakly invertible and vice versa is proved in lemmas \ref{acyclic have lifting} and \ref{lifting is acyclic}.

Let $g'$ be the left inverse of $g$. Since $f \circ \id = g'\circ g\circ f= (g'\circ h)\circ k$ there is a morphism $k'$ such that $f\circ k' = g'\circ h$ and $k'\circ k = \id$, so $k$ has its own left inverse.
\[\xymatrix{
X\ar[d]_f \ar[r]_{k} \ar@/^2ex/[rr]^{\id} & W\ar[d]^h \ar[r]_{k'} & X\ar[d]^f\\
Y \ar[r]^{g} \ar@/_2ex/[rr]_{\id} & Z \ar[r]^{g'} & Y
}\]

Let $\phi$ be a homotopy between $\id_Z$ and $g\circ g'$ such that $\phi\circ (\id\times g)=g\circ\pi_1$. This homotopy induces a homotopy $\chi$ between $\id_W$ and $k\circ k'$ by lemma \ref{triple lift}.
\begin{align*}
(\id, k\circ k')\circ (k+k) &= (k,k) = k\circ \pi_1\circ(c\times\id_X)\\
h\circ (k \circ \pi_1) &= (\phi\circ (\id_W\times h))\circ (\id_{\simplex[1]} \times k)\\
h\circ (\id, k\circ k') &= (\phi\circ (\id_W\times h))\circ (c\times \id_W)
\end{align*}
\[\xymatrix{
X+X\ar[d]_{c\times \id}\ar[r]^{k+k} & W+W\ar[d]_(.3){c\times \id}\ar[r]^(.6){(\id,k\circ k')} & W\ar[d]^h\\
\simplex[1]\times X\ar[r]_{\id\times k} \ar[urr]^(.3){k\circ \pi_1}  & \simplex[1]\times W\ar[r]_(.6){\phi\circ (\id\times h)} \ar@{.>}[ur]_\chi & Z
}\]
Because the homotopy satisfies $\chi\circ(\id_{\simplex[1]}\times k) = k\circ \pi_1$, $k$ is weakly invertible.
\end{proof}

\begin{lemma} If $f\of X\to Y$, $g\of Y\to Z$ and if $g$ and $g\circ f$ are acyclic cofibrations, then $f$ is an acyclic cofibration \label{left cancellation} \end{lemma}

\begin{proof} Let $k\of A\to B$ be a fibration and let $a\of X\to A$ and $b\of Y\to B$ satisfy $k\circ a=b\circ f$. Because $B$ is a complex, there is a $b'\of Z\to B$ such that $b'\circ g = b$. Lifting properties also imply that there is an $a'\of Z\to A$ such that $a'\circ g\circ f = a$ and $k\circ a'= b'$. So $b'\circ g$ is a filler for $k\circ a=b\circ f$. Generalizing, this construction lifts $f$ against all fibrations and that makes $f$ an acyclic cofibration.
\[\xy
(0,28)*+{X}="x",(14\halfrootthree,21)*+{A}="a",(0,14)*+{Y}="y",(14\halfrootthree,7)*+{B}="b",(0,0)*+{Z}="z"
\ar^a "x";"a" \ar_f "x";"y" \ar^k "a";"b" \ar^(.25)b "y";"b" \ar_g "y";"z"
\ar@{.>}_{b'} "z";"b" \ar@{.>}^(.75){a'} "z";"a"
\endxy\]
\end{proof}

\begin{lemma} If $f\of X\to Y$, $g\of Y\to Z$ and if $f$ and $g\circ f$ are acyclic fibrations, then $g$ is an acyclic fibration. \label{right cancellation}\end{lemma}

\begin{proof} Being acyclic is equivalent to being contractible (see definition \ref{contractible}) by example \ref{acyclic means contractible} and lemma \ref{Reedy}. Reasoning dual to that in the proof of lemma \ref{left cancellation} gives $g$ the right lifting property for the family of cycle inclusions.

\[\xy
(28,28)*+{X}="x", (0,21)*+{\coproduct{n\of\nno}\cycle[n]}="v", (28,14)*+{Y}="y", (0,7)*+{\coproduct{n\of\nno}\simplex[n]}="w", (28,0)*+{Z}="z"
\ar@{.>}^(.67){a'} "v";"x" \ar_(.67)a "v";"y" \ar "v";"w"
\ar@{.>}_(.75){b'} "w";"x" \ar_(.67)b "w";"z"
\ar^f "x";"y" \ar^g "y";"z"
\endxy\]

Dual reasoning works because the unique morphism $\coproduct{n\of\nno}\cycle[n]$ is a cofibration. A member of $\base(\coproduct{n\of\nno}\cycle[n])$ is a face if it is the pair $\tuplet{n,\id_n}$ for some $n\of\nno$ and this is a decidable property. No faces are in the image of $\bang$.
\end{proof}

\begin{lemma} If $f\of X\to Y$, $g\of Y\to Z$ and if $g$ and $g\circ f$ are acyclic fibrations, then $f$ is a weak equivalence.\label{shared retract 2} \end{lemma}

\begin{proof} By proposition \ref{factor1}, $f$ factors as an acyclic fibration $h\of W\to Y$ following a cofibration $k\of X\to W$. Because $(g\circ f)\circ \id = (g\circ h)\circ k$ and $g\circ f$ is an acyclic fibration, $k$ has a left inverse $k'\of X\to W$ which satisfies $g\circ f\circ k' = g\circ h$. 
\[\xymatrix{
X\ar[d]_k \ar[r]^\id & X\ar[d]^{g\circ f}\\
W\ar[r]_{g\circ h} \ar@{.>}[ur]^{k'} & Z
}\]

Let $c\of 1+1\to\simplex[1]$ be the same cycle as above. There is homotopy $\phi$ between $\id_W$ and $k\circ k'$ by lemma \ref{triple lift} and the following equations.
\begin{align*}
(\id,k\circ k')\circ(k+k) &= (k,k) = (k\circ\pi_1)\circ (c\times\id_X)\\
(g\circ h)\circ (k\circ \pi_1) &= (g\circ h\circ \pi_1)\circ(\id\times k)\\
(g\circ h)\circ (\id,k\circ k') &= (g\circ h\circ \pi_1)\circ(c\times\id_W)
\end{align*}
\[\xymatrix{
X+X \ar[d]_{c\times \id}\ar[r]^{k+k} & W+W\ar[d]_(.3){c\times\id}\ar[r]^(.6){(\id,k\circ k')} & W\ar[d]^{g\circ h}\\
\simplex[1]\times X \ar[r]_{\id\times k}\ar[urr]^(.3){k\circ\pi_1}& \simplex[1]\times W\ar[r]_{g\circ h\circ \pi_1}\ar@{.>}[ur]_\phi & Z
}\]
Because $\phi\circ (\id_{\simplex[1]}\times k)=k\circ \pi_1$, definition \ref{weakly invertible} says that $k$ is a weakly invertible cofibration and lemma \ref{acyclic have lifting} says that $k$ is an acyclic cofibration. Therefore $f = h\circ k$ is a weak equivalence.\end{proof}

\begin{lemma} If $f\of X\to Y$, $g\of Y\to Z$ and if $f$ and $g\circ f$ are acyclic cofibrations, then $g$ is a weak equivalence.\label{shared retract} \end{lemma}

\begin{proof} \hide{The morphism $g$ factors as a fibration $h\of W\to Z$ following an acyclic cofibration $k\of Y\to W$ by lemma \ref{factor2}. Lifting properties give $h$ a right inverse $h'$.
\[\xymatrix{
X\ar[r]^{k\circ f}\ar[d]_{g\circ f} & W\ar[d]^h\\
Z\ar[r]_\id \ar[ur]^{h'} & Z
}\]
Lemma \ref{triple lift} provides a homotopy $\phi$ between $\id_W$ and $h'\circ h$, because the following equations hold.
\begin{align*}
(\id,h'\circ h)\circ(k\circ f+k\circ f)&= (k\circ f,k\circ f) = (k\circ f\circ \pi_1)\circ(c\times \id_Y)\\
h\circ(k\circ f\circ\pi_1) &= g\circ f\circ\pi_1 = (h\circ\pi_1)\circ(\id_{\simplex[1]}\times (k\circ f))\\
h\circ(\id_W,h'\circ h) &= (h,h) = (h\circ\pi_1)\circ(c\times\id_W)
\end{align*}

\[\xymatrix{
X+X\ar[r]^{k\circ f+k\circ f}\ar[d]_{c\times \id} & W+W\ar[r]^(.6){(\id,h'\circ h)}\ar[d]_(.3){c\times\id} & W\ar[d]^h\\
\simplex[1]\times X\ar[r]_{\id\times (k\circ f)}\ar[urr]^(.3){k\circ f\circ\pi_1} & \simplex[1]\times W\ar[r]_(.6){h\circ \pi_1}\ar@{.>}[ur]_{\phi} & Y
}\]}

The morphism $g$ factors as a fibration $h\of W\to Z$ following an acyclic cofibration $k\of Y\to W$ by lemma \ref{factor2}. The dual of the proof of lemma \ref{shared retract 2} doesn't show that $h$ is an acyclic fibration directly, but it does show that $h$ has a right inverse $h'\of Z\to W$ and that there is a homotopy $\phi\of\simplex[1]\times W\to W$ between $\id_W$ and $h'\circ h$.

Let $a\of I\to J$ be an arbitrary cofibration and let $i\of I\to W$ and $j\of J\to Z$ satisfy $j\circ a=h\circ i$. Lemma \ref{triple lift} deforms $h'\circ j$ into a filler. Let $c_i$ be the morphisms $1\to \simplex[0]$.
\begin{align*}
\phi\circ(\id\times i)\circ (c_1\times \id) &= h'\circ h\circ i = h'\circ j\circ a \\
h\circ h' \circ j &= j = j\circ \pi_1\circ(c_1\times\id) \\
j\circ \pi_1\circ (\id\times a) &= j\circ a\circ \pi_1 = h\circ i \circ \pi_1\\
h\circ \phi\circ(\id\times i) &= h \circ \pi_1 \circ (\id\times i) = h\circ i \circ \pi_1
\end{align*}

\[\xymatrix{
I\ar[r]^{a}\ar[d]_{c_1\times \id} & J\ar[r]^(.6){h'\circ j}\ar[d]_(.3){c_1\times\id} & W\ar[d]^h\\
\simplex[1]\times I\ar[r]_{\id\times a}\ar[urr]^(.3){\phi\circ(\id\times i)} & \simplex[1]\times J\ar[r]_(.6){j\circ \pi_1}\ar@{.>}[ur]_{\psi} & Z
}\]
The filler is $\psi\circ c_0$. Since this works for arbitrary cofibrations $a$, $h$ is an acyclic fibration and $f = h\circ k$ is a weak equivalence.
\end{proof}

\end{document}