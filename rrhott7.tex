\documentclass{amsart}
\usepackage{a4}
\usepackage{amssymb, amsmath, amsthm}
\usepackage[all]{xy}
\usepackage{cite}
\usepackage{url}
\usepackage{graphicx}


\title{Realizability of Univalence\\
Modest Kan complexes}
\author[W. P. Stekelenburg]{Wouter Pieter Stekelenburg}
\address{Faculty of Mathematics, Informatics and Mechanics\\
University of Warsaw\\
Banacha 2\\
02-097 Warszawa\\
Poland}
\email{w.p.stekelenburg@gmail.com}

\theoremstyle{plain}
\newtheorem{theorem}{Theorem}
\newtheorem*{theorem*}{Theorem}
\newtheorem{lemma}[theorem]{Lemma}
\newtheorem{prop}[theorem]{Proposition}
\newtheorem{corol}[theorem]{Corollary}


\theoremstyle{definition}
\newtheorem{defin}[theorem]{Definition}
\newtheorem{remark}[theorem]{Remark}
\newtheorem{axiom}[theorem]{Axiom}
\newtheorem{example}[theorem]{Example}


\newcommand\hide[1]{}
\newcommand\cat\mathcal
\newcommand\set[1]{\left\{#1\right\}}

\newcommand\id{\mathrm{id}}
\newcommand\ri{^*}

\newcommand\Set{\mathsf{Set}}
%\newcommand\sSet{\mathsf{sSet}}
\newcommand\N{\mathbb N}
\newcommand\pow{\mathbf P}
\newcommand\Asm{\mathsf{Asm}}

\newcommand\sier{{\mathbf 2}}

\begin{document}



\begin{abstract}
A \emph{modest Kan complex} is a \emph{modest simplicial set} which has a right lifting property with respect to horn inclusions $\Lambda_k[n] \to \Delta[n]$. This paper shows that there is a univalent universe of modest Kan complexes among \emph{simplicial assemblies}.
\end{abstract}

\maketitle

\newcommand\pers{\mathbf{PER}}
\section{Introduction}
A PER (Partial Equivalence Relation) is a symmetric transitive relation of the natural numbers. A morphism of PERs $R\to S$ is a function $f$ of the equivalence classes, for which there is a partial recursive function $\phi$ such that $\phi(x)\in f(y)$ for all $x\in y$. Together they form a category $\pers$ which has a lot of interesting properties. PERs provide a semantics for the polymorphic $\lambda$-calculus. \cite{MR1099188,MR2074932,MR1003196}%\cite{MR1265066,MR1099188,MR1167031,MR2729220,MR2074932,MR1076404,MR1003196,MR1239602}

The category $\pers$ is closely related to the category of modest sets, which is a subcategory of the effective topos. \cite{MR1097022,MR1023803,MR2479466}

This paper is essentially about \emph{simplicial PERs}, i.e.\ the simplicial objects of $\pers$ and their potential use as a model of \emph{homotopy type theory}. We study these through the related category of \emph{simplicial modest sets}. 

Concretely, we show that inside the category of assemblies the \emph{category of discrete opfibrations} over a fixed base category all have their own versions of modest sets and all have a generic modest morphism (theorem \ref{genmod}). We define injective morphisms for arbitrary families of morphisms and show that they too have a generic modest morphism (theorem \ref{geninjmod}). Finally we introduce the \emph{simplicial assemblies}. The subcategory of Kan complexes is a model category (theorem \ref{modelcat}) and the domain of the generic modest Kan fibration s a Kan complex (theorem \ref{complex}). Ultimately we show that the generic modest fibration is \emph{univalent} (theorem \ref{univalence}).

\section{Assemblies}
This section provides some background on the category of assemblies and the category of modest sets. For general information about the effective topos see \cite{MR2479466}.

\begin{defin}[Assemblies] Let $\N$ be the set of natural numbers and let $\pow\N$ be its powerset.
An \emph{assembly} is a pair $(X,\phi)$ where $X$ is a set and $\phi:X\to \pow \N$ is a function which assigns a non empty set of numbers $\phi(x)$ to each element of $X$.

Let $(X,\phi)$ and $(Y,\chi)$ be assemblies. A partial recursive function $f$ \emph{tracks} $g:X\to Y$ if $f: \phi(x) \to \chi(g(x))$. A morphism $(X,\phi)\to (Y,\chi)$ is a function $g:X\to Y$ which is tracked. With composition and identities defined as in the category of sets, $\Asm$ is the category of assemblies and morphisms of assemblies.
\end{defin}

The category of assemblies has a number of useful properties which we will mention without proving them here.

\newcommand\nno{\mathbf N}
\begin{lemma} The category of assemblies\dots
\begin{itemize}
\item is finitely complete and cocomplete;
\item is locally cartesian closed, regular and extensive;
\item has a natural number object $\nno$;
\end{itemize}\label{essential_properties}%hier: werk de bad boxes weg
\end{lemma}

\begin{proof} See \cite{MR2479466,MSC:8896618,RealCats}. \end{proof}

\subsection{Internal Logic}\label{internal}
This section sketches a type theory which can be soundly interpreted in $\Asm$ thanks to the properties mentioned in lemma \ref{essential_properties}. It roughly comes down to Heyting arithmetic with dependent types. The proofs in this paper will sometimes reason inside this type theory to work out details that are difficult to express in commutative diagrams. 

\newcommand\of{\mathord{\ :\ }}
The internal language of $\Asm$ is a dependent type theory. Types are interpreted as assemblies and terms are interpreted as morphism of assemblies. The statement $x\of\tau$ declares that $x$ is of type $\tau$. If $f:S\to T$ is a morphism of assemblies and $S$ and $T$ are interpretations of types $\sigma$ and $\tau$, then $f(x)$ is a term and $x\of \sigma$ implies $f(x)\of \tau$.  

The internal language allows for various type constructions, in particular products like the \emph{terminal type} $1$, the \emph{binary product} $\tau\times\sigma$ and the \emph{arrow types} $\tau\to\sigma$ all with the usual semantics. Because $\Asm$ is locally cartesian closed, types can also come in families indexed over other types: $\set{\tau_i|i\of \sigma}$ and these families have indexed products $\prod_{i\of \sigma} \tau_i$ and coproducts $\coprod_{i\of \sigma} \tau_i$. Finally, systems of equations define subtypes of any type. \[\set{x\of\tau| f(x)=g(x)\land h(x)=k(x)} \] Subtypes correspond to subobjects.
%citeer iets?

Because $\Asm$ is extensive, types also have various coproducts, like the \emph{initial type} 0, the \emph{binary coproduct} $\tau+\sigma$. Regularity gives existential quantification over subtypes: if $\alpha$ is a subtype of $\sigma\times\tau$, then $\exists x\of\sigma.\alpha$ is a subtype of $\tau$. By combining existential quantification with the other type constructors, we get \emph{finitary disjunctions} $\bot$, $\alpha\vee \beta$, \emph{implications} $\alpha\to \beta$ and \emph{universal quantification} $\forall x\of\sigma.\alpha$ of subtypes. In fact, $\Asm$ is a Heyting category, i.e.\ it provides a sound interpretation of first order constructive logic if each subtype declarations $x\of \alpha$ is treated as a sentence $\alpha(x)$.

The natural number object $\nno$ makes $\Asm$ a model of Heyting arithmetic. 

In summary, $\Asm$ has a rich internal language containing Heyting arithmetic and a constructive version of the theory of small categories. In this paper we often reason within $\Asm$ rather than about $\Asm$ in a metalanguage, in order to demonstrate the relevant results.

\subsection{Modest sets}
The category of modest sets is a subcategory of the category of assemblies, which is complete in a suitable internalized sense and equivalent to an internal category of the category of assemblies.

\newcommand\bang{!}
\begin{defin}[Modesty]
Let $\nabla 2$ be the assembly $(\set{0,1}, i \mapsto \N)$. A morphism $f:X\to Y$ of assemblies is modest of the following diagram is a pullback:
\[\xymatrix{
X \ar[r]^{\id^\bang} \ar[d]_{f} & X^{\nabla 2} \ar[d]^{f^{\nabla 2}}\\
Y \ar[r]_{\id^\bang} & Y^{\nabla 2}
}\]
Here $\id^\bang$ stands for composition with the unique maps $\bang_X:X\to 1$ and $\bang_Y: Y\to 1$ to the terminal object. This is another way of saying that $f$ is \emph{right orthogonal} to $\bang:\nabla 2\to 1$ and to the multiple $W\times \bang$ for every assembly $W$. A \emph{modest set} is an assembly $X$ for which $\bang:X\to 1$ is modest (this means that $\id^\bang$ is an isomorphism).
\end{defin}

\begin{lemma} Modest morphisms\dots
\begin{itemize}
\item are closed under composition, pullbacks and products, including dependent products; 
\item include all monomorphisms and the unique map $\bang_\nno:\nno \to 1$;%all gedefinieerd?
%\item are also closed under all colimits; kom hier aan het einde op terug
\item are pullbacks of a single \emph{generic modest morphism} $\mu$.
\end{itemize}\end{lemma}

\begin{proof} See \cite{MR1023803,MR2479466,MR1097022}. \end{proof}

The \emph{generic modest morphism} $\mu:E\to B$ induces an internal category $\pers$. The object of objects of $\pers$ is $B$. The object of morphisms is the indexed coproduct $\coprod_{(i,j)\of B\times B} (E_i\to E_j)$ (as expressed in the internal language of $\Asm$). Since it corresponds to modest sets, it is a complete internal category. Contrary to complete internal categories of $\Set$, which are posets by a theorem of Freyd, $\pers$ is not a poset. \hide{cite Freyd!?}

The global sections functor $\Gamma:\Asm \to\Set$ turns $\pers$ into the category with subquotients of $\N$ as objects and tracked functions as morphisms which we described in the introduction of this paper.

\newcommand\simcat{{\mathord{\triangle}}}%{\mathsf{fo}}
%\newcommand\Simcat{{\mathord{\vartriangle}}}%{\mathsf{FO}}
\newcommand\dual{^{op}}
\section{Modest opfibrations}
This section introduces discrete opfibrations, which act like functors from internal categories to $\Asm$. We construct a generic modest morphism for in the category of opfibrations over an arbitrary base category.

\subsection{Discrete opfibrations}
In order to mimic assembly-valued functors $\cat B \to \Asm$ we use a kind of functor $\cat E\to \cat B$ with the property that the fibres are discrete categories and that a morphism $f:i\to j$ in $\cat B$ induces a functor $f':\cat E_i\to\cat E_j$ between the fibres. Both of these properties come from the following.

\newcommand\cod{\mathrm{cod}}
\newcommand\dom{\mathrm{dom}}
\newcommand\catasm{\mathsf{cat(Asm)}}
\newcommand\tot{\mathord{\int}}
\newcommand\rat{\mathbf{s}}
A functor $F:\cat E\to\cat B$ is a \emph{discrete opfibration} if the following square is pullback.
\[\xymatrix{
\cat E^\sier \ar[r]^\dom \ar[d]_{F^\sier} & \cat E\ar[d]^F\\
\cat B^\sier \ar[r]_\dom & \cat E
}\]
Here $\sier$ is the category with two objects $0,1$ and one non identity arrow $0\to 1$, so $\cat B^\sier$ and $\cat B^\sier$ are the arrow categories. In both cases $\dom$ is the projection of the arrows to their domains. In other words, discrete opfibration are right orthogonal to the functors $c\mapsto(c,0):\cat C\to \cat C\times\sier$.

The category $\cat B$ acts on $\cat E$ in the following way. For each object $e$ of $\cat E$ and each morphism $\phi:F(e)\to b$ in $\cat B$ there is a unique morphism $\phi_e:e\to \phi\cdot e$ such that $F(\phi_e) = \phi$. 

\begin{lemma} If $G:\cat B\to \cat C$ is a discrete opfibration, then $F:\cat A\to \cat B$ is the discrete opfibration if and only of $GF$ is. Moreover, discrete opfibrations are stable under pullback. \label{disc1} \end{lemma}

\begin{proof} This holds for any class of right orthogonal morphisms, for straightforward reasons. \end{proof}

For each internal category $\cat B$ of $\Asm$, $\Asm^{\cat B}$ is the category whose objects are discrete opfibrations with codomain $\cat B$ and whose morphisms are commutative triangles. 

We now enter the internal language of $\Asm$ (see subsection \ref{internal}) to demonstrate that $\Asm^{\cat B}$ indeed functions as a category of presheaves.

\begin{lemma} Let $F/\cat D$ be the category that has morphisms $f:Fc\to d$ as objects and where a morphism $f\to f'$ is a pair $(g,g')$ where $g$ is a morphism of $\cat C$, $g'$ of $\cat D$ and $Fg'\circ f=f'\circ Fg$. For each functor $F:\cat C\to \cat D$ opfibrations are orthogonal to the functor $I(F):\cat C \to F/\cat D$ which sends $c$ to $F\id_c$. \label{Yoneda} \end{lemma}

\begin{proof} Let $G:\cat D\to\cat E$ be an opfibration and let $D:\cat C\to \cat \cat D$ and $E:F/\cat D \to \cat E$ satisfy $GD=EI$. Define $H:F/\cat D\to \cat C$ by $H(f:Fc\to d) = E(f,\id_{d})\cdot D(c)$ for objects of $\cat B^\sier$. For $(g,g'):(f:Fc\to d) \to (f':Fc'\to d')$ let $H(g,g') = E(g,g')_{H(f)}$. The functor $H$ satisfies $HI(b) = G(b)$ and $GH(f) = G(E(f,\id_{Fc})\cdot D(c))$.
\end{proof}

\begin{lemma} The category $\Asm^{\cat B}$ has all finite limits and colimits and is locally cartesian closed and regular. \end{lemma}

\begin{proof} Finite limits is trivial with lemma \ref{disc1}. The functor $\cat C \mapsto \cat C^\sier$ preserves all coproducts because the category $\set\to$ is connected. If a discrete opfibration $F:\cat C\to\cat D$ is a regular epimorphism of objects, then so is $F^\sier$, which explains coequalizers and regularity. Since every slice category of $\Asm^{\cat B}$ is another category of the form $\Asm^{\cat C}$ it suffices to show cartesian closure. 

Let $F:\cat C\to\cat B$ and $G:\cat D\to\cat B$ be opfibrations. The opfibration $G^F:\cat D^{\cat C}_{\cat B}$ is defined as follows. The objects of $\cat D^{\cat C}_{\cat B}$ are pairs $(b,H)$ where $b$ is an object of $\cat B$ and $H:b/\cat B \times_{\cat B} \cat C\to \cat D$ is a functor that commutes with $\cod\times F:b/\cat B \times_{\cat B} \cat C \to \cat B$ and $G:\cat D \to \cat B$. A morphism $(b,H) \to (b', H')$ is a morphism $f:b\to b'$ such that $H(x\circ f,Y) = H'(x,Y)$. This is an opfibration because for each object $(b,H)$ and each $f:b\to b'$, $(b,H\circ f\ri)$, where $f\ri$ is composition with $f$ to the right, is the unique lifting.

Let $E:\cat E\to\cat B$ be another fibration. The opfibration $G:\cat D\to\cat B$ is orthogonal to $\cat E\times_{\cat B}\cat C \to E/\cat B\times_{\cat B} \cat C$ by lemma \ref{Yoneda} and there is a bijection between functors $E/\cat B\times_{\cat B} \cat C \to\cat D$ which commute with $\cod\times F$ and $G$ and functors $\cat E \to \cat D^{\cat C}_{\cat B}$ which commute with $E$ and $G^F$ by the definition given above. Note that this also works when $F$ is not an opfibration.
\end{proof}

\subsection{Total category}
Let $\catasm$ be the category of internal categories and functors in $\Asm$. There is an obvious functor $\tot:\Asm^{\cat B} \to \catasm$ that sends a discrete opfibration $F:\cat E\to\cat B$ to its domain: the \emph{category of elements} of the discrete opfibration. We remain in the internal language of $\Asm$ for now (see subsection \ref{internal}).

\begin{lemma} The category-of-elements functor $\tot$ has a right adjoint. \end{lemma}

\begin{proof} For an arbitrary internal category $\cat C$ let $|\rat\cat C|$ be the following category. The objects are pairs $(i,F)$ where $i$ is an object of $\cat B$ and $F$ is a functor $i/\cat B \to \cat C$. A morphism $(i,F) \to (j,G)$ is a morphism $\phi:i\to j$ such that $F\circ \phi\ri = G$. Here $\phi\ri$ means composition with $\phi$ on the right. The opfibration $\rat\cat C:|\rat\cat C|\to\cat B$ is the projection to the first variable. %zo moet het toch wel?

For an arbitrary functor $H:\cat C \to \cat D$, let $\rat H$ be the functor that satisfies $\rat H(i,F) = (i,HF)$ on objects and $\rat H(\phi,f)=(\phi,Hf)$ on morphisms.

Let $D(\cat C):\rat\cat C \to\cat B$ be the functor which sends $(i,F)$ to $i$ and which is the identity on morphisms. This is a discrete opfibration because for each object $(i,F)$ of $\rat\cat C$ and each morphism $\phi:P(i,F) \to j$ there is a unique morphism $\phi:(i,F) \to (j,F\circ \phi\ri)$ over $\phi$.

This functor $\rat$ has the property that there is a bijection between functors $\cat C\to|\rat\cat D|$ which commute with $F:\cat C\to\cat B$ and $\rat \cat D$ and functors $F/\cat B\to \cat D$, because of the definition of $\rat$. We can compose $G:\cat C\to\cat D$ with $\dom:F/\cat C \to \cat C$ to get the commutative triangle $F\to \rat \cat D$. For $H:f\to\rat \cat D$ we compose the corresponding $H':F/\cat B \to \cat D$ with the functor $I:\cat C\to F/\cat B$ of lemma \ref{Yoneda}.

Hence we get an adjunction $\tot\dashv\rat$.
\end{proof}

\subsection{Modest opfibrations}
We bring modest morphisms to $\Asm^{\cat B}$ by considering morphisms of discrete opfibrations whose underlying functors are modest.

\begin{defin} A discrete opfibration $F:X\to Y$ is \emph{modest} if its object map $F_0:X_0\to Y_0$ is a modest morphism. \end{defin}

\begin{lemma} There is a generic modest opfibration. \end{lemma}

\begin{proof} Let $\mu:E\to B$ be the generic modest morphism of $\Asm$. The category of $\pers$ was defined with $B$ as object of objects and with $\pers(i,j) = E_j^{E_i}$ as homsets. We reenter the internal language of $\Asm$ to define the internal category of pointed PERs $\pers_*$ in a similar way. Its object of objects is $E$ and $\pers_*(i,j) = \set{f\of E_{\mu(j)}^{E_\mu(i)}| f(i)=j}$; the idea is that $E$ is a set of PERs $E_i$ paired with a chosen point $1\to E_i$ and that morphisms preserve points.

There is a forgetful functor $U:\pers_* \to \pers$; for objects $U(i) = \mu(i)$ and for morphism $U$ is the identity. This is a discrete opfibration because for each morphism $f:E_i \to E_j$ and each point $p:1\to E_i$, $f(p):1\to E_j$ is he unique point that turns $f$ into a morphism of pointed PERs. We claim that this is the generic modest opfibration.

Let $F:\cat C\to \cat D$ be a modest opfibration. There are morphisms $e:\cat C_0\to E$ and $b:\cat D_0\to B$ such that $b\circ g = \mu\circ e$ is a pullback square. We turn these into functors in the following way. In $\catasm/\cat D$  the action $x\mapsto f\cdot x$ for $f\of \cat D_1$ induces a morphism $\dom(\cat D) \to F^F$ where $\dom(\cat D):\cat D^{\to}\to\cat D$ is the domain functor and $F^F$ is the exponential as fibration. Because $F^F \simeq b\ri(U^U)$ we get an object map which turns $b:\cat D_0\to B$ into a functor $\cat D\to\pers$.

In $\catasm/\cat C$ for the action $x\mapsto f\cdot x$ for $f\of \cat C_1$ induces a morphism $F\dom(\cat C) \to F^F$. This time we rely on $F\ri(F^F) \simeq (b\circ F)\ri(U^U) = (\mu\circ e)\ri(U^U)$ to turn the morphism $e:\cat C_0\to E$ into a functor $e:\cat C\to \pers_*$.


The new functors satisfy $U\circ e = b\circ F$ and even form a pullback square. For each $f:i\to j$ in $\cat B$ and each $g:E_{\mu(k)} \to E_{\mu(l)}$ in $\pers_*$ such that $b(f) = U(g)$, there is an object $k'$ of $Y_i$ such that $F(k') = i$ and $e(k') = k$ because the object maps for a pullback square. Hence there is a unique $g':k'\to l'$ in $\cat B$ such that $Fg' = f$. Because $U(e(g')) = b(f)$, $e(g')$ is the unique morphism $g:k\to l$ for which $U(g)=b(f)$. Hence the square $U\circ e = b\circ F$ is a pullback.

The discrete opfibration $U:\pers_*\to\pers$ is clearly modest itself and hence a generic modest morphism.
\end{proof}

A morphism $F:\cat C \to\cat B$ in $\Asm^\cat B$ is \emph{modest} precisely when $\tot(F)$ is modest. This has the following consequence.

\begin{theorem} The category $\Asm^\cat B$ has a generic modest morphism. \label{genmod} \end{theorem}

\begin{proof} The morphism $\rat U:\rat\pers_* \to\rat\pers$ is the generic modest morphism of $\Asm^{\cat B}$. If $\tot(F):\tot(\cat E) \to \tot(\cat B)$ is a pullback of $U:\pers_*\to\pers$ along some $X:\tot(\cat B) \to \cat \pers$, then $F$ is the pullback of $\rat U$ along the transpose $X^t$ of $X$ for the following reasons.
there is
Discrete opfibrations are orthogonal to coslice categories by lemma \ref{Yoneda}. This means that for every point $p\of \pers_*$ and every functor $F:i/\cat B\to\pers$ such that $U(p) = F(\id_i)$, there is a unique functor $F_*:i/\cat B\to\pers_*$ such that $UF_* = F$ and $F_*(\id_i) = p$.

The co-unit $\epsilon_{\pers}: \rat \pers \to \pers$ is the functor that sends $(i,F)$ in $\rat \pers$ to $F(\id_i)$ and a morphism $f:(i,F) \to (j,G)$ to $Ff:F(\id_i) \to F(f)=G(\id_j)$. The description for $\epsilon_{\pers_*}$ is the same. If $(i,F)$ is an object of $\rat \pers$ and $j$ an object of $\pers_*$ such that $Uj = \epsilon(i,F) = F(\id_i)$, we get a unique functor $F_*$ such that $\epsilon(i,F_*) = j$ and $UF_* = F$ as explained above. Hence the naturalness square of the co-unit is a pullback. This implies that $\rat U$ is modest and that if $\tot(F)$ is a pullback of $U$, then $F$ is a pullback of $\rat U$.
\end{proof}

\subsection{Orthogonality and completeness}
Just like in $\Asm$, orthogonality characterizes the modest morphisms of $\Asm^{\cat B}$.

\newcommand\disc{_{\rm disc}}
\begin{lemma} Let $\nabla 2\disc$ be the discrete category whose object of objects is $\nabla 2$. In $\catasm$ the modest morphisms are precisely those that are right orthogonal to the constant discrete opfibrations $\nabla 2\disc \times\cat B \to\cat B$. \end{lemma}

\hide{\tot preserves discrete opfibrations. Can't we use that?}
\begin{proof} A discrete opfibration $F:\cat C\to\cat D$ is modest if the underlying object map $F_0:\cat C_0 \to \cat D_0$ is modest. Due to the adjunction between $X\mapsto X\disc$ and $Y\mapsto Y_0$, $\tot F_0$ is right orthogonal to $\nabla 2$ when $\tot F$ is orthogonal to $\nabla 2\disc$. The constant discrete opfibration is just a multiple and therefore right orthogonality is preserved.
\end{proof}


\begin{prop} For each internal category $\cat B$ of $\Asm$, $\Asm^{\cat B}$ has small complete internal categories.\end{prop}

\begin{proof} The class of discrete opfibrations which are right orthogonal to $\nabla 2\disc$, are closed under all existing limits and the fibration is essentially small thanks to the generic modest morphism. \end{proof}

\newcommand\sAsm{\mathsf{sAsm}}

\section{Injectives}
The categories $\Asm^{\cat B}$ are canonically enriched over $\Asm$. Enrichment modifies the lifting properties which ordinarily define Kan fibrations in simplicial sets. This section shows that for any internal family of morphisms in $\Asm^{\cat B}$, there is a generic modest injective morphism (theorem \ref{geninjmod}).

\subsection{Enriched injectives}

\newcommand\pb{\ar@{}[dr]|<\lrcorner}
\newcommand\po{\ar@{}[dr]|>\ulcorner}
\newcommand\nat{\mathrm{nat}}
\begin{defin} Let $\nat:(\Asm^{\cat P})\dual\times \Asm^{\cat P} \to \Asm$ be the functor which sends each pair of opfibration $X,Y$ over $\cat P$ to the assembly $\nat(X,Y)$ of morphisms between them.

A morphism $f:X\to Y$ has the \emph{global right lifting property} with respect to a a morphism $g:I\to J$--and $g$ has the global \emph{left} lifting property with respect to $f$--if the morphism $(f_!,g\ri) = (\nat(\id_J,f),\nat(g,\id_X))$ in the diagram below is a \emph{split} epimorphism.
\[\xymatrix{ 
\nat(J,X) \ar[drr]^{\nat(g,\id_X)} \ar[ddr]_{\nat(\id_J,f)} \ar[dr]|(.6){(f_!,g\ri)}\\
& \bullet \ar[d]\ar[r]\pb & \nat(I,X) \ar[d]^{\nat(\id_I,f)}\\
& \nat(J,Y) \ar[r]_{\nat(g,\id_Y)} & \nat(I,Y)
}\]
\end{defin} 

\begin{remark}[Lifting power] The ordinary right lifting property only says that composition with $(f_!,g\ri)$ induces a surjective function on global sections, while the \emph{local} right lifting property only requires that $(f_!,g\ri)$ is a regular epimorphism. The global version is stronger than either of those. \label{liftingpower} \end{remark}

\begin{example} Modest sets have the global right lifting property with respect to $\bang:\nabla 2\to 1$. \end{example}

\begin{example} Discrete opfibrations have the global right lifting property with respect to $0:1\to \set\to$, though they are not the only functors that have it. \end{example}

\begin{defin}[Injective and anodyne] An $I$-indexed-family of morphisms in $\Asm^{\cat B}$ a morphism $a:D\to E$ in $\Asm^{I\disc\times\cat B}$. Let $I\ri:\Asm^{\cat B}\times \Asm^{I\disc\times \cat B}$ be the functor which sends each opfibration $F:\cat E\to \cat B$ to the opfibration $I\disc\times F:I\disc\times\cat E\to I\disc\times\cat B$. A morphism $f:X\to Y$ is \emph{$a$-injective} if $I\ri f$ has the global right lifting property with respect to $a$. A morphism is \emph{$a$-anodyne} if it has the global left lifting property with respect to all $a$-injectives. \end{defin}


\subsection{Injectives as algebras}\label{IAA}
Injectives are a kind of Lambek algebra for a functor $S:\Asm^{\cat B}\to \Asm^{\cat B}$. This allows us to construct a generic modest injective in $\Asm^{\cat B}$.

\begin{defin}[Algebras] A \emph{pointed endofunctor} is a pair $(F,\phi)$ where $F$ is an endofunctor of a category $\cat C$ and $\phi:\id_{\cat C}\to F$ is a natural transformation. An \emph{algebra} for $(F,\phi)$ is a pair $(X,f:FX\to X)$ where $f\circ \phi_X = \id_X$. \end{defin}

\begin{lemma} There is a pointed endofunctor $(S:\Asm^{\cat B\times \to}\to\Asm^{\cat B\times \to}, \sigma)$ such that $f:X\to Y$ is $a$-injective if it has an algebra structure. \end{lemma}

\begin{proof}
For each morphism $f:X\to Y$ in $\Asm^{\cat B}$ let $\nat(a,I\ri (f))$ be the following object of $\Asm^{I\disc\times \cat B}$. In the internal language (see subsection \ref{internal} it is the following subtype.
\[ \set{(x,y)\of \nat(D,I\ri(X))\times \nat(E,I\ri(Y))| f\circ x = y\circ h } \]
Alternatively, the following pullback square defines it. 
\[\xymatrix{
\nat(a,I\ri(f)) \ar[r]\ar[d]\pb & \nat(D,I\ri(X)) \ar[d]^{\nat(D,I\ri(f))}\\
\nat(E,I\ri(Y)) \ar[r]_{\nat(a,I\ri(Y))} & \nat(D,I\ri(Y))
}\]

The functor $\nat(E,I^*(-))$ is a composition of three functors which have a left adjoint. The first one $I^*$ has $I_!:\Asm^{I\disc\times \cat B}\to \Asm^{\cat B}$ which is composition with the opfibration $I\disc\times \cat B\to \cat B$. The functor $\nat(E,-)$ is the composition of the exponential $-^E$ and the functor $\cat B_*:\Asm^{I\disc}\to\Asm^{I\disc\times \cat B}$ which is right adjoint to the functor $\cat B\ri:\Asm^{I\disc}\to\Asm^{I\disc\times \cat B}$ which sends each opfibration $F:\cat E\to I\disc$ to $F\times\cat B:\cat E\times\cat B\to I\disc\times\cat B$. Hence $I_!(\cat B^*(-)\times E)$ is left adjoint to $\nat(E,I^*(-))$.

The transposes of the projections $\pi_0: \nat(a,I\ri (f))\to \nat(D,I\ri(X))$ and $\pi_1:\nat(a,I\ri (f))\to \nat(E,I\ri(Y))$ satisfy $f\circ \pi_0^t = \pi_1^t\circ I_!(\cat B^*(\nat(a,I\ri (f)))\times f)$. The pointed endofunctor comes from the pushout of $I_!(\cat B^*(\nat(a,I\ri (f)))\times f)$ along $\pi_0^t$, as defined below.
\[ \xymatrix{
I_!(\cat B^*(\nat(a,I\ri (f)))\times D) \ar[r]^(.7){\pi_0^t} \ar[d]_{I_!(\cat B^*(\nat(a,I\ri (f)))\times f)}\po &
X \ar[r]^{\id_X}\ar[d]_{\sigma(f)} & X \ar[d]^f \\
I_!(\cat B^*(\nat(a,I\ri (f)))\times E) \ar[r] \ar@/_2ex/[rr]_{\pi_1^t} & S(X) \ar[r]^{S(f)} &  Y 
}\]

By definition $f$ is $a$-injective if the canonical morphism $(a^*,f_!):\nat(E,I\ri(X))\to \nat(a,I\ri(f))$ has a section $g$. The transpose $g^t$ satisfies $f\circ g^t = \pi_1^t$ and $g^t\circ I_!(\cat B^*(\nat(a,I\ri (f)))\times f) = \pi_0^t$ and hence factors though $S(X)$ giving an algebra structure to $f$. On the other hand each algebra structure on $f$ induces an inverse of $(a^*,f_!)$.
\end{proof}

\begin{remark} Nothing forces $S(f)$ to be injective, so the construction above does not necessarily induce a factorization into anodyne and injective morphisms. Small object arguments don't help because $\Asm^{\cat B}$ is not cocomplete. The internal category $\rat\pers$ is \emph{algebraically complete}\cite{Freyd91}. Therefore $(S,\sigma)$ generates a free monoid in the monoidal category $\rat\pers^{\rat\pers}$, which is a monad on $\rat\pers$ whose algebras are $(S,\sigma)$-algebras. This means that modest morphisms factorize into modest injective and morphisms which have the global left lifting property with respect to modest injectives. This is not exactly the same thing as anodyne, unfortunately. \label{modfac}
\end{remark}

\subsection{Generic modest injectives} We delve deeper into modest injectives.

\begin{theorem} Every family of morphisms $a$ has a generic modest $a$-injective. \label{geninjmod} \end{theorem}

\begin{proof} The functor $S$ is defined by a combination of limits and colimits. Pullbacks preserve all of them because $\Asm^{\cat B}$ is locally cartesian closed. The factorization with $S$ above is therefore stable under pullback, i.e.\ if $f: A\to B$ is the pullback of $g:C\to D$ along some $h:B\to D$, then $S(f):S(A)\to B$ is the pullback of $S(g):S(C) \to D$.
\[\xymatrix{
A\ar[r]_{\sigma(f)} \ar@/^2ex/[rr]^{f}\ar[d]\pb & S(A) \ar[r]_{S(f)}\ar[d]\pb & B\ar[d]^h\\
C\ar[r]^{\sigma(g)} \ar@/_2ex/[rr]_{g} & S(C) \ar[r]^{S(g)} & D
}\]

Reentering the internal language of $\Asm$ (see subsection \ref{internal}) let $\rat\pers^S$ be the following object of fibrewise left inverses of $\sigma(\rat U):\rat\pers_* \to S(\rat\pers_*)$ over $\rat\pers$:
\[ \rat\pers^S_i = \set{ f:S(\rat\pers_*)_i\to (\rat\pers_*)_i| f\circ\sigma(\rat U)_i = \id_{(\rat\pers_*)_i}} \] 
There is a morphism $\rat U^{S(\rat\pers_*)}:\rat\pers^S\to \rat\pers$ which simply sends each left inverse to its index.
When we pull back $\rat U$ along $\rat U^{S(\rat U)}$ we obtain a modest morphism $\rat U^S:\rat\pers_*^S \to \rat\pers^S$.

First we show that $\rat U^S$ is a modest $a$-injective.
The object $S(\rat\pers_*^S)$ is the fibred product $\rat\pers^S\times_{\rat\pers}S(\rat\pers_*)$ and by definition of $\rat\pers^S$ there is an application morphism $\epsilon:S(\rat\pers_*^S)\to \rat\pers_*$ which makes $\epsilon\circ \sigma(\rat U^S)$ equal to the projection $\pi:\rat\pers_*^S\to\rat\pers_*$.
\[\xymatrix{
\rat\pers_*^S\ar[r]_{\sigma(\rat U^S)} \ar@/^2ex/[rr]^{\rat U^S}\ar[d]_\pi\pb & S(\rat\pers_*^S) \ar[r]_{S(\rat U^S)}\hide{\ar[d]_{(}^{)}\pb}\ar[dl]_\epsilon & \rat\pers^S\ar[d]^{\rat U^{S(\rat U)}}\\
\rat\pers_*\ar[r]^{\sigma(\rat U)} \ar@/_2ex/[rr]_{\rat U} & S(\rat\pers_*) \ar[r]^{S(\rat U)} & \rat\pers
}\]
\hide{Warning: in this diagram $\sigma(\rat U)\circ\epsilon$ is not equal to the projection $S(\rat\pers_*^S) \to S(\rat\pers_*)$, hence the parentheses in the diagram above. }Because $\rat U\circ \epsilon = \rat U^{S(\rat U)}\circ S(\rat U^S)$, $\epsilon$ factors through $\rat\pers^S$ in a unique $\epsilon':S(\rat\pers_*^S)\to\rat\pers_*^S$ which satisfies $\rat U^S\circ\epsilon' = S(\rat U^S)$ and $\pi\circ \epsilon'= \epsilon$. Because $\rat U^S\circ\epsilon'\circ \sigma(\rat U^S) = \rat U^S$ and $\pi\circ \epsilon'\circ \sigma(\rat U^S) = \pi$, $\epsilon'\circ \sigma(\rat U^S)$ is the unique factorization of $(\rat U^S,\pi)$ through itself, i.e.\ $\id_{\rat\pers_*^S}$. Hence $\rat U^S$ is a modest $a$-injective.

Next we show that every modest $a$-injective is a pullback of $\rat U^S$.
Let $f:X\to Y$ be the pullback of $\rat U$ along $\chi:Y\to \rat \pers$. Let $g:S(X) \to X$ satisfy $f\circ g = S(f)$ and $g\circ \sigma(f) = \id_X$. The morphism $S(f):S(X)\to Y$ is the pullback of $S(\rat U):S(\rat\pers_*) \to \rat\pers$. We take advantage of fibrewise isomorphisms $\alpha_y:X_y \to (\rat\pers_*)_{\chi(y)}$ and $\beta_y:S(X)_y\to S(\rat\pers_*)_{\chi(y)}$ set up by these pullbacks. The morphism $\chi$ factors through $\rat\pers^S$ thanks to the following mapping: 
\[ y\of Y  \mapsto \alpha_y\circ g_y\circ \beta_y^{-1}\quad:\quad S(\rat\pers_*)_{\chi(y)}\to (\rat\pers_*)_{\chi(y)} \]

\hide{generalized elements}
The following is a diagram chasing way to find that morphism.
Work inside $\Asm^{\cat B}/Y$, where we have actual isomorphisms $\alpha':f\to \chi\ri(\rat U)$ and $\beta':S(f)\to\chi\ri(S(\rat U))$. 
The morphism $\alpha'\circ g\circ (\beta')^{-1}:\chi\ri(S(\rat U)) \to \chi\ri(\rat U)$ has a transpose $\gamma:1\to \chi\ri(\rat U)^{\chi\ri(S(\rat U))}$. 
Local cartesian closure implies that the canonical morphism $\kappa:\chi\ri(\rat U^{S(\rat U)})\to \chi\ri(\rat U)^{\chi\ri(S(\rat U))}$ is an isomorphism.
The transpose $(\kappa^{-1}\circ\gamma)^t:\chi\to \rat U^{S(\rat U)}$ in $\Asm^{\cat B}/\rat\pers$ is what we are looking for.

Note that $h\circ \sigma(f) = \id_X$. For this reason $(\kappa^{-1}\circ\gamma)^t$ actually goes to $\rat\pers^S$, which only contains the right inverses of sections of $\sigma(\rat U)$.

\hide{Kunnen we vast aantonen met de juiste pullbacks, maar misschien gaat dat te ver.}

Because $\rat U^{S(\rat U)}\circ (\kappa^{-1}\circ\gamma)^t = \chi$ and $f\simeq \chi\ri(\rat U)$, $f\simeq [(\kappa^{-1}\circ\gamma)^t]\ri(\rat U^S)$. Since every modest $a$-injective is a pullback of $\rat U^S$, $\rat U^S$ is a generic modest $a$-injective.
\end{proof}

\section{Homotopy}
This section starts the exploration of \emph{simplicial assemblies} and their homotopy. If we extend the cofibrations and acyclic Kan fibrations of simplicial sets to simplicial assemblies we still get a factorization system (proposition \ref{factor1}). Not all monomorphisms in $\sAsm$ are cofibrations because the internal logic of $\Asm$ does not satisfy the principle of the excluded middle. Cofibrations are monomorphisms $f:X\to Y$ for which some properties of simplices in $Y$ are nonetheless decidable (proposition \ref{charcof}).

\subsection{Simplicial assemblies}
The simplex category exists as internal category $\simcat$ of $\Asm$. The \emph{category of simplicial assemblies} $\sAsm$ is the category of opfibrations $\Asm^{\simcat\dual}$ over the dual $\simcat\dual$ of $\simcat$.

The category of assemblies has a natural number object $\nno$. The simplex category $\simcat$ has $\nno$ as object of objects, but when we refer to its objects, we surround numbers by square brackets: $[n]$. The homset $\simcat([m],[n])$ is the object of nondecreasing morphisms $\set{i|i\leq m}\to\set{j\leq m}$. Among the morphisms of $\simcat$ the \emph{face maps} are regularly used below. The morphism $\delta^n_i:[n-1]\to [n]$ is the unique injective nondecreasing morphism which skips $i$: it represents the \emph{face opposite to the edge $i$}.

Of course, there is another kind of simplicial assembly--a presheaf on the external simplex category. Those are more general, because in $\sAsm$ there is a recursive function tracking the action map $(f,x)\mapsto f\cdot x$, while for the presheaves there may only be recursive functions tracking $x\mapsto f\cdot x$ for each $f$ separately.

The simplex $\Delta[n]$ is the opfibration $\simcat/[n]\dual\to\simcat\dual$. For each morphism $\phi:[m]\to[n]$ we let $\Delta(\phi):\Delta[m] \to \Delta[n]$ be the morphism that sends $\xi:[i]\to [m]$ to $\phi\circ \xi:[i]\to [n]$. The cycle $\partial\Delta[n]$ is the subopfibration $C[n]\subseteq \simcat/[n]\dual$ whose objects are the nondecreasing maps $[m]\to[n]$ that are not surjective. For $n>0$ the horn $\Lambda_k[n]$ is the subopfibration $H_k[n]\subseteq\simcat/[n]\dual$ whose objects are the nondecreasing maps $[m]\to[n]$ that are not surjective on the complement of $\set k\subseteq[n]$. These seemingly classical definitions work because equality of numbers is recursively decidable.

These simplicial assemblies form a family $\set{\Lambda_k[n]\to\Delta[n]|n>0,k\leq n}$ of \emph{horn inclusions} and $\set{\partial\Delta[n]\to\Delta[n]|n\of \nno}$ of \emph{cycle inclusions}. We introduce the following terms based on these families.
\begin{itemize}
\item A \emph{Kan fibration} is a injective morphism relative to the family of horn inclusions.
\item An \emph{acyclic Kan fibration} is a injective morphism relative to the family of cycle inclusions.
\item A \emph{Kan complex} or \emph{Kan fibrant object} is a simplicial assembly $X$ for which $\bang_X:X\to 1$ is a fibration.
\item A \emph{cofibration} is an anodyne morphism relative to the family of cycle inclusions.
\item A \emph{cofibrant object} is a simplicial assembly $X$ for which $\bang_X:0\to X$ is a fibration.
\item An \emph{acyclic cofibration} is a injective morphism relative to the family of horn inclusions.
\item A \emph{weak equivalence} is a morphism which factors as an acyclic fibration following an acyclic cofibration.
\end{itemize}
We often leave out `Kan' and simply talk about fibrations and complexes in the rest of this paper.

For every pair of morphisms $f,g:X\to Y$ a \emph{homotopy} between them is a map $h:\Delta[1]\times X\to Y$ such that $h\circ(\delta^1_0\times\id_X)=f$ and $h\circ(\delta^1_1\times\id_X)=g$ where $\delta^1_i$ are the face maps mentioned above. The morphisms $f$ and $g$ are \emph{homotopic} if there is a homotopy between them. A morphism $f:X\to Y$ is a homotopy inverse of $g:Y\to X$ if $f\circ g$ is homotopic to $\id_Y$ and $g\circ f$ is homotopic to $\id_X$. If $f$ has a homotopy inverse, then $f$ is a homotopy equivalence.

\subsection{Decidability} Not all monomorphisms of simplicial assemblies are cofibrations apparently because not all subobjects in the category of assemblies have complements. The next few subsections explain the connection between decidability and cofibrancy. 

\begin{defin}[Locally decidable] A monomorphism in $\Asm$ or $\sAsm$ is \emph{decidable} if it is isomorphic to a coproduct inclusion. A monomorphism $m:X\to Y$ is locally decidable if the object map of $\tot m:\tot X\to \tot Y$ is decidable in $\Asm$. \end{defin}

The terminology is slightly misleading. Decidable monomorphisms in $\sAsm$ cover all locally decidable monomorphisms, but they also cover some monomorphisms which are not locally decidable. Both $\Asm$ and $\sAsm$ have \emph{inductive} subtypes which are connected to the latter type of monomorphism, but not necessarily to the former. %cite hott book

\begin{example} All cycle inclusions $\partial\Delta[n]\to\Delta[n]$ are locally decidable, because equality of numbers is decidable. \end{example}

Not all locally decidable monomorphisms are cofibrations, but many are. The distinction is degeneracy.

\begin{defin} Let $X$ be a simplicial assembly. A simplex $x$ of $X$ is degenerate if there is an epimorphism $e$ in $\simcat$ and a simplex $y$ in $X$ such that $x=e\cdot y$. A \emph{face} is a nondegenerate simplex. \end{defin}

\begin{lemma} Let $f:X\to Y$ be a monomorphism in $\sAsm$. If $f$ is locally decidable and degeneracy is decidable for simplices in the complement of $X$, then $f$ is a cofibration. \label{Reedy} \end{lemma}

\begin{proof} Let $Y_j$ be the union of $X$ with all $j$-dimensional faces of $Y$. There is an assembly $S_j$ of $j$-dimensional faces of $Y$ which are not simplices of $X$ thanks to decidability. Let $\simcat\ri:\Asm\to\sAsm$ be the constant simplicial assembly functor. The inclusion $X\to Y_0 = X+\simcat\ri(S_0)$ is a pushout of $S_0$ copies of the cofibrations $0\to 1$ and hence a cofibration. For $j>0$, if $y\of S_j$ then $Y_j\cap y$ is the boundary of $y$, hence $Y_{j-1}\to Y_j$ is a pushout of $S_j$ copies of the cofibration $\partial\Delta[j]\to\Delta[j]$ and hence a cofibration. Because $f:X\to Y$ is the colimit of the inclusions $X\to Y_j$ and those inclusions are compositions of cofibrations, $f$ is a cofibration.
\end{proof}

Proposition \ref{charcof} below shows that these locally decidable monomorphisms are all cofibration in $\sAsm$. The locally decidable monomorphisms include all pullbacks of cofibrations as the following lemma shows.

\begin{lemma} Locally decidable monomorphisms are stable under pullback and there is a classifier for locally decidable monomorphism in $\sAsm$. \label{lodeuni} \end{lemma}

\begin{proof} The category of assemblies $\Asm$ is extensive, so pullbacks preserve coproduct inclusions. Either coproduct inclusion $1\to 1+1$ is a generic decidable monomorphism. For every locally decidable monomorphism $f:X \to Y$ in $\sAsm$ the opfibration $\tot f$ is a pullback of $1:1\to \sier$. The underlying map of $\tot f$ is decidable and therefore a pullback of the underlying map of $1:1\to \sier$. Because $f$ is an opfibration the arrow map is a pullback too. So a morphism in $\sAsm$ is locally decidable if and only if it is a pullback of $\rat 1:\rat 1\to \rat \sier$.
%verwijs naar [1] als categorie
\end{proof}

\subsection{Factorization}
The category $\sAsm$ has to few colimits for the small object argument to work for arbitrary families of morphism, but we don't need it for cofibrations and acyclic fibrations. We heavily rely on the internal language of $\sAsm$ to show this (see subsection \ref{internal}).

\begin{prop} Every morphism $f:X\to Y$ factors as an acyclic fibration following a cofibration. \label{factor1} \end{prop}

\begin{proof} Let $Z$ be the following simplicial assembly. First let the assemblies $Z'[n]$ for $n\of\nno$ consist of quadruples $(a,b,c,d)$ where $a:[n]\to [p]$ is an epimorphism of $\simcat$, $b:\simcat/[p]\dual \to \sier$ is a functor, $c:\set b \to \tot X$ is a functor on the full subcategory $\set b$ of $\simcat/[p]\dual$ on the objects $\xi$ for which $b(\xi)=1$ and $d\of Y[p]$. 

Let $\top$ stand for the functor $\simcat/[n]\dual \to \sier$ which sends everything to $1$. Equivalently, $\top$ is the unique functor $\simcat/[n]\dual \to \sier$ satisfying $\top(\id_{[n]})=1$. If a simplex $(a,b,c,d)$ of $Z'$ has $b=\top$, then it represents a simplex of $X$. Let $Z[n]\subseteq Z'[n]$ consists of those quadruples $(a,b,c,d)$, where $a=\id$ whenever $b=\top$. This subobject exists because $b=\top$ is equivalent to $b(\id_{[n]}) = 1$, which is a decidable property of $b$.

In order to define the action of morphisms in the simplicial assembly, we use a case distinction for $b=\top$ and use the fact that in $\simcat$ every morphism $\phi:[m]\to[n]$ factors uniquely as a monomorphism $m(\phi)$ following an epimorphism $e(\phi)$. For every $\phi:[m]\to [n]$ and $(a,b,c,d)\of Z[n]$ let
\[ \phi\cdot (a,b,c,d) = \left\{\begin{array}{cc}
(\id,\top,a\circ\phi_\bang,\phi\cdot d) & b(\phi) = 1 \\
(e(a\circ \phi), b\circ m(a\circ \phi), c \circ m(a\circ \phi)_\bang, m(a\circ \phi)\cdot d ) & b(\phi)=0
\end{array}\right.
\]
Here $\phi_\bang:\simcat/[m]\dual \to \simcat/[n]\dual$ is defined as composition with $\phi$ and $m(a\circ \phi)_\bang$ is defined similarly.

There is a morphism $g:X\to Z$ satisfying $g[n](x) = (\id_{[n]},\top,x',f[n](x))$ where and $x':\simcat/[n]\dual \to x$ is the functor which sends $\xi$ to $\xi\cdot x$. This morphism is a cofibration by lemma \ref{Reedy}.

There is a morphism $h:Z\to Y$ satisfying $h[n](a,b,c,d) = a\cdot d$ and this map is an acyclic fibration for the following reasons. Let $z:\partial\Delta[n]\to Z$, $y:\Delta[n]\to Y$ and $h\circ z = z\circ k_n$ where $k_n:\partial\Delta[n]\to \Delta[n]$ is the inclusion of the boundary. The functor $\tot z: C[n]\to \tot Z$ is a system of simplices $z(\xi) = (a(\xi),b(\xi),c(\xi),d(\xi))$ for $\xi\of C[n]$. A filler $w:\Delta[n]\to Z$ corresponds to a simplex $z=(a,b,c,d)\of Z[n]$ for which $a\cdot d\of Y[n]$ corresponds with $y$ and where $\xi\cdot z=z(\xi)$ for all $\xi\of C[n]$. 

We make a case distinction.
\begin{itemize}
\item If $a(\xi)=\id$ for all monomorphisms $\xi$, then $a=\id$. Let the functor $b:\simcat/[n]\to \sier$ send all epimorphisms to $0$. For every other morphism $\phi:[i]\to [j]$, we let $b(\phi) = b(m(\phi))(e(\phi))$. If $b(\phi)=1$, then $c(\phi) = c(m(\phi))(e(\phi))$. Finally $d$ is the simplex in $Y[n]$ which correspond with the morphism $y$.

\item If $a(\xi)\neq \id$ for some monomorphisms $\xi$, then there is a greatest monomorphism $\mu:[m]\to[n]$ such that $\mu\neq \id$ but $a(\mu)=\id$. The reason is that there are finitely many monomorphisms to $[n]$ and the subcategory of those monomorphisms $\xi$ for which $a(\xi)=\id$ is closed under pushouts. In this case $a$ should be the unique inverse of $\mu$ for which $e(a\circ \xi) = a(\xi)$ for all monomorphisms. We then let $z = a\cdot z(\mu)$.
% laat b zich uitsluitend om mono's bekommeren. Scheelt nogal.
\end{itemize}

These two cases cover all possible commutative squares with the cycle inclusion $k_n$, because there are finitely many monomorphisms $\xi:[m]\to [n]$ and $a(\xi) = \id$ is a decidable property. For this reason, $h$ has the global right lifting property which respect to the family of all cycle inclusions. Therefore $h$ is an acyclic cofibration.
\end{proof} %dit werkt nog steeds!

%In the proof above, we take advantage of the fact that $\simcat$ is a \emph{Reedy category}. %cite??

The factorization in the proof above turns the implication in lemma \ref{Reedy} into an equivalence.

\begin{prop} A monomorphism $f:X\to Y$ is a cofibration if and only if it is decidable whether simplices of $Y$ are in $X$ and whether simplices outside of $X$are degenerate. \label{charcof} \end{prop}

\begin{proof} Lemma \ref{Reedy} shows the `if' direction. For `only if' factor $f:X\to Y$ as in the proof of lemma \ref{factor1} to get (another) cofibration $g:X\to Z$ and an acyclic fibration $h:Z\to Y$. Because $\id_Y\circ f = h\circ g$ there is a $k:Y\to Z$ such that $h\circ k = \id_Y$ and $k\circ f = g$ by the global lifting property. Reasoning internally, for each simplex $y$ of $Y$, $k(y)=(a,b,c,d)$. The simplex $y$ is in the image of $x$ if and only if $b=\top$ and this is decidable. A simplex $y$ for which $b\neq \top$ is nondegenerate if and only if $a=\id$ for the following reasons. Since $k$ is a section of $h$ it is a monomorphism and monomorphisms preserve nondegeneracy. The morphism $k$ commutes with the actions of morphisms in $\simcat$ and degenerates of $Z$ outside of the image of $g$ have $a\neq\id$.
\end{proof}

\section{Kan complexes}
Because acyclic cofibrations are less stable under pullback than general cofibrations, it is not clear we can factorize arbitrary morphisms as fibrations following acyclic cofibrations by a similar construction as the one we used in proposition \ref{factor1}. We retreat to the category $\Asm_{f}$ of fibrant objects and use the extra structure to get model structure here (theorem \ref{modelcat}).

\subsection{Pushout products}
The proof for a model structure on simplicial assemblies relies on the pushout-product construction, which preserves (acyclic) cofibrations.

\begin{defin} For each pair of morphisms $f:W\to X$ and $g:Y\to Z$ the \emph{pushout product} $f\otimes g$ is the unique factorization of the cospan $(f\times \id_Z , \id_X\times g)$ though the pushout of $f\times \id_Y$ and $\id_W\times g$. 
\[\xymatrix{
W\times Y \ar[r]^{f\times \id} \ar[d]_{W\times g} \po & X\times Y \ar[d] \ar[dr]^{f\times \id} \\
W\times Z \ar[r] \ar@/_2ex/[rr]_{f\times\id} & \bullet \ar[r]^(.4){f\otimes g} & X\times Z
}\]
\end{defin}

Because $\sAsm$ is cartesian closed, the pushout product with a fixed morphism has a right adjoint.

\begin{defin} For each pair of morphisms $f:W\to X$ and $g:Y\to Z$ the \emph{pullback exponential} $g^f: $ is the unique factorization of the span of the span $(Y^f, g^X): $ though the pullback of $Z^f$ and $g^W$. 
\[\xymatrix{
Y^X \ar@/^2ex/[drr]^{Y^f} \ar@/_2ex/[ddr]_{g^X}\ar[dr]^(.6){g^f} \\
& \bullet \ar[r]\ar[d]\pb & Y^W \ar[d]^{g^W} \\
& Z^X \ar[r]_{Z^f} & Z^W
}\]
\end{defin}

\begin{lemma}[Pushout product] If $f$ and $g$ are cofibrations, then so is $f\otimes g$. Moreover, if either $f$ or $g$ is acyclic then so is $f\otimes g$. \label{pushprod} \end{lemma}

\begin{proof} Reasoning internally, a simplex $(x,z)$ of $X\times Z$ is outside of the image $f\otimes g$ and nondegenerate if and only if both $x$ is outside of the image of $f$ and nondegenerate and $z$ is outside of the image of $g$ and nondegenerate. This is a decidable property of simplices of $X\times Z$ and hence $f\otimes g$ is a cofibration by proposition \ref{charcof}. 

Assume $f$ is acyclic. Let $e$ be an arbitrary fibration. The problem of filling a commutative square with $f\otimes g$ opposite to $e$ reduces to the problem of filling commutative squares with the pushout product $h\otimes k$ of a cycle inclusion $k:\partial\Delta[m]\to \Delta[m]$ and a horn inclusion $h:\Lambda_k
[n]\to\Delta[n]$ opposite to $e$. By standard simplicial homotopy, $h\otimes k$ is an acyclic cofibration (see \cite{Hovey99} section 3.3, \cite{GJSHT} section I.5). 

The reduction uses the pullback exponential. For each horn inclusion $h:\Lambda_k[n]\to \Delta[n]$, $e^h$ is an acyclic fibration, because the problem of lifting a cycle inclusion $k:\partial\Delta[m]\to \Delta[m]$ reduces to the problem of lifting the acyclic cofibration $h\otimes k$. Since $h\otimes g$ has the global left lifting property with respect to all fibrations, it is an acyclic fibration. For the fibration $e$ this implies that $e^g$ is a fibration, because the problem of lifting a horn $h$ against $e^g$ reduces the the problem of lifting $g$ against the acyclic fibration $e^h$. This means that $f$ has the left lifting property with respect to $e^g$. By generalization $f\otimes g$ has the left lifting property with respect to all fibrations, which means it is an acyclic cofibration.
\end{proof}

Lemma \ref{pushprod} has a counterpart for pullback exponentials.

\begin{corol} If $e:U\to V$ is a fibration and $f:W\to X$ is a cofibration, then $e^f$ is a fibration. Moreover, if either $f$ or $e$ are acyclic, then $e^f$ is acyclic. \label{pullexp}
\end{corol}

\subsection{Deformation retracts}
In $\sAsm$ \emph{weak equivalences} are morphisms which factor as acyclic fibrations following acyclic cofibration. For the model structure on Kan complexes a specific type of weak equivalences plays an important role.

\begin{defin} A \emph{deformation retract} is a morphism $f:X\to Y$ with a left inverse $g:Y\to X$ and a homotopy $h:\Delta[1]\times Y\to Y$ between $\id_Y$ and $f\circ g$, i.e.\ $h\circ (\Delta(\delta^1_0)\times \id_Y) = \id_Y$ and $h\circ (\Delta(\delta^1_1)\times \id_Y) = f\circ g$, where $\Delta\delta^1_i:1\to \Delta[1]$ are the points of $\Delta[1]$. %hopeloze terminology
\end{defin}

\begin{lemma} Let $f:X\to Y$ be a deformation retract with left inverse $g$ and homotopy $h$. If $f$ is a cofibration, it is an acyclic cofibration. If $g$ is a fibration, then it is an acyclic fibration. \label{drisac}\end{lemma}

\begin{proof} The morphism $f$ is a retract of $\Delta(\delta^1_1)\otimes f$.
\[\xymatrix{
&X \ar[r]^{\Delta(\delta^1_1)\times X}\ar[d]^f\po & \Delta[1]\times X \ar[dr]^{\pi_1} \ar[d] \\
X\ar[d]_f \ar@/^2ex/[urr]^(.3){\Delta(\delta^1_0)\times X} &Y \ar[r]\ar[dr]_(.4){\Delta(\delta^1_1)\times Y} \ar@/^3ex/[rr]^(.6)g & \bullet \ar[d]^{\Delta(\delta^1_1)\otimes f} \ar[r]_{(g,\pi_1)} & X\ar[d]^f \\
Y\ar[rr]_{\Delta(\delta^1_0)\times Y}&  & \Delta[1]\times Y \ar[r]_{h} & Y
}\]
Because $\Delta(\delta^1_1)$ is an acyclic cofibration, $f$ is an acyclic cofibration is it is a cofibration.

Thanks to the transpose $h^t:Y\to Y^{\Delta[1]}$ of $h$, $g$ is a retract of $g^{\Delta(\delta^1_1)}$.
\[\xymatrix{
Y\ar[d]_g \ar[r]^{h^t} & Y^{\Delta[1]} \ar[d]_{g^{\Delta(\delta^1_1)}} \ar[dr]^{Y^{\Delta(\delta^1_1)}} \ar[rr]^{Y^{\Delta(\delta^1_0)}}&& Y \ar[d]^g\\
X \ar[r]^{(X^\bang,f)} \ar[dr]_{X^\bang} \ar@/_3ex/[rr]_(.4)f & \bullet\ar[r]\ar[d]\pb & Y \ar[d]_g & X \\
& X^{\Delta[1]} \ar[r]_{X^{\Delta(\delta_1^1)}} \ar@/_2ex/[urr]_(.7){X^{\Delta(\delta^1_0)}}& X
}\]
Because $\Delta(\delta^1_1)$ is an acyclic cofibration $f$ is an acyclic fibration is it is a fibration.
\end{proof}

While acyclic fibrations and cofibrations are trivially weak equivalences, lemma \ref{drisac} does not show that the morphism $f$ and $g$ are weak equivalences. We will work on that now.

\begin{lemma} Let $f:X\to Y$ be a deformation retract with left inverse $g:Y\to X$ and homotopy $h$. The split monic $f$ factors as an acyclic fibration following an acyclic cofibration. \label{driswe} \end{lemma}

\begin{proof} By lemma \ref{factor1} $f$ factors as an acyclic fibration $a:Z\to Y$ and cofibration $b:X\to Z$. The morphism $b' = g\circ a$ is a left inverse of $b'$ and $a\circ b\circ b' = f\circ g\circ a$ is homotopic to $a$ thanks to $h\circ(\id_{\Delta[1]}\times a)$. The morphism $a$ reflects this homotopy because it is an acyclic fibration.

Let $c_1$ be the cycle $1+1\to \Delta[1]$. The pushout product $c_1\otimes b$ is a cofibration by lemma \ref{pushprod}. The domain of $c_1\otimes b$ is $W=\Delta[1]\times X +_{X+X} (Z+Z)$ and there is a morphism $d=(b\circ\pi_1,(\id_Z,b\circ b')):W\to Z$ such that $a\circ d = h\circ (\id_{\Delta[1]}\times a)\circ c_1\otimes b$.
\[\xymatrix{
X+X\ar[r]^{a+a} \ar[d]_{c_1\times \id} \po & Z+Z\ar[d] \ar@/^3ex/[drr]^{(\id,b\circ b')} \\
\Delta[1]\times X \ar[r]\ar[dr]_{\id\times g}\ar@/^4ex/[rrr]^{b\circ\pi_1} & W \ar[rr]_d \ar[d]^{c_1\otimes b} && Z\ar[d]^a \\
& \Delta[1]\times Z \ar[r]_{\id\times a} \ar@{.>}[urr] & \Delta[1]\times Y \ar[r]_h & Y
}\]
The filler $\Delta[1]\times Z\to Z$ is a homotopy between $\id_Z$ and $b\circ b'$. Lemma \ref{drisac} now tells us that $b$ is an acyclic cofibration.
\end{proof}

There is a conventional method for factoring morphisms between Kan complexes as fibrations following deformation retracts, which allows us to prove the following proposition, which brings a model structure on Kan complexes much closer.

\begin{prop} Every morphism $f:X\to Y$ between Kan complexes factors as a fibration following an acyclic cofibration. \label{factor2} \end{prop}

\begin{proof} Because $\bang_Y:Y\to 1$ is a fibration, the morphism $(d_0,d_1):Y^{\Delta[1]} \to Y\times Y$ defined by composition with the cycle inclusion $1+1\to \Delta[1]$ is a fibration and the components $d_i:Y^{\Delta[1]}\to Y$ are acyclic fibrations by corollary \ref{pullexp}. Pulling back $(d_0,d_1)$ along $f\times \id_Y$ produces the \emph{homotopy graph} $Y/f$ of $f$ together with projections $f_0:Y/f\to X$ and $f_1:Y/f\to Y$ where $f_0$ is an acyclic fibration because it is the pullback of $d_0$ and $f_1$ is a fibration because it is the composition of the fibrations $(f_0,f_1)$ and $X\times Y\to Y$.
\[\xymatrix{
Y/f \ar[r]\ar[d]_{(f_0,f_1)} \pb & Y^{\Delta[1]} \ar[d]^{(d_0,d_1)} \\
X\times Y \ar[r]_{f\times\id} & Y\times Y
}\]

There is a deformation retract $r=(\id_X,Y^\bang\circ f):X\to Y/f$ with $f_1$ as left inverse. By lemma \ref{driswe} $r = g\circ h$ for some acyclic fibration $g$ and some acyclic cofibration $h$. This means $f = (f_0\circ g)\circ h$ where $f_0\circ g$ is a fibration and $h$ is a cofibration.
\end{proof}

%hebben we de volgende nog nodig?


\hide{

\begin{lemma} For a cofibration $f:X\to Y$ between complexes the following are equivalent. \label{defret}
\begin{enumerate}
\item The cofibration $f$ is acyclic.
\item The cofibration $f$ is a deformation retract.
\end{enumerate}
\end{lemma}

\begin{proof}
Assume 1. Because $\bang_X:X\to 1$ is a fibration and $\bang_Y\circ f = \bang_X\circ \id_X$ there is a $g:Y\to X$ such that $f\circ g = \id$ by the global lifting property.

\[\xymatrix{
X\ar[d]_f\ar[r]^\id & X\ar[d] \\
Y\ar[r] \ar[ur]^g & 1
}\]

For standard reasons the morphism $d=(d_0,d_1):Y^{\Delta[1]}\to Y\times Y$ defined by composition with the cofibration $(\Delta(\delta^1_0),\Delta(\delta^1_1)):1+1 \to \Delta[1]$ is a fibration when $Y$ is a Kan complex. There is an inclusion $Y^\bang:Y\to Y^{\Delta[1]}$ defined by composition with $\bang_Y:Y\to 1$ and $d\circ Y^\bang:Y\to Y\times Y$ is the diagonal map. Therefore $(d_0,d_1)\circ Y^\bang\circ f = (\id_Y,f\circ g)\circ f$ and by the global lifting property there is a morphism $Y\to Y^{\Delta[1]}$ whose transpose $h:\Delta[1]\times Y\to Y$ is a homotopy between $f\circ g$ and $\id_Y$. Hence 1 implies 2.

\[\xymatrix{
X\ar[r]^f \ar[d]_f & Y \ar[r]^{Y^\bang} & Y^{\Delta[1]}\ar[d]^{d} \\
Y \ar[rr]_{(\id_Y,f\circ g)}\ar[urr]^{h^t} && Y\times Y
}\]

Assume 2. Let $S_0$ be the object of all 0-simplices of $Y$ which are not in the image of $X$. Let $X_0$ be the union of $X$ with all $y\of Y$ together with the paths $h(\cdot,y)$ between $y$ and $f(g(y))$. Since the inclusion of $X$ is a pushout of $S_0$ copies of the acyclic cofibration $\Delta(\delta^1_0)$, it is an acyclic cofibration.

The $p_j$ of the cycle $\partial\Delta[j]\to \Delta[j]$ and the horn $\Delta(\delta^1_0):1\to \Delta[1]$ is the boundary of the `prism' $\Delta[1]\times\Delta[j]$ minus one side. It is an acyclic cofibration for standard reasons (see \cite{Hovey99} lemma 3.3.3). %cite Hovey
\[\xymatrix{
\partial\Delta[j]\ar[r] \ar[d]\po & \Delta[1]\times\partial\Delta[j] \ar[d] \ar[dr] \\
\Delta[j]\ar[r]\ar@/_2ex/[rr] & \bullet \ar[r]^(.4){p_j} & \Delta[1]\times \Delta[j]
}\]

For each $j>0$ let $Y_j$ be the union of $X$ which $j$-simplices of $Y$ and let $S_j$ be the object of faces of $Y$ outside the image of $f$. For each $y:\Delta[j]\to Y$ in $S_j$, we have $f(g(y)):\Delta[j]\to Y_{j-1}$ and $h\circ \partial y:\Delta[1]\times \partial \Delta[j]\to Y_{j-1}$ because they consist either of simplices in $X$ or of less than $j$-dimensional simplices of $Y$. For this reason the inclusion of $Y_{j-1}\to Y_j$ is a pushout of a product of $S_j$ copies of $p_j$ and hence an acyclic cofibration. The morphism $X\to Y$ is a colimit of compositions of acyclic cofibrations and therefore an acyclic cofibration. Hence 2 implies 1.


\hide{For standard reasons the morphisms $d_i:Y^{\Delta[1]}\to Y$ defined by compositions with the acyclic cofibrations $\Delta\delta^1_i:1 \to \Delta[1]$ are acyclic fibrations. Thanks to the homotopy $h$, $g$ is a retract of $d_1$ and hence an acyclic fibration. Hence 2 implies 3.
\[\xymatrix{
Y\ar[d]_g \ar[r]_{h^t}\ar@/^2ex/[rr]^\id & Y^{\Delta[1]} \ar[r]_{d_0}\ar[d]_{d_1} & Y\ar[d]^g \\
X\ar[r]^f \ar@/_2ex/[rr]_\id & Y\ar[r]^g & X
}\]

Assume 3.} 

\end{proof}
}

\subsection{Weak equivalences}
To show that weak equivalences, fibrations and cofibrations form a model structure, we now only need to show that weak equivalences satisfy $2$-out-of-$3$, if we want to get a model structure.

\begin{lemma}[2-out-of-3] Let $f:X\to Y$ and $g:Y\to Z$ be morphisms of $\sAsm$. If any two of $f,g$ or $g\circ f$ are weak equivalences, then all three are. \end{lemma}

\begin{proof}
Compositions of acyclic fibrations are acyclic fibrations and the same holds for acyclic cofibrations. To show that all compositions of weak equivalences are weak equivalences, we just have to show that $g\circ f$ factors as an acyclic fibration following an acyclic cofibration, when $g$ is an acyclic cofibration and $f$ is an acyclic fibration.

By proposition \ref{factor1}, $g\circ f=h\circ k$ for some acyclic fibration $h:W\to Z$ and a cofibration $k:X\to W$. Since $Y$ is fibrant, $g$ has a left inverse $g'$ by the lifting property. Since $f \circ \id = g'\circ g\circ f= (g'\circ h)\circ k$ there is a morphism $k'$ such that $f\circ k' = g'\circ h$ and $k'\circ k = \id$, so $k$ has its own left inverse.
\[\xymatrix{
X\ar[d]_f \ar[r]_{k} \ar@/^2ex/[rr]^{\id} & W\ar[d]^h \ar[r]_{k'} & X\ar[d]^f\\
Y \ar[r]^{g} \ar@/_2ex/[rr]_{\id} & Z \ar[r]^{g'} & Y
}\]
The morphism $Z^{c_1}:Z^{\Delta[1]}\to Z\times Z$ defined by composition with the cycle $c_1:1+1\to\Delta[1]$, is a fibration because $Z$ is fibrant. The transpose of the filler of the commutative square $(\id_Z,g\circ g')\circ g = Z^{c_1}\circ (Z^\bang\circ g)$ is a homotopy $\chi$ between $\id_Z$ and $g\circ g'$. We lift $\chi$ to a homotopy of $\kappa:\Delta[1]\times W\to W$ between $\id_W$ and $k\circ k'$ using the pushout product $c_1\otimes k$ which is a cofibration by lemma \ref{pushprod}. 
The domain of $c_1\otimes k$ is $V = \Delta[1]\times X +_{X+X} (W+W)$ and there is a map $l=(k\circ \pi_1,\id_W,k\circ k'): V\to W$ such that $h\circ l= \chi\circ (\id_{\Delta[1]}\times h)\circ c_1\otimes k$. The lifting property induces the homotopy $\kappa$.
\[\xymatrix{
X+X\ar[r]^{k+k} \ar[d]_{c_1\times \id} \po & W+W\ar[d] \ar[drr]^{(\id,k\circ k')} \\
\Delta[1]\times X \ar[r]\ar[dr]_{\id\times k} & V \ar[rr]_l\ar[d]^{c_1\otimes k} && W\ar[d]^h \\
& \Delta[1]\times W \ar[r]_{\id\times k} & \Delta[1]\times Z \ar[r]_(.6)\chi & Z
}\] %hergebruik\dots maar hetzelf principe speelt een rol
\hide{overeenkomst: deformatie retract na acyclische fibratie is acyclische fibratie na acyclische cofibratie }

Lemma \ref{drisac} tells us that $k$ is acyclic because it is a deformation retract. So at this point we know that weak equivalences are closed under composition.

The second case we tackle is where $g$ and $g\circ f$ are weak equivalences. 

First assume that $g$ and $g\circ f$ are acyclic fibrations. By proposition \ref{factor1}, $f$ factors as an acyclic fibration $h:W\to Y$ following a cofibration $k:X\to W$. Because $(g\circ f)\circ \id = (g\circ h)\circ k$ and $g\circ h$ is an acyclic fibration, $k$ has a left inverse $k':X\to W$ which satisfies $g\circ f\circ k' = g\circ h$. The pushout product of the cycle $c_1:1+1\to \Delta[1]$ with $k$ is a cofibration by corollary \ref{pushprod}. The domain of $c_1\otimes k$ is the pushout $V=\Delta[1]\times X+_{X+X}(W+w)$ and there is a morphism $a=(k\circ \pi_1,\id_W,k\circ k'):V\to W$ such that $h\circ a = h\circ \pi_1\circ (c_1\otimes k)$. The filler $\Delta[1]\times W\to W$ is a homotopy between $\id_W$ and $k\circ k'$. By lemma \ref{drisac} $k$ is an acyclic cofibration and $f$ is a weak equivalence.

Next assume that $g$ and $g\circ f$ are acyclic cofibrations. The monomorphism $g$ reduces decisions on membership and degeneracy of a simplex $y$ in $Y$ to the same questions about $g(y)$ in $Z$. Therefore $f$ is a cofibration. Because $X$, $Y$ and $Z$ are fibrant, $g$ and $g\circ f$ are deformation retracts. Construction of left inverses and homotopies of $f$ from those of $g$ and $g\circ f$ is easy. They prove that $f$ is a deformation retract and acyclic by lemma \ref{drisac}.


In the general case where $g$ and $g\circ f$ are general weak equivalences, we factor $f$ as an acyclic fibration $h:W\to Y$ following a cofibration $k:X\to W$. Because weak equivalences are closed under composition, $g\circ h$ is a weak equivalence. We only need to show that $k$ is acyclic. Factor both $g\circ f$ and $g\circ h$ as acyclic fibrations following acyclic cofibrations, so $g\circ f = a\circ b$ and $g\circ h = c\circ d$. The lifting properties induce a morphism $l$ such that $l\circ b = d\circ k$ and $c\circ l = a$. The morphism $l$ is a weak equivalence because $a$ and $c$ are acyclic fibrations. Because of closure under composition, the morphism $l\circ b = c\circ k$ is both a weak equivalence and a cofibration and hence an acyclic cofibration. Since $c$ is an acyclic cofibration, so is $k$.

The last case is where $f$ and $g\circ f$ are weak equivalences.

First assume that $f$ and $g\circ f$ are acyclic cofibrations. The morphism $g$ factors as an fibration $h:W\to Z$ following an acyclic cofibration $k:Y\to W$ by lemma \ref{factor2}. The fibration $h$ has a right inverse $h'$ which satisfies $h'\circ g\circ f = k\circ f$ because $h\circ (k\circ f) = \id \circ (g\circ f)$. Because $k\circ f$ is an acyclic cofibration and $h'\circ h\circ k\circ f = k\circ f$, there is a homotopy between $h\circ h'$ and $\id_W$ for the following reasons. The morphism $W^{c_1}:W^{\Delta[1]}\to W\times W$ defined by composition with the cycle $c_1:1+1\to\Delta[1]$, is a fibration for standard reasons.  
There is a commutative square $W^{c_1}\circ W^{\bang}\circ k\circ f = (\id_W,h'\circ h)\circ (k\circ f)$ and the transpose of the filler $W\to W^{\Delta[1]}$ is the homotopy. Since $h:W\to Z$ is the left inverse of a deformation retract, lemma \ref{drisac} tells us it is an acyclic fibration.
%dat zou een heel goed lemma zijn. Pak daar eventueel andere horns erbij.

Next assume that $f$ and $g\circ f$ are acyclic fibrations. Let $k_n$ be the cycle $\partial\Delta[n]\to\Delta[n]$. Let $a:\partial\Delta[n]\to Y$ and $b:\Delta[n]\to Y$ satisfy $b\circ k_n = g \circ a$. Because $\partial\Delta[n]$ is cofibrant, there is an $a':\partial\Delta[n]\to X$ such that $f\circ a' = a$ and hence $(g\circ f)\circ a' = b\circ c$. There is a filler $d:\Delta[n]\to X$ for this commutative square. The morphism $f\circ d$ is a filler for the square $b\circ c_n = g \circ a$. This proves $g$ is an acyclic fibration.

In the general case where $f$ and $g\circ f$ are weak equivalences is now dual to the case where $g$ and $g\circ f$ are weak equivalences. This means that weak equivalences indeed satisfy 2-out-of-3.
\end{proof}

\hide{ Dat pushout products met de punten $1\to \Delta[1]$ acyclische cofibraties is het leeuwendeel van de standaardreden die langskomen.
Aan de andere kant kunnen we de volledige pushout product wellicht middels adjuncties uit de liftings van individuele acyclische cofibraties krijgen.
Pushout product preserves shit\dots preserves veel dankzij adjunctie. Maakt het werken met dependent products ook gemakkelijker.
}


This lead us to the next theorem of this paper.

\begin{theorem} The category $\sAsm_f$ of Kan complexes is a model category.\label{modelcat} \end{theorem}

\begin{remark} Theorem \ref{modelcat} belongs to the dependent type theory sketched in subsection \ref{internal}. This means that internally simplicial objects of toposes with natural number objects form model categories. \end{remark}

\section{Universe}
This section shows that $\rat\pers^S$ is a complex (theorem \ref{complex}) and that $\rat U^S$ is univalent in $\sAsm$ (theorem \ref{univalence}).

\subsection{Fibrancy}
Morphisms $\Delta[n] \to \rat\pers$ are transposes of functors $\simcat/[n]\dual \to \pers$. Similarly, for each horn $\Lambda_k[n]$ morphisms $\Lambda_k[n] \to \rat\pers$ are transposes of functors $H_k[n]\dual \to \pers$. 
The problem is to show that any $f:H_k[n]\dual \to \pers$ has an extension $g$ to $\simcat/[n]\dual$, such that the transpose of $g$ factors through $\rat\pers^S$, the category of algebras.

The lowest dimensional case where $n=1$ is special. The horns $\Delta(\delta^1_0),\Delta(\delta^1_1):1\to \Delta[1]$ are split monomorphisms, because they are sections of the map $!:\Delta[1]\to 1$. The map $!$ corresponds to the forgetful functor $\dom:\simcat/[1]\to\simcat$. We let $\dom^*(f)$ be the extension of each functor $f:\simcat\dual \to \pers$ along either $\delta^1_i$. This construction corresponds to sending a modest complex $X$ to the projection $\Delta[1]\times X \to \Delta[1]$. This map is trivially a fibration.

We present a construction which works for all $n>1$ below. This construction does not always produce fibrations for $n=1$, so we still need the construction above.

Let $n>1$. Let $H:H_k[n] \to \simcat/[n]$ be the inclusion. Composition determines a functor $H\ri:\pers^{\simcat/[n]\dual} \to \pers^{(H_k[n])\dual}$ and because $\pers$ is complete and $H\ri$ preserves all limits, this functor has a right adjoint $H_*:\pers^{(H_k[n])\dual}\to\pers^{\simcat/[n]\dual}$. More importantly, $H_*$ can be defined in such a way that it is a strict inverse of $H\ri$:
\[ H_*(f)(\phi) = \left\{\begin{array}{cc} 
f(\phi) & \phi\of (H_k[n])_0\\
\lim\limits_{\substack{\alpha\to \phi\\\alpha\of (H_k[n])_0}} f(\alpha) & \neg(\phi\of (H_k[n])_0)
\end{array}\right.\]
This is useful, because we are looking for an extension $g$ of $f$ such that $H\ri(g) = f$. Sadly, $H_*$ corresponds to the dependent product along $h:\Lambda_k[n] \to \Delta[n]$, which does not preserve fibrations.

\newcommand\norm[1]{\left\Vert#1\right\Vert}
The solution is that $g(\delta^n_{k})$, where $\delta^n_k$ is the face opposite to the point $k$, equals $H_*f(\id)$ i.e.\ the problematic simplices get a supporting edge over $k$. We extend $g$ to other objects $\xi:[m]\to[n]$ by adding more of these supporting edges. We use the internal language again (see subsection \ref{internal}).

Define the \emph{distance} of $\xi$ to $H_k[n]$ as follows.
\[ \norm\xi = \#\left( \prod_{\substack{i\of [n]\\i\neq k}}\set{p\of [m]|\xi(p)=i}\right)\]
Here, $\#$ stands for the number of elements in this finite set. The distance $\norm\xi$ is the number of ways $\delta^n_k$ factors through $
\xi$.

Next we define a functor $K:\simcat/[n]\to\simcat/[n]$. For each object $\xi:[m]\to[n]$ we let $K\xi:[m+\norm\xi]\to[n]$ satisfy:
\[ K\xi(i) = \left\{ \begin{array}{cl} 
\xi(i) & \xi(i)<k\\
k & \xi(i)\geq k, \xi(i-\norm\xi)<k\\
\xi(i-\norm\xi) & \xi(i-\norm\xi)\geq k
\end{array}\right.\]
If we view $\xi$ as a finite nondecreasing sequence then this functor simply adds $\norm\xi$ $k$'s to the sequence in such a way that the new sequence is still nondecreasing.

\hide{Symmetrie in de definitie is wel leuk, maar splitsen maakt het product te klein.}
In order to define $K$ for morphisms, we introduce some extra notation. For $\xi:[m]\to[n]$ and $i\of[n]$, let $\xi_i$ be the partial ordered set $\set{p\of[m]|\xi(p)=i}$. Using ordinal arithmetic, we get the following isomorphism:
\[ [m+\norm\xi] \simeq \sum_{i<k} \xi_i+\prod_{i\neq k}\xi_i+\sum_{i\geq k} \xi_i\]
Of course, $i\of[n]$. A morphism $\phi:\xi\to\xi'$ of $\simcat/[n]$ is a sequence of $n+1$ nondecreasing maps $\phi_i:\xi_i\to\xi'_i$ to which we apply the same construction:
\[ K\phi = \sum_{i<k} \phi_i+\prod_{i\neq k}\phi_i+
\sum_{i\geq k} \phi_i \]

Composition to the right defines a functor $K^*:\pers^{\simcat/[n]}\to\pers^{\simcat/[n]}$, which has a left adjoint $K_!$ because $\pers$ has all finite colimits.
To show that $K^*H_*$ preserves fibrations, we show that $K_!H^*$ preserves acyclic cofibrations.

\subsection{Preservation of acyclic cofibrations}
Suppose we have a horn inclusion $j:\Lambda_l[m] \to \Delta[m]$ and a morphism $\Delta(\xi):\Delta[m] \to\Delta[n]$. We will first show that $K_!$ sends this horn to a monomorphism.

\begin{lemma} Let $\delta^m_{pq}:[m-2]\to[m]$ for $p,q\of[m]$, $p<q$, be the unique nondecreasing map that only skips $p$ and $q$. Seen as subobject of $K(\xi)$, $K(\xi\circ\delta^m_{pq})$ is the intersection of $K(\xi\circ\delta^m_{p})$ and $K(\xi\circ\delta^m_{q})$. \label{intersection}\end{lemma}%n>1!

\begin{proof} Ordinal sums and products preserve pullbacks and therefore so does $K$. \end{proof} %is dit genoeg?

The morphism $\Delta(\xi):\Delta[m] \to \Delta[n]$ and $\Delta(\xi)\circ j:\Lambda_k[n]\to \Delta[n]$ are modest, which allows us to apply $K_!$ to them.

\begin{corol} The domain $K_!(\Lambda_k[n])$ of $K_!(\Delta(\xi)\circ j)$ is a subobject of $\Delta[m+\norm\xi]$, which is the domain of $K_!(\xi)$. \end{corol}

\begin{proof} The effect of $K_!$ on any map $\Delta(\chi):\Delta[m]\to\Delta[n]$ is straightforward: $K_!\Delta(\chi)=\Delta(K\chi)$. Because $K_!$ is a left adjoint, it preserves colimits. The morphism $\Delta(\xi)\circ j:\Lambda_k[n]\to\Delta[n]$ is a colimit of the diagram which consists of the objects $\Delta(\xi\circ\delta^m_{p})$ for $p\neq l$ and their intersections $\Delta(\xi\circ\delta^m_{pq})$. Since $K_!$ preserves these intersections by lemma \ref{intersection}, $K_!(\Delta(\xi)\circ j)$ is the union of $K_!(\Delta(\xi\circ\delta^m_p))$ for $p\neq l$. \end{proof}

We introduce some notation in order to describe the effect of $H^*K_!$.

\newcommand\face\delta %geen dimensie indicatie
\newcommand\ka\kappa
\newcommand\la\lambda
\newcommand\im{\exists_}
\begin{enumerate}
\item We want to keep track of the elements which $K$ adds to the domain of $\xi:[m]\to[n]$. For this we use a \emph{nondecreasing} injection $\ka$ which sends the product $\prod_{i\neq k}\xi_i$ to the interval in $[m+\norm\xi]$ which starts at the least $i$ such that $K\xi(i)=k$.
\item There is another injection $\la:[m]\to[m+\norm\xi]$ which skips the image of $\ka$: $\la(i)=i$ if $\xi(i)<k$ and $\la(i)=i+\norm\xi$ if $\xi(i)\geq k$.
\item We extend the face notation. For each $U\subseteq [m+\norm\xi]$ let $\face(U)$ be the face of $\Delta[m+\norm\xi]$ which is opposite to all points in $U$.
\item To apply $\ka$ and $\la$ to all elements of a subset of their domains we use $\im \ka$ and $\im\la$.
\end{enumerate}


We can describe the action of $K_!$ using unions of faces. Let $A = \im\ka(\prod_{i\neq k}\xi_i)$ and for each $q\of[m]$ let:
\[ A_q = \set{\kappa(\vec p)\of A\middle|\xi(q) \neq k, p_{\xi(q)}=q } \]
The set $A_q$ contains the supporting edges of $\xi$ which are not supporting edges for $\xi\circ\delta^m_q$. Therefore, the functor $K$ sends the face $\delta^m_p:\xi\circ\delta^m_p \to \xi$ to $\face(A_p\cup\set{\lambda(p)})$. Preserved unions imply
\[ K_!(\Lambda_k[m]) = \bigcup_{p\neq l} \face(A_p\cup\set{\lambda(p)})\]

The effect of $H\ri$ is also easy to describe in terms of unions of faces.
\begin{align*}
H\ri K_!(\Delta[m]) &= \bigcup_{i\neq k} \face(\im\la(\xi_i)) \\
H\ri K_!(\Lambda_l[m]) &= \bigcup_{\substack{i\neq k\\p\neq l}} \face(\im\la(\xi_i)\cup A_p\cup\set{\lambda(p)})
\end{align*}

We first proof a technical lemma about acyclic cofibrations.

\begin{lemma}[Face completion] Let $F$ be an inhabited decidable set of faces of $\Delta[p]$ which all have an edge $e$ in common. The inclusion $\bigcup F\to \Delta[p]$ is an acyclic cofibration. \label{facecom} \end{lemma}

\begin{proof} For all $j\of[p]$ let $F_j$ be the union of $F$ with the set of $j$-dimensional faces of $\Delta[p]$ which contain the edge $e$. Because $F$ is inhabited, $\bigcup F$ contains $e$ and therefore $F_0=F$. Because $\Delta[p]$ is a $p$-dimensional face of $\Delta[p]$ which contains $e$, $\bigcup F_p = \Delta[p]$. For $j>0$ let $S_j$ be the set of $j$-dimensional faces of $\bigcup F_j$ which are not already contained in $\bigcup F_{j-1}$. If a $j$-dimensional face $\face(\Sigma)$ of $\bigcup F_j$ opposes $e$, it is part of a higher dimensional face which is a member of $F$. Therefore each face $\face(\Sigma)\of S_j$ contains $e$. For this reason $\face(\Sigma)\cap \bigcup F_{j-1}$ is the horn whose central edge is $e$. The inclusion $\bigcup F_{j-1}\to\bigcup F_j$ is therefore the pushout of a coproduct of horn inclusions indexed over $S_j$ and therefore an acyclic cofibration. Because acyclic cofibrations are closed under composition, $\bigcup F = F_0\to F_p = \Delta[p]$ is an acyclic cofibration. 
\end{proof}


\begin{lemma}[Descent] The inclusion $H\ri K_!(\Lambda_l[m]) \to H\ri K_!(\Delta[m])$ is an acyclic cofibration. \end{lemma}


\begin{proof} If $\norm\xi=0$ and hence $\prod_{i\neq k}\xi_i=\emptyset$, then neither $K_!$ nor $H^*$ change anything about the horn $\Lambda_l[m]\to\Delta[m]$, so we only need to worry about the cases where $\norm\xi>0$.

We will add the faces $U_i = \face(\im\la(\xi_i))$ of $H\ri K_!(\Delta[m]) = \bigcup_{i\neq k} \face(\im\la(\xi_i))$, saving the most difficult case $i=\xi(l)$ for last.

Note that because $n>1$, there are always $i\of[n]$ such that $i\neq k$, $i\neq \xi(l)$. The intersection of $H\ri K_!(\Lambda_l[m])$ with $\face(\im\la(\xi_i))$ is:
\[ H\ri K_!(\Lambda_l[m])\cap U_i = \bigcup_{p\neq l} \face(\im\la(\xi_i)\cup A_p\cup\set{\la(p)})\]
The set $F = \set{\face(\im\la(\xi_i)\cup A_p\cup\set{\la(p)})|p\neq l}$ is inhabited and decidable. Each face in $F$ contains the edge $\la(l)$. Hence the inclusion $H\ri K_!(\Lambda_l[m])\cap U_i\to U_i$ is an acyclic cofibration by lemma \ref{facecom}.

Let $L$ be the union of $H\ri K_!(\Lambda_l[m])$ with $U_i$ for all $i\of[n] -\set{k, \xi(l)}$. The inclusion $H\ri K_!(\Lambda_l[m])\to L$ is a pushout of a coproduct of acyclic cofibrations indexed over $[n]-\set{k,\xi(l)}$ and hence is an acyclic cofibration. If $\xi(l)=k$, then $L=H\ri K_!(\Lambda_l[m])$ and we are done. Otherwise we still have to deal with the face $\face(\im\la(\xi_{\xi(l)}))$.

Let $V_p =\face(\im\la(\xi_{\xi(l)}\cup\set p)\cup A_p)$. If $p\neq l$ these are faces of $U_{\xi(l)}$ which are part of $L$. Hence $L$ is the following union of faces.
\[ L = \left(\bigcup_{i\of[n]-\set{k,\xi(l)}} U_i\right)\cup\left(\bigcup_{p\of[m]-\set l} V_{\set p}\right) \] 

For each $\vec p\of\prod_{i\of[n]-{k}} \xi_i$ let $B_{\vec p} = \set{q\of[m]\middle | \xi(q)\of[n]-\set{k,\xi(l)} , q>p_{\xi(q)} }$ and let $W_{\vec p} = \face(\im\la(\xi_{\xi(l)}\cup B_{\vec p}))$. For all $j\of\nno$, let $L_{j} = L\cup \bigcup_{ \ka(\vec p) < j } W_{\vec p}$. By this definition $L_0=L$ and $L_{m+\norm\xi+1} = H^*K_!\Delta[m]$ because $B_{\vec p}=\emptyset$ and therefore $W_{\vec p} = U_{\xi(l)}$ if $p_i$ are the maximal elements of $\xi_i$ for each $i\of[n]-\set k$. 

For every $j\of\nno$ the inclusion $L_j\to L_{j+1}$ is an acyclic cofibration for the following reasons.

As long as $K\xi(j)<k$, $L_j = L$ because $\bigcup_{ \ka(\vec p) < j } W_{\vec p}$ is empty. If $K\xi(j+1)<k$ too, then $L_j\to L_{j+1}$ is an acyclic cofibration because it is an identity.

If $K\xi(j+1) = k$ then $j+1=\kappa(\vec p)$ or $j+1 = \la(p)$ for $p\of\xi_k$. We first consider the case that $j+1=\ka(\vec p)$. If $\vec p,\vec q\of\prod_{i\neq k} \xi_i$ and $p_i\leq q_i$ for all $i\of[n]-k$, then $\vec p \leq \vec q$ in the lexicographical order of the ordinal product and hence $\ka(\vec p)\leq \ka(\vec q)$. Therefore $W_{\vec p}\subseteq L_{\ka(\vec q)}$. For that reason, the intersection $L_j\cap W_{\vec p}$ is the union of the following families of faces.
\begin{align*}
U_i\cap W_{\vec p} &= \face(\im\la(\xi_i\cup\xi_{\xi(l)}\cup B_{\vec p})) &&\textrm{ for $i\of[n]-\set{k,\xi(l)}$}\\
V_q\cap W_{\vec p} &= \face(\im\la(\xi_{\xi(l)}\cup\set p\cup B_{\vec p})) &&\textrm{ for $q\of[m]-\set l$}\\
W_{\vec r}\cap W_{\vec p} &= \face(\im\la(\xi_{\xi(l)}\cup B_{\vec r}\cup B_{\vec p}))&&\textrm{ if $\kappa(\vec r)<\kappa(\vec p)$ }
\end{align*}
Let $\vec p[l]\of\prod_{i\neq k} \xi_i$ satisfy $\vec p[l]_i = l$ if $i=\xi(l)$ and $\vec p[l]_i = p_i$ if $i\neq \xi(l)$. The intersection $L_j\cap W_{\vec p}$ is a union of faces which contain the supporting point $\kappa(\vec p[l])$ for the following reasons. The supporting edge $\kappa(\vec p[l])$ is a member of $U_i$ and $W_{\vec r}$ because those faces are only opposed to edges in the images of $\la$. The faces $V_q$ contains $\kappa(\vec p[l])$ if $\kappa(\vec p)\not\of A_q$. If $\xi(q) = k$, then $A_q=\empty$ and if $\xi(q) = \xi(l)$, then $\ka(\vec p[l])$ is not of type $A_q$ because $\ka(\vec p[l])\of A_l$ and $A_l$ and $A_q$ are disjoint because $q\neq l$. Otherwise, $q = p_{\xi(q)}$ by definition of $A_q$. 

Either $\xi(q-1) = \xi(q)$ or $q$ is the least member of $\xi_{\xi(q)}$. If $\xi(q-1)=\xi(q)$ let $\vec p[q-1] \of \prod_{i\neq k} \xi_i$ satisfy $\vec p[q]_\xi(q) = q-1$ if $i=\xi(l)$ and $\vec p[\xi(q)]_i = p_i$ if $i\neq \xi(q)$. By definition, $B_{\vec p[q-1]} = B_{\vec p} \cup \set q$ and therefore $V_q\cap W_{\vec p}\subseteq W_{\vec p[q-1]}$. As noted before $\ka(\vec p[q-1])<\ka(\vec p)$, so $W_{\vec p[\vec p]}\subseteq L_{j}$. 
If $q$ is the least member of $\xi_{\xi(q)}$ then $\xi_{\xi(q)}\subseteq B_{\vec p} \cup \set q$ and therefore $V_q\cap W_{\vec p}\subseteq U_{\xi(q)}$. 

In all cases where $V_q$ opposes $\ka(\vec p[l])$, some other face of $L_j\cap W_{\vec p}$ contains both $V_q$ and $\vec p[l]$. Therefore $L_j\cap W_{\vec p}$ is a union of faces which contain the supporting edge $\ka(\vec p[l])$. By lemma \ref{facecom} $L_j\cap W_{\vec p}\to W_{\vec p}$ is an acyclic cofibration and so is $L_j\to L_{j+1}$.

As $\kappa(\vec p)$ grows, $B_{\vec p}$ shrinks to an empty set. By the time $j+1 = \la(p)$ for some $p\of \xi_k$, $L_j = H^*K_!\Delta[m]$ and $L_j\to L_{j+1}$ is the identity. The same holds for all $L_j\to L_{j+1}$ where $K\xi(j+1)>k$ and where $j+1>m+\norm\xi$.

Since acyclic cofibrations are closed under composition, $L\to H^*K_!\Delta[m]$ and $H^*K_!\Lambda_l[m] \to H^*K_!\Delta[m]$ are acyclic cofibrations.
\end{proof}%here

\begin{remark} The seemingly classical reasoning above is actually constructive because we are working with finite and decidable sets of number--or equivalently with functions $\nno \to \set{0,1}$--in every case. The proof does not rely on quirks of the category of assemblies. It shows that Kan fibrations \emph{descent} along horn inclusions in the dependent type theory sketched in subsection \ref{internal}. \end{remark}

\begin{theorem} The simplicial assembly $\rat\pers^S$ is a complex. \label{complex} \end{theorem}

\begin{proof} The two lemmas above show that each map $f:\Lambda_k[n] \to \rat\pers^S$ has an an extension $K^*H_!f:\Delta[n] \to \rat\pers^S$. The construction is sufficiently constructive to turn into an algebra structure $S(\rat\pers^S) \to \rat\pers^S$. \end{proof}

\subsection{Univalence}
A fibration $f:X\to Y$ is \emph{univalent} if weakly equivalent pullbacks of $f$ are homotopic. For the generic modest fibration $\sigma:\tilde U \to U$ this means we can turn a weak equivalence of modest complexes $w:X\to Y$ into a modest fibration $W\to \Delta[1]$ such that the pullbacks $\Delta(\delta^1_0)\ri(W) = Y$ and $\Delta(\delta^1_1)\ri(W) = X$--such a fibration is a \emph{correspondence} between $X$ and $Y$. More precisely, for each pair of functors $f,g:\simcat\dual\to\pers$ which correspond to modest complexes and each natural transformation $w:f\to g$ which corresponds to a weak equivalence there should be a functor $h:\simcat/[1]\dual \to \pers$ such that the composition with the constant inclusions $(\delta^1_i)_!:\simcat\dual\to \simcat/[1]\dual$ are equal to $f$ and $g$. 

%Because the factorization system are limited to modest simplicial sets, we can only show that the generic modest morphism satisfy this property for pullbacks along morphisms $f:X\to \rat\pers^S$ whose domains are modest. We call this property \emph{modest univalence}.

We reason internally again. The objects of $\simcat/[1]$ are morphisms $\chi:[n]\to [1]$. We can think of these morphisms as pairs $([n],[i])$ for $i\geq -1$, $i\leq n$, where $[i]\subseteq [n]$ is an initial segment and $[-1]$ stands for the empty initial segment. A morphism $\phi:\chi\to\chi'$ corresponds to a pair of morphism $(\phi:[n]\to[n'],\psi:[i]\to [i'])$ where $\psi$ is a restriction of $\phi$. 

\newcommand\hcg{\mathsf{hcg}}
For each pair $f,g:\simcat\dual\to\pers$ and each natural transformation $\iota:f\to g$, we define the \emph{homotopy cograph} $\hcg(\iota):\simcat/[1]\dual \to \pers$ as follows. On objects:
\[ \hcg(\iota)([n],[i]) = \left\{\begin{array}{cc}
f([n]) & i=n \\
\set{(x,y)\of g(n)\times f(i)| x\cdot [i] = \iota(y)} & -1< i < n \\
g([n]) & i=-1
\end{array}\right.\]
Here $x\cdot [i]$ is the restriction of $x$ along the inclusion $[i]\to [n]$ of $[i]$ as an initial segment of $[n]$.

Let $(\phi,\psi):([m],[i]) \to ([n],[j])$ be any morphism. Note that the following restrictions apply: if $j=-1$, then $i=-1$; if $j=n$, then $i=n$. The reason is that $(\delta^1_0,\delta^1_1):1+1\to \Delta[1]$ induces a sieve on $\simcat/[1]$. With those in mind, let:
\[ \hcg(\iota)(\phi,\psi) = \left\{\begin{array}{cc}
f(\psi) & j=n\\
f(\psi)\circ \pi_1 & -1 < j < n, i=m\\
g(\phi) \times f(\psi) & -1<i<m\\
g(\phi)\circ\pi_0 & -1< j < n, i=0\\
g(\phi) & j=0
\end{array}\right.\]

The graph has $f$ gradually fading out as $i$ counts down to $0$. The functor $\hcg(\iota):\simcat/[1]\dual\to\pers$ satisfies $\hcg(\iota)\circ(\delta^1_1)_! = f$ and $\chi(\iota)\circ(\delta^1_0)_! = g$. Therefore it induces a modest fibration $\hcg(\iota)\ri(\vec \mu): Z\to \Delta[1]$ such that $f\ri(\vec \mu)$ and $g\ri(\vec \mu)$ are pullbacks along $\Delta(\delta_i^1)$.

\begin{lemma} If $\iota:f\to g$ induces a weak equivalence of modest fibrations, then $\hcg(\iota)$ factors through $\rat\pers^S$. \end{lemma}

\begin{proof} We start with an analysis of what happens to horns in this construction.
Let $n>0$ and $i,k\leq n$. These are the indices for the horn $\Lambda_k[n] \to \Delta[n]$ together with a morphism $\Delta([n],[i]):\Delta[n]\to\Delta[1]$. 

If we pullback the horn along $\Delta(\delta^1_1):1\to \Delta[1]$, then we get a monomorphism which we express as an inclusion unions of faces of $\Delta([n])$ to get the following formula:
\begin{equation}
\bigcup_{\substack{q\leq n\\q\neq k}} \face(\set{p\of[n] | i < p}\cup\set{q}) \to \face\set{p\of[n] | i < p}
\end{equation}
\begin{itemize}
\item For $i<n-1$, both sides equal $\face\set{p\of[n] | i < p}$, because $\set{p\of[n] | i < p}\cup\set{n}$ and $\set{p\of[n] | i < p}\cup\set{n-1}$ equal $\set{p\of[n] | i < p}$ and $k$ is unequal to either $n-1$ or $n$. Hence the pullback of the horn is the acyclic cofibration $\id_{\Delta[i]}: \Delta[i]\to\Delta[i]$, or, if $i=-1$ the acyclic cofibration $\id_0:0\to 0$.
\item For $i=n-1$ and $k<n$, both sides are the face $\face\set{n}$ and the pullback of the horn is the acyclic cofibration $\id_{\Delta[n-1]}: \Delta[n-1]\to\Delta[n-1]$. For $k=n$ however, we find that the pullback is the cycle $\partial \Delta[n-1] \to \Delta[n-1]$.
\item For $i=n$ the pullback is all of $\Lambda_k[n] \to \Delta[n]$.
\end{itemize}

With this analysis we can tackle the problem of lifting horns along $\hcg(\iota)\ri(\vec\mu):Z\to \Delta[1]$ and this hinges on the morphism $([n],[i]):\Delta[n] \to \Delta[i]$ just like the definition of $\hcg(\iota)$ does.

\begin{enumerate}
\item if $i=-1$, then filling $\Lambda_k[n] \to \Delta[n]$ in $\hcg(\iota)\ri(\vec\mu)$ reduces to filling it in $g\ri(\vec\mu)$.
\item if $i=n$, then filling $\Lambda_k[n] \to \Delta[n]$ in $\hcg(\iota)\ri(\vec\mu)$ reduces to filling it in $f\ri(\vec\mu)$.
\item if $-1<i<n$, then filling $\Lambda_k[n] \to \Delta[n]$ in $\hcg(\iota)\ri(\vec\mu)$ reduces to filling it in $g\ri(\vec\mu)$ and filling its pullback along $\Delta([n],[i])$ in $f\ri(\vec\mu)$ in a commutative way.
\end{enumerate}

We can do $1$ and $2$ because $f,g$ are complexes. The third $3$ is mostly not a problem, because the pullbacks are identities. There is just one exception: $k=n$ and $\Delta([n],[n-1])$. 
To fill this cycle in $f$, in a way that is consistent with the filler of the horn in $g$, we use the fact that $\iota$ factors an acyclic fibration following an acyclic fibration. For these two types of maps, we have an elegant solution.

Firstly assume that $\iota$ is an acyclic fibration. When we fill $\Lambda^n_n$ in $g\ri(\vec\mu)$, we get an $n-1$ simplex opposite to the edge $n$. The cycle $\partial\Delta[n-1] \to \Delta[n-1]$ commutes with the acyclic fibration $\iota\ri(\vec\mu)$, which means there is a filler for is. That way we fill $\partial\Delta[n-1]$ in $f\ri(\vec\mu)$ in a way that commutes with the filler of $\Lambda^n_n$ in $Y$.
\[\xymatrix{
\partial\Delta[n-1] \ar[d]\ar[dr]\ar[rr] &  & f\ri(\vec\mu)\ar[d]^{\iota\ri(\vec\mu) }\\
\Delta[n-1] \ar@{.>}[urr]\ar[dr]_{\Delta(\delta^n_n)} & \Lambda_n[n] \ar[d]\ar[r] & g\ri(\vec\mu) \ar[d]^\bang \\
& \Delta[n] \ar@{.>}[ur]\ar[r] & 1
}\]

Secondly assume that $\iota$ is an acyclic cofibration. Because $f\ri(\vec\mu)$ is a complex, $\iota$ has a right inverse $\iota'$. After filling the horn $\Lambda_n[n]$ in $Y$ we use this right inverse to find a suitable filler in $X$.
\[\xymatrix{
f\ri(\vec\mu)\ar[d]_{\iota\ri(\vec\mu)} \ar[r]^\id & f\ri(\vec\mu) \ar[d] \\
g\ri(\vec\mu) \ar[r] \ar@{.>}[ur]^{\iota'} & 1 
}\]

Since every weak equivalence $\iota$ factors as an acyclic fibration following an acyclic cofibration, $\hcg(\iota)$ is a Kan complex. \end{proof}

\hide{
Is de cograaf niet een manier om een factorisatie te geven in fibraties en cofibraties? Acyclische fibraties dan? Kunnen we dat niet toepassen om een categorie van cofibrante objecten te krijgen?

Het gaat om het morfisme $Z\to Y$. We zien dat het vuller van horns in $Z$ meestal wordt geredceert tot hetzelfde probleem in $Y$, maar nu eisen we dat de horn in $Y$ er al is\dots
Het uiterste geval met $X$ is dan nog wel een probleem
}

\begin{theorem} The generic modest fibration $U^S$ is univalent. \label{univalence}\end{theorem}

\begin{proof} For every pair $f,g:X\to \rat\pers^S$ and each homotopy equivalence $h:f\ri(U^S) \to g\ri(U^S)$ we get a homotopy $k:\simcat/[1]\times X \to \rat \pers^S$ between $f$ and $g$ by applying the $\hcg$ construction pointwise for the various points of $X$. 
\end{proof}


\section{Conclusion}
We have established that there is a univalent generic modest Kan fibration in the model category of Kan complexes $\sAsm_f$. This may be a useful model of homotopy type theory.

Before ending this paper, want want to add a few interesting observations.

\subsection{Dependent types}  The following lemma show that we can interpret dependent types in the category of Kan complexes $\sAsm_f$. It belong to the type theory of subsection \ref{internal}.

\begin{prop} Fibrations are closed under dependent products up to weak equivalence. \end{prop}

\begin{proof} Acyclic cofibration are stable under pullback along fibrations with cofibrant domains for the following reasons. Cofibrant objects are simplicial assemblies where degeneracy is decidable. Cofibrations are locally decidable monomorphisms and locally decidable monomorphisms with cofibrant domains are cofibrations by proposition \ref{charcof}. Therefore the pullback of a cofibration along a morphism with a cofibrant domain is a cofibration. Since $\sAsm_f$ is proper, pullbacks along fibrations preserve weak equivalences and pullbacks along fibrations with cofibrant domains preserve acyclic cofibrations. This implies that dependent products of fibrations along fibrations with cofibrant domains are fibrations.

For each morphism $f:X\to Y$ splits as a fibration $f_1:Y/f\to Y$ following a deformation retract $r:X\to Y/f$ with inverse $f_1$ and $Y/f$ has a cofibrant replacement $c:Z\to Y/f$. The homotopical dependent product of a fibration $g:W\to X$ along $f$ is $\widetilde{\prod_f}(g) = \prod_{f_1\circ c}(f_1\circ c)\ri(g)$.
\end{proof}

\newcommand\ex{_{\rm ex}}
\subsection{Exact completions}
%In the effective topos, modest morphisms are still closed under dependent products, but the generic modest morphism is no longer generic and dependent products fail to satisfy the Beck-Chevalley condition over some pullback squares \cite{MR1023803}. So $\pers$ is only complete in a weakened sense. 
The constructions in this paper allow us to show that toposes can have complete internal categories in a strong sense.

\begin{prop} The exact completion $\Asm\ex$ of $\Asm$ \dots
\begin{enumerate}
\item It is a topos. 
\item It has a generic modest morphism.
\end{enumerate}
\end{prop}

\begin{proof}
For 1. see \cite{MR1981211}. In order to show 2., we a category of truncated simplicial assemblies.

Let $\simcat_0$ be the full subcategory of $\simcat$ on the objects $[0]$ and $[1]$. The category $\Asm^{\simcat_0\dual}$ is the category of $0$-truncated $n$-simplices.

Let $Z:\simcat_0\to\simcat$ be the inclusion functor. It induces a functor $Z\ri:\sAsm \to \Asm^{\simcat_0\dual}$. If we apply $Z\ri$ to cycle and horn inclusions we find that the higher dimensional ones become isomorphisms. The only cycle inclusions are which don't turn into isomorphisms are $0\to 1$, $1+1 \to \Delta[1]$. For horn inclusions we are left with $\Delta(\delta^1_i):1\to \Delta[1]$ and $\Lambda_i[2] \to \Delta[2]$.

The injective objects, i.e.\ the complexes, for these 5 horn inclusions are precisely the \emph{pseudoequivalence relations} of \cite{MR1600009}. The category $\Asm\ex$ therefore has the same objects as the full subcategory of complexes $\Asm^{\simcat_0}_f\subseteq \Asm^{\simcat_0}$. Morphisms of $\Asm\ex$ are equivalence classes of morphisms in $\Asm^{\simcat_0}_f$ for the relation of being homotopic. 

We have a generic modest injective in $\Asm^{\simcat_0}_f$ by theorem \ref{geninjmod}. Pullbacks of fibrations are preserved by the quotient functor $Q:\Asm^{\simcat_0}_f \to\Asm\ex$ because they are examples of homotopy pullbacks \cite{GJSHT}. Therefore $Q(\rat U^S)$ is a generic family of modest sets in $\Asm\ex$.
\end{proof}

\begin{remark}[Lifting power revisited] If we use the local rather than the global lifting property (see remark \ref{liftingpower}) in the definition of fibrations in $\Asm^{\simcat_0\dual}$, then the homotopy category is equivalent to the effective topos. The effective topos has no generic modest morphism \cite{MR1023803}. %wrong Awodey
\end{remark}

\subsection{Building factorization systems}
A nice way to get an anodyne-injective factorization system is to generate a free monoid from the functor $S$ in subsection \ref{IAA} in the monoidal category of functors $\sAsm \to\sAsm$.  Garner shows that this is possible in the category of simplicial sets in \cite{MR2899720}, which is a reflective subcategory of $\sAsm$. The algebraic completeness of $\rat \pers$ makes it easy to find the free monoid of $S$ there (see remark \ref{modfac}). I could not figure out how to combine these two constructions into one.

\subsection*{Acknowledgments} 
I am grateful to the Warsaw Center of Mathematics and Computer Science for the opportunity to write this paper. I am also grateful for discussions with Marek Zawadowski and the seminars on simplicial homotopy theory he organized during my stay at Warsaw University. Richard Garner, Peter LeFanu Lumsdaine and Thomas Streicher made invaluable comments on early drafts of this paper.

\bibliographystyle{alpha}

\bibliography{realizability}{}


\end{document}


